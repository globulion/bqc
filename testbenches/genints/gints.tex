% FWEAVE v1.62 (September 23, 1998)

% --- FWEB's macro package ---
\input fwebmac.sty

% --- Initialization parameters from FWEB's style file `fweb.sty' ---
\Wbegin[;]
  % #1 --- [LaTeX.class.options;LaTeX.package.options]
{article;}
  % #2 --- {LaTeX.class;LaTeX.package}
{1em}
  % #3 --- {indent.TeX}
{1em}
  % #4 --- {indent.code}
{CONTENTS.tex}
  % #5 --- {contents.TeX}
{ % #6 ---
 {\&\WRS}
  % #1 --- {{format.reserved}{format.RESERVED}}
 {\|}
  % #2 --- {format.short_id}
 {\>\WUC}
  % #3 --- {{format.id}{format.ID}}
 {\>\WUC}
  % #4 --- {{format.outer_macro}{format.OUTER_MACRO}}
 {\>\WUC}
  % #5 --- {{format.WEB_macro}{format.WEB_MACRO}}
 {\@}
  % #6 --- {format.intrinsic}
 {\.\.}
  % #7 --- {{format.keyword}{format.KEYWORD}}
 {\.}
  % #8 --- {format.typewriter}
 {}
  % #9 --- (For future use)
}
{\M}
  % #7 --- {encap.prefix}
{;}
  % #8 --- {doc.preamble;doc.postamble}
{INDEX}
  % #9 --- {index.name}


% --- Beginning of user's limbo section ---

\def\title{--- INTEGRAL STORAGE AND PROCESSING ---}





% --- Limbo text from style-file parameter `limbo.end' ---
\FWEBtoc

\WN1.  getint. This function withdraws $(ij,kl)$ two-electron integral
from the \WCD{ \.{file}}.

\WY\WP \Wunnamed{defs}{gints.f}%
\WMd{}\WUC{ARB}\5
$\WO{1}$\Wendd
\WP\WMd{}\WUC{YES}\5
$\WO{0}$\Wendd
\WP\WMd{}\WUC{NO}\5
$\WO{1}$\WY\Wendd
\WP\WMd{}\.{ERR}\5
${-}\WO{10}$\Wendd
\WP\WMd{}\WUC{OK}\5
$\WO{10}$\WY\Wendd
\WP\WMd{}\WUC{BYTES\_PER\_INTEGER}\5
$\WO{4}$\Wendd
\WP\WMd{}\WUC{LEAST\_BYTE}\5
$\WO{1}$\WY\Wendd
\WP\WMd{}\WUC{END\_OF\_FILE}\5
${-}\WO{1}$\WY\Wendd
\WP\WMd{}\WUC{NO\_OF\_TYPES}\5
$\WO{20}$\Wendd
\WP\WMd{}\WUC{INT\_BLOCK\_SIZE}\5
$\WO{20}$\WY\Wendd
\WP\WMd{}\WUC{LAST\_BLOCK}\5
$\WO{1}$\Wendd
\WP\WMd{}\WUC{NOT\_LAST\_BLOCK}\5
$\WO{0}$\WY\Wendd
\WP\WMd{}\WUC{ERROR\_OUTPUT\_UNIT}\5
$\WO{6}$\WY\Wendd
\WP\WMd{}\WUC{MAX\_BASIS\_FUNCTIONS}\5
$\WO{255}$\Wendd
\WP\WMd{}\WUC{MAX\_PRIMITIVES}\5
$\WO{1000}$\Wendd
\WP\WMd{}\WUC{MAX\_CENTRES}\5
$\WO{50}$\WY\Wendd
\WP\WMd{}\WUC{MAX\_ITERATIONS}\5
$\WO{60}$\WY\Wendd
\WP\WMd{}\WUC{UHF\_CALCULATION}\5
$\WO{10}$\Wendd
\WP\WMd{}\WUC{CLOSED\_SHELL\_CALCULATION}\5
$\WO{20}$\WY\Wendd
\WY\WP \Wunnamed{code}{gints.f}%
\7
\&{integer} \&{function} \1$\>{getint}(\.{file},\ \39\|i,\ \39\|j,\ \39\|k,\ %
\39\|l,\ \39\>{mu},\ \39\>{val},\ \39\>{pointer})$\2\1\7
\&{integer}~\1\.{file}$,$ \|i$,$ \|j$,$ \|k$,$ \|l$,$ \>{mu}$,$ \>{pointer}\2\6
\&{double} \&{precision}~\1\>{val}\2\6
\&{save}\1\2\7
\&{integer}~\1\>{max\_pointer}$,$ \>{id}$,$ \>{iend}\2\6
\&{double} \&{precision}~\1\>{zero}\2\6
\&{double} \&{precision}~\1\>{labels}$(\WUC{INT\_BLOCK\_SIZE}),$ \>{value}$(%
\WUC{INT\_BLOCK\_SIZE})$\2\6
\&{data} ~\1\>{max\_pointer}${/}\WO{0}{/},$ \>{iend}{/}\WUC{NOT\_LAST%
\_BLOCK}{/}$,$ \>{zero}${/}\WO{0.0\^D00}{/}$\2\7
\WC{ File must be rewound before first use of this function           and
pointer must be set to 0 }\7
$\&{if}\,(\>{pointer}\WS\>{max\_pointer})$ \&{then}\1\6
$\&{if}\,(\>{iend}\WS\WUC{LAST\_BLOCK})$ \&{then}\1\6
$\>{val}=\>{zero};$\6
$\|i=\WO{0};$\6
$\|j=\WO{0};$\6
$\|k=\WO{0};$\6
$\|l=\WO{0}$\6
$\>{max\_pointer}=\WO{0};$\6
$\>{iend}=\WUC{NOT\_LAST\_BLOCK}$\6
$\>{getint}=\WUC{END\_OF\_FILE}$\6
\&{return}\2\6
\&{end} \&{if}\6
$\&{read}\,(\.{file})$ \>{max\_pointer}$,$ \>{iend}$,$ \>{labels}$,$ \>{value}\6
$\>{pointer}=\WO{0}$\2\6
\&{end} \&{if}\6
$\>{pointer}=\>{pointer}+\WO{1}$\6
\&{call} $\>{unpack}(\>{labels}(\>{pointer}),\ \39\|i,\ \39\|j,\ \39\|k,\ \39%
\|l,\ \39\>{mu},\ \39\>{id})$\6
$\>{val}=\>{value}(\>{pointer})$\6
$\>{getint}=\WUC{OK}$\7
\&{return}\2\6
\&{end}\WY\Wendc
\fi % End of section 1 (sect. 1, p. 1)

\WM2.


\fi % End of section 2 (sect. 1.1, p. 2a)

\WN3.  putint. This function is just happy.

\WY\WP \Wunnamed{code}{gints.f}%
\7
\&{subroutine} \1$\>{putint}(\>{nfile},\ \39\|i,\ \39\|j,\ \39\|k,\ \39\|l,\ %
\39\>{mu},\ \39\>{val},\ \39\>{pointer},\ \39\>{last})$\2\1\6
\&{implicit} \1\&{double} \&{precision}$\,(\|a-\|h,\ \39\|o-\|z)$\2\6
\&{save}\1\2\7
\&{integer}~\1\>{nfile}$,$ \|i$,$ \|j$,$ \|k$,$ \|l$,$ \>{mu}$,$ \>{pointer}$,$
\>{last}\2\6
\&{double} \&{precision}~\1\>{labels}$(\WUC{INT\_BLOCK\_SIZE}),$ \>{value}$(%
\WUC{INT\_BLOCK\_SIZE})$\2\6
\&{double} \&{precision}~\1\>{val}\2\6
\&{data} ~\1\>{max\_pointer}{/}\WUC{INT\_BLOCK\_SIZE}{/}$,$ \>{id}${/}%
\WO{0}{/}$\2\7
\Wc{     id is now unused}\7
$\&{if}\,(\>{last}\WS\.{ERR})$\1\6
\&{go} \&{to} $\WO{100}$\2\6
$\>{iend}=\WUC{NOT\_LAST\_BLOCK}$\6
$\&{if}\,(\>{pointer}\WS\>{max\_pointer})$ \&{then}\1\6
$\&{write}\,(\>{nfile})$ \>{pointer}$,$ \>{iend}$,$ \>{labels}$,$ \>{value}\6
$\>{pointer}=\WO{0}$\2\6
\&{end} \&{if}\6
$\>{pointer}=\>{pointer}+\WO{1}$\6
\&{call} $\>{pack}(\>{labels}(\>{pointer}),\ \39\|i,\ \39\|j,\ \39\|k,\ \39\|l,%
\ \39\>{mu},\ \39\>{id})$\6
$\>{value}(\>{pointer})=\>{val}$\6
$\&{if}\,(\>{last}\WS\WUC{YES})$ \&{then}\1\6
$\>{iend}=\WUC{LAST\_BLOCK}$\6
$\>{last}=\.{ERR}$\6
$\&{write}\,(\>{nfile})$ \>{pointer}$,$ \>{iend}$,$ \>{labels}$,$ \>{value}\2\6
\&{end} \&{if}\7
\Wlbl{\WO{100}\Colon\ }\&{return}\2\6
\&{end}\WY\Wendc
\fi % End of section 3 (sect. 2, p. 2b)

\WM4.
\fi % End of section 4 (sect. 2.1, p. 3)

\WN5.  genint. This subroutine generates one- and two-electron integrals.

\WY\WP \Wunnamed{code}{gints.f}%
\7
\&{subroutine} \1$\>{genint}(\>{ngmx},\ \39\>{nbfns},\ \39\>{eta},\ \39%
\>{ntype},\ \39\>{ncntr},\ \39\>{nfirst},\ \39\>{nlast},\ \39\>{vlist},\ \39%
\>{ncmx},\ \39\>{noc},\ \39\|S,\ \39\|H,\ \39\>{nfile})$\2 \&{integer}~\1%
\>{ngmx}$,$ \>{nbfns}$,$ \>{noc}$,$ \>{ncmx}\2\6
\&{double} \&{precision}~\1\>{eta}$(\WUC{MAX\_PRIMITIVES},\ \39\WO{5}),$ %
\>{vlist}$(\WUC{MAX\_CENTRES},\ \39\WO{4})$\2\6
\&{double} \&{precision}~\1\|S$(\WUC{ARB}),$ \|H$(\WUC{ARB})$\2\6
\&{integer}~\1\>{ntype}$(\WUC{ARB}),$ \>{nfirst}$(\WUC{ARB}),$ \>{nlast}$(%
\WUC{ARB}),$ \>{ncntr}$(\WUC{ARB}),$ \>{nfile}\2\7
\&{integer}~\1\|i$,$ \|j$,$ \|k$,$ \|l$,$ \>{ltop}$,$ \>{ij}$,$ \>{ji}$,$ %
\>{mu}$,$ \|m$,$ \|n$,$ \>{jtyp}$,$ \>{js}$,$ \>{jf}$,$ \>{ii}$,$ \>{jj}\2\6
\&{double} \&{precision}~\1\>{generi}$,$ \>{genoei}\2\6
\&{integer}~\1\>{pointer}$,$ \>{last}\2\6
\&{double} \&{precision}~\1\>{ovltot}$,$ \>{kintot}\2\6
\&{double} \&{precision}~\1\>{val}$,$ \>{crit}$,$ \>{alpha}$,$ \|t$,$ \>{t1}$,$
\>{t2}$,$ \>{t3}$,$ \>{sum}$,$ \>{pitern}\2\6
\&{double} \&{precision}~\1\WUC{SOO}\2\6
\&{double} \&{precision}~\1\>{gtoC}$(\WUC{MAX\_PRIMITIVES})$\2\6
\&{double} \&{precision}~\1\>{dfact}$(\WO{20})$\2\6
\&{integer}~\1\>{nr}$(\WUC{NO\_OF\_TYPES},\ \39\WO{3})$\2\6
\&{data} ~\1\>{nr}${/}\WO{0},\ \39\WO{1},\ \39\WO{0},\ \39\WO{0},\ \39\WO{2},\ %
\39\WO{0},\ \39\WO{0},\ \39\WO{1},\ \39\WO{1},\ \39\WO{0},\ \39\WO{3},\ \39%
\WO{0},\ \39\WO{0},\ \39\WO{2},\ \39\WO{2},\ \39\WO{1},\ \39\WO{0},\ \39\WO{1},%
\ \39\WO{0},\ \39\WO{1},\ \39\WO{0},\ \39\WO{0},\ \39\WO{1},\ \39\WO{0},\ \39%
\WO{0},\ \39\WO{2},\ \39\WO{0},\ \39\WO{1},\ \39\WO{0},\ \39\WO{1},\ \39\WO{0},%
\ \39\WO{3},\ \39\WO{0},\ \39\WO{1},\ \39\WO{0},\ \39\WO{2},\ \39\WO{2},\ \39%
\WO{0},\ \39\WO{1},\ \39\WO{1},\ \39\WO{0},\ \39\WO{0},\ \39\WO{0},\ \39\WO{1},%
\ \39\WO{0},\ \39\WO{0},\ \39\WO{2},\ \39\WO{0},\ \39\WO{1},\ \39\WO{1},\ \39%
\WO{0},\ \39\WO{0},\ \39\WO{3},\ \39\WO{0},\ \39\WO{1},\ \39\WO{0},\ \39\WO{1},%
\ \39\WO{2},\ \39\WO{2},\ \39\WO{1}{/}$\2\6
\&{data} ~\1\>{crit}$,$ \>{half}$,$ \>{onep5}$,$ \>{one}$,$ \>{zero}${/}\WO{1.0%
\^D-08},\ \39\WO{0.5\^D+00},\ \39\WO{1.5\^D+00},\ \39\WO{1.0\^D+00},\ \39%
\WO{0.0\^D+00}{/}$\2\6
\&{data} ~\1\>{dfact}${/}\WO{1.0},\ \39\WO{3.0},\ \39\WO{15.0},\ \39\WO{105.0},%
\ \39\WO{945.0},\ \39\WO{10395.0},\ \39\WO{135135.0},\ \39\WO{2027025.0},\ \39%
\WO{12}\ast\WO{0.0}{/}$\2\6
\&{data} ~\1\>{gtoC}${/}\WUC{MAX\_PRIMITIVES}\ast\WO{0.0\^D+00}{/}$\2\7
$\>{mu}=\WO{0}$\7
\WX{\M{6}}Copy GTO contraction coeffs to gtoC\X \X\7
\WX{\M{7}}Normalize the primitives\X \X\7
\WC{ one electron integrals }\7
$\WRS{DO}\,\|i$  $=$ $\WO{1},\ \39\>{nbfns}$ $\WRS{DO}\,\|j$  $=$ $\WO{1},\ \39%
\|i$\6
$\>{ij}=(\|j-\WO{1})\ast\>{nbfns}+\|i;$\6
$\>{ji}=(\|i-\WO{1})\ast\>{nbfns}+\|j$\6
$\|H(\>{ij})=\>{genoei}(\|i,\ \39\|j,\ \39\>{eta},\ \39\>{ngmx},\ \39%
\>{nfirst},\ \39\>{nlast},\ \39\>{ntype},\ \39\>{nr},\ \39\WUC{NO\_OF\_TYPES},\
\39\>{vlist},\ \39\>{noc},\ \39\>{ncmx},\ \39\>{ovltot},\ \39\>{kintot})$\6
$\|H(\>{ji})=\|H(\>{ij})$\6
$\|S(\>{ij})=\>{ovltot};$\6
$\|S(\>{ji})=\>{ovltot}$ \.{END} \WRS{DO} \.{END} \WRS{DO}\6
$\&{write}\,(\ast,\ \39\ast)$ $\.{"\ ONE\ ELECTRON\ INTEGRALS\ C\0OMPUTED"}$\7
\&{rewind} \>{nfile};\6
$\>{pointer}=\WO{0}$\6
$\>{last}=\WUC{NO}$\6
$\|i=\WO{1};$\6
$\|j=\WO{1};$\6
$\|k=\WO{1};$\6
$\|l=\WO{0}$\7
$\WRS{DO}\,\WO{10}$ \1\6
$\WUC{WHILE}(\>{next\_label}(\|i,\ \39\|j,\ \39\|k,\ \39\|l,\ \39\>{nbfns})\WS%
\WUC{YES})$\2\6
$\WUC{IF}(\|l\WS\>{nbfns})\>{last}=\WUC{YES}$\6
$\>{val}=\>{generi}(\|i,\ \39\|j,\ \39\|k,\ \39\|l,\ \39\WO{0},\ \39\>{eta},\ %
\39\>{ngmx},\ \39\>{nfirst},\ \39\>{nlast},\ \39\>{ntype},\ \39\>{nr},\ \39%
\WUC{NO\_OF\_TYPES})$ $\WUC{IF}(\@{dabs}(\>{val})<\>{crit})$ \&{go} \&{to} $%
\WO{10}$\6
$\WUC{CALL}\>{putint}(\>{nfile},\ \39\|i,\ \39\|j,\ \39\|k,\ \39\|l,\ \39%
\>{mu},\ \39\>{val},\ \39\>{pointer},\ \39\>{last})$\6
\Wlbl{\WO{10}\Colon\ }\WUC{CONTINUE}\7
\&{return} \&{end}\WY\Wendc
\fi % End of section 5 (sect. 3, p. 4)

\WM6.

\WY\WP\4\4\WX{\M{6}}Copy GTO contraction coeffs to gtoC\X \X${}\WSQ{}$\7
\&{do} $\|i=\WO{1},\ \39\>{ngmx}$\1\6
$\>{gtoC}(\|i)=\>{eta}(\|i,\ \39\WO{5})$\2\6
\&{end} \&{do}\WY\Wendc
\WU section~\M{5}.
\fi % End of section 6 (sect. 3.1, p. 5)

\WM7.


\WY\WP\4\4\WX{\M{7}}Normalize the primitives\X \X${}\WSQ{}$\7
\WC{ First, normalize the primitives }\7
$\>{pitern}=\WO{5.568327997\^D+00}$\5
\WC{ pi**1.5 }\6
\&{do} $\|j=\WO{1},\ \39\>{nbfns}$\1\6
$\>{jtyp}=\>{ntype}(\|j);$\6
$\>{js}=\>{nfirst}(\|j);$\6
$\>{jf}=\>{nlast}(\|j)$\6
$\|l=\>{nr}(\>{jtyp},\ \39\WO{1});$\6
$\|m=\>{nr}(\>{jtyp},\ \39\WO{2});$\6
$\|n=\>{nr}(\>{jtyp},\ \39\WO{3})$\6
\&{do} $\|i=\>{js},\ \39\>{jf}$\1\6
$\>{alpha}=\>{eta}(\|i,\ \39\WO{4});$\6
$\WUC{SOO}=\>{pitern}\ast(\>{half}\WSl\>{alpha})\WEE{\WO{1.5}}+\WUC{D00}$\6
$\>{t1}=\>{dfact}(\|l+\WO{1})\WSl\>{alpha}\WEE{\|l}$\6
$\>{t2}=\>{dfact}(\|m+\WO{1})\WSl\>{alpha}\WEE{\|m}$\6
$\>{t3}=\>{dfact}(\|n+\WO{1})\WSl\>{alpha}\WEE{\|n}$\6
$\&{write}\,(\ast,\ \39\ast)$ $\>{dfact}(\|l+\WO{1}),$ $\>{dfact}(\|m+\WO{1}),$
$\>{dfact}(\|n+\WO{1}),$ \|l$,$ \|m$,$ \|n\6
$\>{eta}(\|i,\ \39\WO{5})=\>{one}\WSl\@{dsqrt}(\WUC{SOO}\ast\>{t1}\ast\>{t2}%
\ast\>{t3})$\2\6
\&{end} \&{do}\2\6
\&{end} \&{do}\7
\WC{ Now normalize the basis functions }\7
\&{do} $\|j=\WO{1},\ \39\>{nbfns}$\1\6
$\>{jtyp}=\>{ntype}(\|j);$\6
$\>{js}=\>{nfirst}(\|j);$\6
$\>{jf}=\>{nlast}(\|j)$\6
$\|l=\>{nr}(\>{jtyp},\ \39\WO{1});$\6
$\|m=\>{nr}(\>{jtyp},\ \39\WO{2});$\6
$\|n=\>{nr}(\>{jtyp},\ \39\WO{3})$\7
$\>{sum}=\>{zero}$\6
\&{do} $\>{ii}=\>{js},\ \39\>{jf}$\1\6
\&{do} $\>{jj}=\>{js},\ \39\>{jf}$\1\6
$\|t=\>{one}\WSl(\>{eta}(\>{ii},\ \39\WO{4})+\>{eta}(\>{jj},\ \39\WO{4}))$\6
$\WUC{SOO}=\>{pitern}\ast(\|t\WEE{\>{onep5}})\ast\>{eta}(\>{ii},\ \39\WO{5})%
\ast\>{eta}(\>{jj},\ \39\WO{5})$\6
$\|t=\>{half}\ast\|t$\6
$\>{t1}=\>{dfact}(\|l+\WO{1})\WSl\|t\WEE{\|l}$\6
$\>{t2}=\>{dfact}(\|m+\WO{1})\WSl\|t\WEE{\|m}$\6
$\>{t3}=\>{dfact}(\|n+\WO{1})\WSl\|t\WEE{\|n}$\6
$\>{sum}=\>{sum}+\>{gtoC}(\>{ii})\ast\>{gtoC}(\>{jj})\ast\WUC{SOO}\ast\>{t1}%
\ast\>{t2}\ast\>{t3}$\2\6
\&{end} \&{do}\2\6
\&{end} \&{do}\6
$\>{sum}=\>{one}\WSl\@{dsqrt}(\>{sum})$\6
\&{do} $\>{ii}=\>{js},\ \39\>{jf}$\1\6
$\>{gtoC}(\>{ii})=\>{gtoC}(\>{ii})\ast\>{sum}$\2\6
\&{end} \&{do}\2\6
\&{end} \&{do}\7
\&{do} $\>{ii}=\WO{1},\ \39\>{ngmx}$\1\6
$\>{eta}(\>{ii},\ \39\WO{5})=\>{eta}(\>{ii},\ \39\WO{5})\ast\>{gtoC}(\>{ii})$\2%
\6
\&{end} \&{do}\WY\Wendc
\WU section~\M{5}.
\fi % End of section 7 (sect. 3.2, p. 6)

\WM8.


\fi % End of section 8 (sect. 3.3, p. 7a)

\WN9.  pack. Store the six electron repulsion labels.

\WY\WP \Wunnamed{code}{gints.f}%
\7
\&{subroutine} \1$\>{pack}(\|a,\ \39\|i,\ \39\|j,\ \39\|k,\ \39\|l,\ \39\|m,\ %
\39\|n)$\2\1\6
\&{double} \&{precision}~\1\|a\2\6
\&{integer}~\1\|i$,$ \|j$,$ \|k$,$ \|l$,$ \|m$,$ \|n\2\7
\&{double} \&{precision}~\1\>{word}\2\6
\&{integer}~\1\>{id}$(\WO{6})$\2\6
$\&{character}{\ast\WO{1}}~$\1\>{chr1}$(\WO{8}),$ \>{chr2}$(\WO{24})$\2\6
$\&{equivalence}\,(\>{word},\ \39\>{chr1}(\WO{1})),\,(\>{id}(\WO{1}),\ \39%
\>{chr2}(\WO{1}))$\1\2\7
$\>{id}(\WO{1})=\|i;$\6
$\>{id}(\WO{2})=\|j;$\6
$\>{id}(\WO{3})=\|k$\6
$\>{id}(\WO{4})=\|l;$\6
$\>{id}(\WO{5})=\|m;$\6
$\>{id}(\WO{6})=\|n$\7
\&{do} $\>{ii}=\WO{1},\ \39\WO{6}$\1\6
$\>{chr1}(\>{ii})=\>{chr2}((\>{ii}-\WO{1})\ast\WUC{BYTES\_PER\_INTEGER}+%
\WUC{LEAST\_BYTE})$\2\6
\&{end} \&{do}\6
$\|a=\>{word}$\6
\&{return}\2\6
\&{end}\Wendc
\fi % End of section 9 (sect. 4, p. 7b)

\WM10.

\fi % End of section 10 (sect. 4.1, p. 7c)

\WN11.  unpack. Regenerate the 6 electron repulsion labels.

\WY\WP \Wunnamed{code}{gints.f}%
\7
\&{subroutine} \1$\>{unpack}(\|a,\ \39\|i,\ \39\|j,\ \39\|k,\ \39\|l,\ \39\|m,\
\39\|n)$\2\1\6
\&{double} \&{precision}~\1\|a\2\6
\&{integer}~\1\|i$,$ \|j$,$ \|k$,$ \|l$,$ \|m$,$ \|n\2\7
\&{double} \&{precision}~\1\>{word}\2\6
\&{integer}~\1\>{id}$(\WO{6})$\2\6
$\&{character}{\ast\WO{1}}~$\1\>{chr1}$(\WO{8}),$ \>{chr2}$(\WO{24})$\2\6
$\&{equivalence}\,(\>{word},\ \39\>{chr1}(\WO{1})),\,(\>{id}(\WO{1}),\ \39%
\>{chr2}(\WO{1}))$\1\2\7
\&{do} $\>{ii}=\WO{1},\ \39\WO{6}$\1\6
$\>{chr2}((\>{ii}-\WO{1})\ast\WUC{BYTES\_PER\_INTEGER}+\WUC{LEAST\_BYTE})=%
\>{chr1}(\>{ii})$\2\6
\&{end} \&{do}\7
$\>{id}(\WO{1})=\|i;$\6
$\>{id}(\WO{2})=\|j;$\6
$\>{id}(\WO{3})=\|k$\6
$\>{id}(\WO{4})=\|l;$\6
$\>{id}(\WO{5})=\|m;$\6
$\>{id}(\WO{6})=\|n$\7
\&{return}\2\6
\&{end}\WY\Wendc
\fi % End of section 11 (sect. 5, p. 8a)

\WM12.

\fi % End of section 12 (sect. 5.1, p. 8b)

\WN13.  next\_label. Generate the next label of electron repulsion integral.

A function to generate  the
four standard loops which are used to generate (or, more rarely) process
the electron repulsion integrals.

The sets of integer values are generated in the usual
standard order in canonical form, that is, equivalent to the set of loops: \\
 \WCD{ \&{do} $\|i=\WO{1},\ \|n$ $\{$ \&{do} $\|j=\WO{1},\ \|i$ $\{$ \&{do} $%
\|k=\WO{1},\ \|i$ $\{$ $\>{ltop}=\|k$ $\&{if}\,(\|i\WS\|k)$ $\>{ltop}=\|j$ %
\&{do} $\|l=\WO{1},\ \>{ltop}$ $\{$ \&{do} \>{something}\>{with}\|i\|j\|k\|l $%
\}$ $\}$ $\}$ $\}$}
\ \\
Note that, just as is the case with the \WCD{ \&{do}}-loops,
the whole process must be {\em initialised\/} by
setting initial values of \WCD{ \|i}, \WCD{ \|j}, \WCD{ \|k} and \WCD{ \|l}.
If the whole set of labels is required then \\
\WCD{ $\|i=\WO{1}$}, \WCD{ $\|j=\WO{1}$}, \WCD{ $\|k=\WO{1}$}, \WCD{ \|l}=0 \\
is appropriate.

Usage is, typically, \\
\WCD{$\|i=\WO{0}$ $\|j=\WO{0}$ $\|k=\WO{0}$ $\|l=\WO{0}$} \\
\WCD{ $\>{while}(\>{next\_label}(\|i,\ \|j,\ \|k,\ \|l,\ \|n)\WS\WUC{YES})$} \\
\WCD{ $\{$} \\
  do something with i j k and l \\
\WCD{ $\}$} \\

\WY\WP \Wunnamed{code}{gints.f}%
\7
\&{integer} \&{function} \1$\>{next\_label}(\|i,\ \39\|j,\ \39\|k,\ \39\|l,\ %
\39\|n)$\2\1\6
\&{integer}~\1\|i$,$ \|j$,$ \|k$,$ \|l$,$ \|n\2\7
\&{integer}~\1\>{ltop}\2\7
$\>{next\_label}=\WUC{YES}$\6
$\>{ltop}=\|k$\6
$\&{if}\,(\|i\WS\|k)$\1\6
$\>{ltop}=\|j$\2\7
$\&{if}\,(\|l<\>{ltop})$ \&{then}\1\6
$\|l=\|l+\WO{1}$\2\6
\&{else}\1\6
$\|l=\WO{1}$\6
$\&{if}\,(\|k<\|i)$ \&{then}\1\6
$\|k=\|k+\WO{1}$\2\6
\&{else}\1\6
$\|k=\WO{1}$\6
$\&{if}\,(\|j<\|i)$ \&{then}\1\6
$\|j=\|j+\WO{1}$\2\6
\&{else}\1\6
$\|j=\WO{1}$\6
$\&{if}\,(\|i<\|n)$ \&{then}\1\6
$\|i=\|i+\WO{1}$\2\6
\&{else}\1\6
$\>{next\_label}=\WUC{NO}$\2\6
\&{end} \&{if}\2\6
\&{end} \&{if}\2\6
\&{end} \&{if}\2\6
\&{end} \&{if}\6
\&{return}\2\6
\&{end}\WY\Wendc
\fi % End of section 13 (sect. 6, p. 9)

\WM14.

\fi % End of section 14 (sect. 6.1, p. 10)

\WN15.  INDEX.
\fi % End of section 15 (sect. 7, p. 11)

\input INDEX.tex
\input MODULES.tex

\Winfo{"fweave gints.web"}  {"gints.web"} {(none)}
 {\Fortran}


\Wcon{15}
\FWEBend
