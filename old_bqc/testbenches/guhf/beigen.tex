% FWEAVE v1.62 (September 23, 1998)

% --- FWEB's macro package ---
\input fwebmac.sty

% --- Initialization parameters from FWEB's style file `fweb.sty' ---
\Wbegin[;]
  % #1 --- [LaTeX.class.options;LaTeX.package.options]
{article;}
  % #2 --- {LaTeX.class;LaTeX.package}
{1em}
  % #3 --- {indent.TeX}
{1em}
  % #4 --- {indent.code}
{CONTENTS.tex}
  % #5 --- {contents.TeX}
{ % #6 ---
 {\&\WRS}
  % #1 --- {{format.reserved}{format.RESERVED}}
 {\|}
  % #2 --- {format.short_id}
 {\>\WUC}
  % #3 --- {{format.id}{format.ID}}
 {\>\WUC}
  % #4 --- {{format.outer_macro}{format.OUTER_MACRO}}
 {\>\WUC}
  % #5 --- {{format.WEB_macro}{format.WEB_MACRO}}
 {\@}
  % #6 --- {format.intrinsic}
 {\.\.}
  % #7 --- {{format.keyword}{format.KEYWORD}}
 {\.}
  % #8 --- {format.typewriter}
 {}
  % #9 --- (For future use)
}
{\M}
  % #7 --- {encap.prefix}
{;}
  % #8 --- {doc.preamble;doc.postamble}
{INDEX}
  % #9 --- {index.name}


% --- Beginning of user's limbo section ---




% --- Limbo text from style-file parameter `limbo.end' ---
\FWEBtoc

\WN1.  beigen. Old faithful Jacobi diagonisation routine. This version
is modified to diagonalise a \lq\lq blocked'' matrix. The matrix is
assumed to be \WCD{ \>{nblock}} sub-matrices along the principal diagonal
of dimension \WCD{ $\>{iblock}(\WO{1}),\ \>{iblock}(\WO{2})$ $,$} \ldots \WCD{
$\>{iblock}(\>{nblock})$} with zeroes
off these blocks. Of course, if \WCD{ $\>{nblock}=\WO{1}$} then \WCD{ $%
\>{iblock}(\WO{1})=\|n$}.

\begin{minipage}{4.5in}
\ \\
\begin{description}
\item[NAME]  \      \\
beigen
\item[SYNOPSIS] \ \\
\begin{verbatim}
   subroutine beigen(H,U,n,init,nblock,iblock);
   implicit double precision (a-h,o-z);
   double precision H(*), U(*);
   integer n, init, nblock, iblock(*);

\end{verbatim}
\item[DESCRIPTION] \ \\
Diagonalises a symmetric matrix which consists of \WCD{ \>{nblock}}
symmetric matrices along the diagonal of \WCD{ \|H}. The eigenvalues and
eigenvectors are returned in order of ascending eigenvalue.
\item[ARGUMENTS] \  \\
\begin{description}
\item[H] Input/Outpu:t Matrix to be diagonalised. On output, the
eigenvalues are on the main diagonal of \WCD{ \|H}.
\item[U] Input/Output: If \WCD{ \>{init}} is non-zero \WCD{ \|U} contains a
guess at
the eigenvector matrix, if not it is initialised to the unit matrix.
Ouput is the eigenvectors.
\item[n] Input: The size of the matrices overall,
\item[init] Input: zero if \WCD{ \|U} is a sensible starting-point for the
process, non-zero if not.
\item[nblock] Input: The number of sub-matrices in \WCD{ \|H}.
\item[iblock] Input: \WCD{ $\>{iblock}(\WO{1})$} \ldots \WCD{ $\>{iblock}(%
\>{nblock})$} are the
dimensions of the sub-matrices.
\end{description}

\item[DIAGNOSTICS] \  \\
None, possibility of infinite loop but unlikely.
\ \\ \ \\
\end{description}
\end{minipage}
\clearpage


\WY\WP \Wunnamed{code}{beigen.f}%
 \&{subroutine} \1$\>{beigen}(\|H,\ \39\|U,\ %
\39\|n,\ \39\>{init},\ \39\>{nblock},\ \39\>{iblock})$\2\7
\WX{\M{2}}Interface declarations\X \X\7
\&{data} ~\1\>{zero}$,$ \>{eps}$,$ \>{one}$,$ \>{two}$,$ \>{four}${/}\WO{0.0%
\^D+00},\ \39\WO{1.0\^D-20},\ \39\WO{1.0\^D+00},\ \39\WO{2.0\^D+00},\ \39%
\WO{4.0\^D+00}{/}$\2\7
\WX{\M{3}}Initialize U\X \X\7
$\>{nmax}=\WO{0}$\6
\&{do} $\>{nsym}=\WO{1},\ \39\>{nblock}$\1\5
\WC{ Loop over the \WCD{ \>{nsym}} sub-matrices }\7
\WC{ \WCD{ \>{nmin}} and \WCD{ \>{nmax}} are the limits of the current
sub-matrix }\7
$\>{nmin}=\>{nmax}+\WO{1}$\6
$\>{nmax}=\>{nmax}+\>{iblock}(\>{nsym})$\7
\WC{ Start sweep through off-diagonal elements; the sweep is repeated
      until the largest off-diagonal element of \WCD{ \|H} is less than \WCD{ %
\>{eps}} }\7
\&{do} $\>{while}(\>{hmax}>\>{eps})$\1\7
\WX{\M{4}}Reduce current off-diagonal to zero\X \X\2\7
\&{end} \&{do}\2\7
\&{end} \&{do}\5
\WC{ End of \WCD{ \>{nsym}} loop on blocks }\7
\WX{\M{8}}Sort the eigenvalues and vectors\X \X\7
\&{return}\6
\.{END}\WY\Wendc
\fi % End of section 1 (sect. 1, p. 1)

\WM2.

\WY\WP\4\4\WX{\M{2}}Interface declarations\X \X${}\WSQ{}$\6
\&{implicit} \1\&{double} \&{precision}$\,(\|a-\|h,\ \39\|o-\|z)$\2\6
\&{double} \&{precision}~\1\|H$(\ast),$ \|U$(\ast)$\2\6
\&{integer}~\1\|n$,$ \>{init}$,$ \>{nblock}$,$ \>{iblock}$(\ast)$\2\WY\Wendc
\WU section~\M{1}.
\fi % End of section 2 (sect. 1.1, p. 2a)

\WM3.

This simply sets \WCD{ \|U} to the unit matrix. It is used if \WCD{ \>{init}}
is zero.
If \WCD{ \>{init}} is not zero, the incoming \WCD{ \|U} is assumed to be a
sensible
starting-point for the calculation.

\WY\WP\4\4\WX{\M{3}}Initialize U\X \X${}\WSQ{}$\6
$\&{if}\,(\>{init}\WS\WO{0})$ \&{then}\1\7
\&{do} $\|i=\WO{1},\ \39\|n$\1\6
$\>{ii}=\|n\ast(\|i-\WO{1})+\|i$\7
\&{do} $\|j=\WO{1},\ \39\|n$\1\6
$\>{ij}=\|n\ast(\|j-\WO{1})+\|i$\6
$\|U(\>{ij})=\>{zero}$\2\6
\&{end} \&{do}\7
$\|U(\>{ii})=\>{one}$\2\6
\&{end} \&{do}\2\7
\&{end} \&{if}\WY\Wendc
\WU section~\M{1}.
\fi % End of section 3 (sect. 1.2, p. 3a)

\WM4. This is the central algorithm which calclautes and uses the
``angle'' which reduces the current off-diagonal element of \WCD{ \|H} to
zero. The effect on the other off-diagonals is computed and the
largest off-diagonal element saved for convergence testing

\WY\WP\4\4\WX{\M{4}}Reduce current off-diagonal to zero\X \X${}\WSQ{}$\6
$\>{hmax}=\>{zero}$\5
\WC{ \WCD{ \>{hmax}} keeps track of the largest off-diagonal
       element of \WCD{ \|H} }\7
\&{do} $\|i=\>{nmin}+\WO{1},\ \39\>{nmax}$\1\7
$\>{jtop}=\|i-\WO{1}$\7
\&{do} $\|j=\>{nmin},\ \39\>{jtop}$\1\7
\WX{\M{5}}Calculate Rotation Angle\X \X\6
\WX{\M{6}}Apply Rotation to U\X \X\6
\WX{\M{7}}Apply Rotation to H\X \X\2\7
\&{end} \&{do}\2\7
\&{end} \&{do}\WY\Wendc
\WU section~\M{1}.
\fi % End of section 4 (sect. 1.3, p. 3b)

\WM5.
\WCD{ \@{tan}} is $tan(2 \theta)$ where $\theta$ is the rotation angle
   which makes \WCD{ $\|H(\>{ij})$} vanish. \WCD{ \|c} and \WCD{ \|s} are $cos %
\theta$ and
$sin \theta$ obtained by the usual ``half-angle formula'' from
$tan(2 \theta)$

\WY\WP\4\4\WX{\M{5}}Calculate Rotation Angle\X \X${}\WSQ{}$\6
\WC{ positions of matrix elements }\6
$\>{ii}=\|n\ast(\|i-\WO{1})+\|i;$\6
$\>{jj}=\|n\ast(\|j-\WO{1})+\|j$\6
$\>{ij}=\|n\ast(\|j-\WO{1})+\|i;$\6
$\>{ji}=\|n\ast(\|i-\WO{1})+\|j$\6
$\>{hii}=\|H(\>{ii});$\6
$\>{hjj}=\|H(\>{jj});$\6
$\>{hij}=\|H(\>{ij});$\6
$\>{hsq}=\>{hij}\ast\>{hij}$\6
$\&{if}\,(\>{hsq}>\>{hmax})$\1\6
$\>{hmax}=\>{hsq}$\2\6
$\&{if}\,(\>{hsq}<\>{eps})$\1\6
\>{cycle}\2\5
\WC{ omit zero H(ij) }\6
$\>{del}=\>{hii}-\>{hjj}$\6
$\@{sign}=\>{one}$\6
$\&{if}\,(\>{del}<\>{zero})$ \&{then}\1\6
$\@{sign}={-}\>{one}$\6
$\>{del}={-}\>{del}$\2\6
\&{end} \&{if}\6
$\>{denom}=\>{del}+\@{dsqrt}(\>{del}\ast\>{del}+\>{four}\ast\>{hsq})$\7
$\@{tan}=\>{two}\ast\@{sign}\ast\>{hij}\WSl\>{denom}$\6
$\|c=\>{one}\WSl\@{dsqrt}(\>{one}+\@{tan}\ast\@{tan})$\6
$\|s=\|c\ast\@{tan}$\Wendc
\WU section~\M{4}.
\fi % End of section 5 (sect. 1.4, p. 4)

\WM6.

\WY\WP\4\4\WX{\M{6}}Apply Rotation to U\X \X${}\WSQ{}$\6
\&{do} $\|k=\WO{1},\ \39\|n$\1\7
$\>{kj}=\|n\ast(\|j-\WO{1})+\|k$\6
$\>{ki}=\|n\ast(\|i-\WO{1})+\|k$\6
$\>{jk}=\|n\ast(\|k-\WO{1})+\|j$\6
$\>{ik}=\|n\ast(\|k-\WO{1})+\|i$\7
$\>{temp}=\|c\ast\|U(\>{kj})-\|s\ast\|U(\>{ki})$\6
$\|U(\>{ki})=\|s\ast\|U(\>{kj})+\|c\ast\|U(\>{ki})$\6
$\|U(\>{kj})=\>{temp}$\7
\WC{ If \WCD{ \|k} is niether \WCD{ \|i} or \WCD{ \|j} then apply the current
rotation }\7
$\&{if}\,((\|i\WS\|k)\WOR(\|j\WS\|k))$\1\6
\>{cycle}\2\6
$\>{temp}=\|c\ast\|H(\>{kj})-\|s\ast\|H(\>{ki})$\6
$\|H(\>{ki})=\|s\ast\|H(\>{kj})+\|c\ast\|H(\>{ki})$\6
$\|H(\>{kj})=\>{temp}$\6
$\|H(\>{ik})=\|H(\>{ki})$\6
$\|H(\>{jk})=\|H(\>{kj})$\2\6
\&{end} \&{do}\7
\WC{ This does not make any off-diagonal element zero; in fact it will,     in
general, re-generate ones which have been zeroized in other cycles }\WY\Wendc
\WU section~\M{4}.
\fi % End of section 6 (sect. 1.5, p. 5a)

\WM7.

\WY\WP\4\4\WX{\M{7}}Apply Rotation to H\X \X${}\WSQ{}$\7
$\|H(\>{ii})=\|c\ast\|c\ast\>{hii}+\|s\ast\|s\ast\>{hjj}+\>{two}\ast\|c\ast\|s%
\ast\>{hij}$\6
$\|H(\>{jj})=\|c\ast\|c\ast\>{hjj}+\|s\ast\|s\ast\>{hii}-\>{two}\ast\|c\ast\|s%
\ast\>{hij}$\6
$\|H(\>{ij})=\>{zero}$\6
$\|H(\>{ji})=\>{zero}$\7
\WC{ This is the key step; it generates one zero off-diagonal element }\Wendc
\WU section~\M{4}.
\fi % End of section 7 (sect. 1.6, p. 5b)

\WM8.

Now Sort the eigenvalues and eigenvectors into ascending order.
OVERALL; i.e. not within each block.

If it is required to sort the eigenvalues and vectors into
ascending order {\em within\/} each block then this coding must be
changed. For example, one may wish to occupy the lowest orbitals
of each of several symmetry types to generate specific states of the
molecule.

\WY\WP\4\4\WX{\M{8}}Sort the eigenvalues and vectors\X \X${}\WSQ{}$\7
$\>{nmax}=\WO{0}$\6
\&{do} $\>{nsym}=\WO{1},\ \39\>{nblock}$\1\7
$\>{nmin}=\>{nmax}+\WO{1}$\6
$\>{nmax}=\>{nmax}+\>{iblock}(\>{nsym})$\6
\&{call} $\>{epsort}(\|H,\ \39\|U,\ \39\|n,\ \39\>{nmin},\ \39\>{nmax})$\2\7
\&{end} \&{do}\WY\Wendc
\WU section~\M{1}.
\fi % End of section 8 (sect. 1.7, p. 6)

\WN9.  epsort.   Sort eigenvectors from n1 to n2 into eigenvalue order

\WY\WP \Wunnamed{code}{beigen.f}%
\&{subroutine} \1$\>{epsort}(\|H,\ \39\|U,\ %
\39\|n,\ \39\>{n1},\ \39\>{n2})$\2\1\6
\&{implicit} \1\&{double} \&{precision}$\,(\|a-\|h,\ \39\|o-\|z)$\2\6
\&{double} \&{precision}~\1\|H$(\ast),$ \|U$(\ast)$\2\6
\&{integer}~\1\|n$,$ \>{n1}$,$ \>{n2}\2\7
\&{double} \&{precision}~\1\>{temp}\2\6
\&{integer}~\1\>{iq}$,$ \>{jq}$,$ \>{ii}$,$ \>{jj}$,$ \|k$,$ \>{ilr}$,$ \>{imr}%
\2\7
$\>{iq}=(\>{n1}-\WO{2})\ast\|n$\6
\&{do} $\|i=\>{n1},\ \39\>{n2}$\1\6
$\>{iq}=\>{iq}+\|n$\6
$\>{ii}=(\|i-\WO{1})\ast\|n+\|i$\6
$\>{jq}=\|n\ast(\|i-\WO{2})$\6
\&{do} $\|j=\|i,\ \39\>{n2}$\1\6
$\>{jq}=\>{jq}+\|n$\6
$\>{jj}=(\|j-\WO{1})\ast\|n+\|j$\6
$\&{if}\,(\|H(\>{ii})<\|H(\>{jj}))$\1\6
\>{cycle}\2\5
\WC{ this means H(1) is lowest! }\6
$\>{temp}=\|H(\>{ii})$\6
$\|H(\>{ii})=\|H(\>{jj})$\6
$\|H(\>{jj})=\>{temp}$\6
\&{do} $\|k=\WO{1},\ \39\|n$\1\6
$\>{ilr}=\>{iq}+\|k$\6
$\>{imr}=\>{jq}+\|k$\6
$\>{temp}=\|U(\>{ilr})$\6
$\|U(\>{ilr})=\|U(\>{imr})$\6
$\|U(\>{imr})=\>{temp};$\2\6
\&{end} \&{do}\2\6
\&{end} \&{do}\2\6
\&{end} \&{do}\6
\&{return}\2\6
\&{end}\WY\Wendc
\fi % End of section 9 (sect. 2, p. 7)

\WN10.  INDEX.
\fi % End of section 10 (sect. 3, p. 8)

\input INDEX.tex
\input MODULES.tex

\Winfo{"fweave beigen.web"}  {"beigen.web"} {(none)}
 {\Fortran}


\Wcon{10}
\FWEBend
