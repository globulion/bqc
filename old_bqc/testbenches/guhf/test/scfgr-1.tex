% FWEAVE v1.62 (September 23, 1998)

% --- FWEB's macro package ---
\input fwebmac.sty

% --- Initialization parameters from FWEB's style file `fweb.sty' ---
\Wbegin[;]
  % #1 --- [LaTeX.class.options;LaTeX.package.options]
{article;}
  % #2 --- {LaTeX.class;LaTeX.package}
{1em}
  % #3 --- {indent.TeX}
{1em}
  % #4 --- {indent.code}
{CONTENTS.tex}
  % #5 --- {contents.TeX}
{ % #6 ---
 {\&\WRS}
  % #1 --- {{format.reserved}{format.RESERVED}}
 {\|}
  % #2 --- {format.short_id}
 {\>\WUC}
  % #3 --- {{format.id}{format.ID}}
 {\>\WUC}
  % #4 --- {{format.outer_macro}{format.OUTER_MACRO}}
 {\>\WUC}
  % #5 --- {{format.WEB_macro}{format.WEB_MACRO}}
 {\@}
  % #6 --- {format.intrinsic}
 {\.\.}
  % #7 --- {{format.keyword}{format.KEYWORD}}
 {\.}
  % #8 --- {format.typewriter}
 {}
  % #9 --- (For future use)
}
{\M}
  % #7 --- {encap.prefix}
{;}
  % #8 --- {doc.preamble;doc.postamble}
{INDEX}
  % #9 --- {index.name}


% --- Beginning of user's limbo section ---





% --- Limbo text from style-file parameter `limbo.end' ---
\FWEBtoc

\WN1.  scfGR.
One routine to form G(R) for closed shells and UHF

\WY\WP \Wunnamed{defs}{scfgr-1.f}%
\WMd{}\WUC{OK}\5
$\WO{0}$\Wendd
\WP\WMd{}\WUC{YES}\5
$\WO{38}$\Wendd
\WP\WMd{}\WUC{END\_OF\_FILE}\5
$\WO{1}$\Wendd
\WP\WMd{}\WUC{CLOSED\_SHELL\_CALCULATION}\5
$\WO{145}$\Wendd
\WP\WMd{}\WUC{UHF\_CALCULATION}\5
$\WO{2345}$\WY\Wendd
\WY\WP \Wunnamed{code}{scfgr-1.f}%
 \&{subroutine}~\1\>{scfGR}$(\|R,\ \39\|G,\ %
\39\|n,\ \39\>{nfile},\ \39\>{ntype});$\2\7
\WC{     This routine adds (repeat adds) G(R) into the array \WCD{ \|G} }\7
\&{implicit} \1\&{double} \&{precision}$\,(\|a-\|h,\ \39\|o-\|z);$\2\6
\&{double} \&{precision}~\1\|R$(\ast),$ \|G$(\ast);$\2\6
\&{integer}~\1\|n$,$ \>{nfile}$,$ \>{ntype};\2\7
$\{$\7
\&{integer}~\1\>{pointer}$,$ \>{spin}$,$ \>{skip};\2\6
\&{integer}~\1\>{getint};\2\5
\WC{ integral reading function }\6
\&{data} ~\1\>{zero}$,$ \>{one}$,$ \>{two}${/}\WO{0.0\^D0},\ \39\WO{1.0\^D0},\ %
\39\WO{2.0\^D0}{/};$\2\7
\&{rewind} \>{nfile};\6
$\>{pointer}=\WO{0};$\5
\WC{ Initialize the repulsion integral file }\7
\WX{\M{2}}Set up \WCD{ \|J} and \WCD{ \|K} multipliers\X \X\7
\WC{ Process the file  }\7
$\&{while}\,(\>{getint}(\>{nfile},\ \39\>{is},\ \39\>{js},\ \39\>{ks},\ \39%
\>{ls},\ \39\>{mu},\ \39\>{va},\ \39\>{pointer})\WI\WUC{END\_OF\_FILE})$ \7
$\{$\7
$\>{ijs}=\>{is}\ast(\>{is}-\WO{1})$ {/} $\WO{2}+\>{js};$\6
$\>{kls}=\>{ks}\ast(\>{ks}-\WO{1})\WSl\WO{2}+\>{ls};$\7
\WX{\M{3}}Run over four spin Combinations\X \X\7
$\}$\7
\WX{\M{4}}Symmetrise the G matrix\X \X\7
\&{return};\7
$\}$\WY\Wendc
\fi % End of section 1

\WM2. Depending on the two possible cases (closed-shell or
UHF) there are two different formulae for the
electron-interaction matrix G(R) and there are two corresponding
basis sizes; spatial orbitals for closed-shell and spin-orbitals
for UHF. This code fragment uses \WCD{ \>{ntype}} to make the
appropriate chioces.

\WY\WP\4\4\WX{\M{2}}Set up \WCD{ \|J} and \WCD{ \|K} multipliers\X \X${}\WSQ{}$%
\6
$\&{if}\,(\>{ntype}\WS\WUC{CLOSED\_SHELL\_CALCULATION})$\1\7
$\{$\7
$\>{n2}=\|n;$\5
\WC{ basis size is spatial basis size }\6
$\>{em}=\>{two};$\6
$\>{en}=\>{one};$\5
\WC{ G(R) = 2J(R) - K(R) }\7
$\}$\2\7
$\&{if}\,(\>{ntype}\WS\WUC{UHF\_CALCULATION})$\1\7
$\{$\6
$\>{n2}=\WO{2}\ast\|n;$\5
\WC{ basis is spin-orbitals, twice spatial basis }\6
$\>{em}=\>{one};$\6
$\>{en}=\>{one};$\5
\WC{ G(R) = J(R) - K(R) }\7
$\}$\2\Wendc
\WU section~\M{1}.
\fi % End of section 2

\WM3. The electron-repulsion integrals are over the spatial
basis functions. If the calculation is of UHF type then
each spatial repulsion integral must due duty for as many
as four repulsion integrals over the spin-orbitals. This
loop over the four cases is omitted if the
calculation is closed-shell. Notice that the whole
algorithm depends on the spin-orbitals being
in the order: $n$ $\alpha$-spin-orbitals them $n$
$\beta$-spin-orbitals.

The repulsion integrals are assumed to be labelled in
standard order {\em but} this order may be disturbed
by adding $n$ to pairs of the four labels. In practice,
the order is only disturbed in case 4: but calling
\WCD{ \&{subroutine}~\>{order}} in all cases would do no harm.

\WY\WP\4\4\WX{\M{3}}Run over four spin Combinations\X \X${}\WSQ{}$\6
\&{do} $\>{spin}=\WO{1},\ \39\WO{4};$\1\7
$\{$\7
\WC{ jump out of the loop if closed shell }\7
$\&{if}\,((\>{spin}>\WO{1})\WW(\>{ntype}\WS\WUC{CLOSED\_SHELL\_CALCULATION}))$%
\1\6
\&{break};\2\6
$\>{skip}=\WUC{NO};$\5
\WC{ skip is a code used to avoid counting some                contributions
twice: see case 4: }\7
$\&{switch}\,(\>{spin})$ \1\7
$\{$\7
\4\&{case} $\WO{1}\Colon\ $\5
\WC{ alpha - alpha }\6
$\{$\6
$\|i=\>{is};$\6
$\|j=\>{js};$\6
$\|k=\>{ks};$\6
$\|l=\>{ls};$\6
\&{break};\6
$\}$\7
\4\&{case} $\WO{2}\Colon\ $\5
\WC{ beta - beta }\6
$\{$\6
$\|i=\>{is}+\|n;$\6
$\|j=\>{js}+\|n;$\6
$\|k=\>{ks}+\|n;$\6
$\|l=\>{ls}+\|n;$\6
\&{break};\6
$\}$\7
\4\&{case} $\WO{3}\Colon\ $\5
\WC{ beta - alpha }\6
$\{$\6
$\|i=\>{is}+\|n;$\6
$\|j=\>{js}+\|n;$\6
$\|k=\>{ks};$\6
$\|l=\>{ls};$\6
\&{break};\6
$\}$\7
\4\&{case} $\WO{4}\Colon\ $\5
\WC{ alpha - beta; adding \WCD{ \|n} to \WCD{ \|k} and \WCD{ \|l}   upsets the
standard order of \WCD{ $\|i,\ \|j,\ \|k,\ \|l$} so they must be re-ordered }\6
$\{$\6
$\&{if}\,(\>{ijs}\WS\>{kls})$\1\6
$\>{skip}=\WUC{YES};$\2\6
$\|i=\>{is};$\6
$\|j=\>{js};$\6
$\|k=\>{ks}+\|n;$\6
$\|l=\>{ls}+\|n;$\6
\&{call} $\>{order}(\|i,\ \39\|j,\ \39\|k,\ \39\|l);$\6
\&{break};\6
$\}$\7
$\}$\2\5
\WC{end of "switch" }\7
$\&{if}\,(\>{skip}\WS\WUC{YES})$\1\6
\&{next};\2\5
\WC{ This is a duplicate, skip it }\7
$\>{emm}=\>{em};$\6
$\>{enn}=\>{en};$\7
\WC{ drop alpha-beta exchange in cases 3 and 4; it is zero by spin integration
}\7
$\&{if}\,(\>{spin}\WG\WO{3})$\1\6
$\>{enn}=\>{zero};$\2\7
\WC{ Now use the current (spin-)orbital integral to increment    G(R) }\7
\&{call} $\>{GofR}(\|R,\ \39\|G,\ \39\>{n2},\ \39\>{emm},\ \39\>{enn},\ \39\|i,%
\ \39\|j,\ \39\|k,\ \39\|l,\ \39\>{va});$\7
$\}$\2\5
\WC{ end of "spin" loop }\WY\Wendc
\WU section~\M{1}.
\fi % End of section 3

\WM4. Only the top triangle of the G matrix has been formed
explicitly, fill up the other half.

\WY\WP\4\4\WX{\M{4}}Symmetrise the G matrix\X \X${}\WSQ{}$\6
\&{do} $\|i=\WO{1},\ \39\>{n2};$\1\7
$\{$\7
\&{do} $\|j=\WO{1},\ \39\|i;$\1\7
$\{$\7
$\>{ij}=\>{n2}\ast(\|j-\WO{1})+\|i;$\6
$\>{ji}=\>{n2}\ast(\|i-\WO{1})+\|j;$\6
$\&{if}\,(\|i\WS\|j)$\1\6
\&{next};\2\5
\WC{ omit the diagonals }\6
$\|G(\>{ji})=\|G(\>{ij});$\7
$\}$\2\6
$\}$\2\WY\Wendc
\WU section~\M{1}.
\fi % End of section 4

\WN5.  INDEX.
\fi % End of section 5

\input INDEX.tex
\input MODULES.tex

\Winfo{"fweave scfgr-1.web"}  {"scfgr-1.web"} {(none)}
 {\Ratfor}


\Wcon{5}
\FWEBend
