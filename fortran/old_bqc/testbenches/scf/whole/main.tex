% FWEAVE v1.62 (September 23, 1998)

% --- FWEB's macro package ---
\input fwebmac.sty

% --- Initialization parameters from FWEB's style file `fweb.sty' ---
\Wbegin[;]
  % #1 --- [LaTeX.class.options;LaTeX.package.options]
{article;}
  % #2 --- {LaTeX.class;LaTeX.package}
{1em}
  % #3 --- {indent.TeX}
{1em}
  % #4 --- {indent.code}
{CONTENTS.tex}
  % #5 --- {contents.TeX}
{ % #6 ---
 {\&\WRS}
  % #1 --- {{format.reserved}{format.RESERVED}}
 {\|}
  % #2 --- {format.short_id}
 {\>\WUC}
  % #3 --- {{format.id}{format.ID}}
 {\>\WUC}
  % #4 --- {{format.outer_macro}{format.OUTER_MACRO}}
 {\>\WUC}
  % #5 --- {{format.WEB_macro}{format.WEB_MACRO}}
 {\@}
  % #6 --- {format.intrinsic}
 {\.\.}
  % #7 --- {{format.keyword}{format.KEYWORD}}
 {\.}
  % #8 --- {format.typewriter}
 {}
  % #9 --- (For future use)
}
{\M}
  % #7 --- {encap.prefix}
{;}
  % #8 --- {doc.preamble;doc.postamble}
{INDEX}
  % #9 --- {index.name}


% --- Beginning of user's limbo section ---

\def\title{--- MAIN TEST BENCH ---}




% --- Limbo text from style-file parameter `limbo.end' ---
\FWEBtoc

\WN1.  INTRODUCTION. This testbench program is the test for UHF (rather DODS)
calculation of H2O, including generation of 1- and 2-electron integrals.
The ERI are stored in the file \WCD{ $\>{fort}\WO{.17}$} which is a binary file
of size 8K.
The working molecule is water in STO3G minimal basis set.

\fi % End of section 1 (sect. 1, p. 1)

\WM2.

\fi % End of section 2 (sect. 1.0.0.1, p. 1a)

\WN3.  DODS.

\WY\WP \Wunnamed{defs}{main.f}%
\WMd{}\WUC{YES}\5
$\WO{0}$\Wendd
\WP\WMd{}\WUC{NO}\5
$\WO{100}$\WY\Wendd
\WP\WMd{}\.{ERR}\5
${-}\WO{10}$\Wendd
\WP\WMd{}\WUC{OK}\5
$\WO{10}$\WY\Wendd
\WP\WMd{}\WUC{END\_OF\_FILE}\5
${-}\WO{1}$\Wendd
\WP\WMd{}\WUC{NOT\_END\_OF\_FILE}\5
$\WO{55}$\WY\Wendd
\WP\WMd{}\WUC{LAST\_BLOCK}\5
$\WO{12}$\Wendd
\WP\WMd{}\WUC{NOT\_LAST\_BLOCK}\5
${-}\WO{12}$\WY\Wendd
\WP\WMd{}\WUC{ARB}\5
$\WO{1}$\WY\Wendd
\WP\WMd{}\WUC{BYTES\_PER\_INTEGER}\5
$\WO{4}$\Wendd
\WP\WMd{}\WUC{LEAST\_BYTE}\5
$\WO{1}$\WY\Wendd
\WP\WMd{}\WUC{NO\_OF\_TYPES}\5
$\WO{20}$\Wendd
\WP\WMd{}\WUC{INT\_BLOCK\_SIZE}\5
$\WO{20000}$\WY\Wendd
\WP\WMd{}\WUC{MAX\_BASIS\_FUNCTIONS}\5
$\WO{255}$\Wendd
\WP\WMd{}\WUC{MAX\_PRIMITIVES}\5
$\WO{1000}$\Wendd
\WP\WMd{}\WUC{MAX\_CENTRES}\5
$\WO{50}$\WY\Wendd
\WP\WMd{}\WUC{MAX\_ITERATIONS}\5
$\WO{60}$\WY\Wendd
\WP\WMd{}\WUC{UHF\_CALCULATION}\5
$\WO{11}$\Wendd
\WP\WMd{}\WUC{CLOSED\_SHELL\_CALCULATION}\5
$\WO{21}$\WY\Wendd
\WP\WMd{}\WUC{MATRIX\_SIZE}\5
$\WO{20000}$\WY\Wendd
\WP\WMd{}\WUC{ERROR\_OUTPUT\_UNIT}\5
$\WO{61}$\Wendd
\WP\WMd{}\WUC{ERI\_UNIT}\5
$\WO{17}$\WY\Wendd
\WY\WP \Wunnamed{code}{main.f}%
 \&{program} \1\>{calcDODS}\2\5
\WC{           TESTBENCH FOR WATER MOLECULE - GENERATING INTEGRALS AND
CALCULATION OF WFN     }\6
\&{double} \&{precision}~\1\>{vlist}$(\WUC{MAX\_CENTRES},\ \39\WO{4})$\2\6
\&{double} \&{precision}~\1\>{eta}$(\WUC{MAX\_PRIMITIVES},\ \39\WO{5})$\2\6
\&{integer}~\1\>{nfirst}$(\WO{7})$\2\6
\&{integer}~\1\>{nlast}$(\WO{7})$\2\6
\&{integer}~\1\>{ntype}$(\WO{7})$\2\6
\&{integer}~\1\>{ncntr}$(\WO{7})$\2\6
\&{integer}~\1\>{nr}$(\WUC{NO\_OF\_TYPES},\ \39\WO{3})$\2\6
\&{integer}~\1\>{nbfns}$,$ \>{ngmx}$,$ \>{noc}$,$ \>{nfile}$,$ \>{ncmx}\2\7
\&{integer}~\1\|i$,$ \|j\2\6
\&{double} \&{precision}~\1\>{vlist1}$(\WO{4}),$ \>{vlist2}$(\WO{4}),$ %
\>{vlist3}$(\WO{4})$\2\6
\&{double} \&{precision}~\1\|u$(\WO{5})$\2\6
\&{double} \&{precision}~\1\>{hydrE}$(\WO{3}),$ \>{hydrC}$(\WO{3})$\2\6
\&{double} \&{precision}~\1\>{oxygE}$(\WO{15}),$ \>{oxygC}$(\WO{15})$\2\6
\&{double} \&{precision}~\1\|S$(\WO{1000}),$ \|H$(\WO{1000}),$ \WUC{HF}$(%
\WO{1000}),$ \|R$(\WO{1000}),$ \>{Rold}$(\WO{1000})$\2\6
\&{double} \&{precision}~\1\|C$(\WO{1000}),$ \>{Cbar}$(\WO{1000}),$ \|V$(%
\WO{1000})$\2\7
\&{double} \&{precision}~\1\>{crit}$,$ \>{damp}$,$ \|E\2\6
\&{double} \&{precision}~\1\>{epsilon}$(\WO{100})$\2\6
\&{integer}~\1\>{scf}\2\6
\&{integer}~\1\>{nelec}$,$ \>{nbasis}$,$ \>{interp}$,$ \>{irite}\2\7
\&{data} ~\1\>{nr}${/}\WO{0},\ \39\WO{1},\ \39\WO{0},\ \39\WO{0},\ \39\WO{2},\ %
\39\WO{0},\ \39\WO{0},\ \39\WO{1},\ \39\WO{1},\ \39\WO{0},\ \39\WO{3},\ \39%
\WO{0},\ \39\WO{0},\ \39\WO{2},\ \39\WO{2},\ \39\WO{1},\ \39\WO{0},\ \39\WO{1},%
\ \39\WO{0},\ \39\WO{1},\ \39\WO{0},\ \39\WO{0},\ \39\WO{1},\ \39\WO{0},\ \39%
\WO{0},\ \39\WO{2},\ \39\WO{0},\ \39\WO{1},\ \39\WO{0},\ \39\WO{1},\ \39\WO{0},%
\ \39\WO{3},\ \39\WO{0},\ \39\WO{1},\ \39\WO{0},\ \39\WO{2},\ \39\WO{2},\ \39%
\WO{0},\ \39\WO{1},\ \39\WO{1},\ \39\WO{0},\ \39\WO{0},\ \39\WO{0},\ \39\WO{1},%
\ \39\WO{0},\ \39\WO{0},\ \39\WO{2},\ \39\WO{0},\ \39\WO{1},\ \39\WO{1},\ \39%
\WO{0},\ \39\WO{0},\ \39\WO{3},\ \39\WO{0},\ \39\WO{1},\ \39\WO{0},\ \39\WO{1},%
\ \39\WO{2},\ \39\WO{2},\ \39\WO{1}{/}$\2\5
\WC{ Coordinates of water in Bohr }\6
\&{data} ~\1\>{noc}${/}\WO{3}{/}$\2\6
\&{data} ~\1\>{vlist1}${/}\WO{0.00000000},\ \39\WO{0.00000000},\ \39%
\WO{1.79523969},\ \39\WO{8.0000}{/}$\2\6
\&{data} ~\1\>{vlist2}${/}\WO{0.00000000},\ \39\WO{0.00000000},\ \39%
\WO{0.00000000},\ \39\WO{1.0000}{/}$\2\6
\&{data} ~\1\>{vlist3}${/}\WO{1.69257088},\ \39\WO{0.00000000},\ \39%
\WO{2.39364599},\ \39\WO{1.0000}{/}$\2\7
\&{do} $\|i=\WO{1},\ \39\WO{4}$\1\6
$\>{vlist}(\WO{1},\ \39\|i)=\>{vlist1}(\|i)$\6
$\>{vlist}(\WO{2},\ \39\|i)=\>{vlist2}(\|i)$\6
$\>{vlist}(\WO{3},\ \39\|i)=\>{vlist3}(\|i)$\2\6
\&{end} \&{do}\5
\Wc{ basis set                 C Hydrogen 1s }\6
\&{data} ~\1\>{hydrE}${/}\WO{3.42525091},\ \39\WO{0.62391373},\ \39%
\WO{0.16885540}{/}$\2\6
\&{data} ~\1\>{hydrC}${/}\WO{0.15432897},\ \39\WO{0.53532814},\ \39%
\WO{0.44463454}{/}$\2\5
\WC{ C Oxygen 1S 2S 2PX 2PY 2PZ }\6
\&{data} ~\1\>{oxygE}${/}\WO{130.7093200},\ \39\WO{23.8088610},\ \39%
\WO{6.4436083},\ \39\WO{5.0331513},\ \39\WO{1.1695961},\ \39\WO{0.3803890},\ %
\39\WO{5.0331513},\ \39\WO{1.1695961},\ \39\WO{0.3803890},\ \39\WO{5.0331513},\
\39\WO{1.1695961},\ \39\WO{0.3803890},\ \39\WO{5.0331513},\ \39\WO{1.1695961},\
\39\WO{0.3803890}{/}$\2\6
\&{data} ~\1\>{oxygC}${/}\WO{0.15432897},\ \39\WO{0.53532814},\ \39%
\WO{0.44463454},\ \39{-}\WO{0.09996723},\ \39\WO{0.39951283},\ \39%
\WO{0.70011547},\ \39\WO{1.15591627},\ \39\WO{0.60768372},\ \39\WO{0.39195739},%
\ \39\WO{0.15591627},\ \39\WO{0.60768372},\ \39\WO{0.39195739},\ \39%
\WO{0.15591627},\ \39\WO{0.60768372},\ \39\WO{0.39195739}{/}$\2\5
\WC{ Do the primitive GTOs in eta }\6
\&{do} $\|i=\WO{1},\ \39\WO{3}$\1\5
\WC{ oxygen 1 }\6
\&{do} $\|j=\WO{1},\ \39\WO{15}$\1\6
$\>{eta}(\|j,\ \39\|i)=\>{vlist1}(\|i)$\6
$\>{eta}(\|j,\ \39\WO{4})=\>{oxygE}(\|j)$\6
$\>{eta}(\|j,\ \39\WO{5})=\>{oxygC}(\|j)$\2\6
\&{end} \&{do}\5
\WC{ hydrogen 2 }\6
\&{do} $\|j=\WO{16},\ \39\WO{18}$\1\6
$\>{eta}(\|j,\ \39\|i)=\>{vlist2}(\|i)$\6
$\>{eta}(\|j,\ \39\WO{4})=\>{hydrE}(\|j-\WO{15})$\6
$\>{eta}(\|j,\ \39\WO{5})=\>{hydrC}(\|j-\WO{15})$\2\6
\&{end} \&{do}\5
\WC{ hydrogen 3 }\6
\&{do} $\|j=\WO{19},\ \39\WO{21}$\1\6
$\>{eta}(\|j,\ \39\|i)=\>{vlist3}(\|i)$\6
$\>{eta}(\|j,\ \39\WO{4})=\>{hydrE}(\|j-\WO{18})$\6
$\>{eta}(\|j,\ \39\WO{5})=\>{hydrC}(\|j-\WO{18})$\2\6
\&{end} \&{do}\2\6
\&{end} \&{do}\5
\WC{ specification of contraction }\6
\&{data} ~\1\>{nfirst}${/}\WO{1},\ \39\WO{4},\ \39\WO{7},\ \39\WO{10},\ \39%
\WO{13},\ \39\WO{16},\ \39\WO{19}{/}$\2\6
\&{data} ~\1\>{nlast}${/}\WO{3},\ \39\WO{6},\ \39\WO{9},\ \39\WO{12},\ \39%
\WO{15},\ \39\WO{18},\ \39\WO{21}{/}$\2\5
\WC{ types of basis functions: s,s,px,py,pz,s,s }\6
\&{data} ~\1\>{ntype}${/}\WO{1},\ \39\WO{1},\ \39\WO{2},\ \39\WO{3},\ \39%
\WO{4},\ \39\WO{1},\ \39\WO{1}{/}$\2\5
\WC{ nuclear center and basis function }\6
\&{data} ~\1\>{ncntr}${/}\WO{1},\ \39\WO{1},\ \39\WO{1},\ \39\WO{1},\ \39%
\WO{1},\ \39\WO{2},\ \39\WO{3}{/}$\2\6
\&{data} ~\1\>{ngmx}${/}\WO{21}{/}$\2\6
\&{data} ~\1\>{nbfns}${/}\WO{7}{/}$\2\6
\&{data} ~\1\>{ncmx}${/}\WO{3}{/}$\2\7
$\&{write}\,(\ast,\ \39\ast)$ $\.{"\ Provide\ the\ number\ of\ el\0ectrons:\
"}$\6
\&{read}$\,(\ast,\ \39\ast)$ \>{nelec}\6
$\>{nbasis}=\WO{7}$\6
$\>{irite}=\WO{12}$\6
$\>{nfile}=\WUC{ERI\_UNIT}$\6
$\>{crit}=\WO{1.00\^D-06}$\6
$\>{damp}=\WO{0.13\^D+00}$\6
$\>{interp}=\WO{0}$\7
\&{call} $\>{genint}(\>{ngmx},\ \39\>{nbasis},\ \39\>{eta},\ \39\>{ntype},\ \39%
\>{ncntr},\ \39\>{nfirst},\ \39\>{nlast},\ \39\>{vlist},\ \39\>{ncmx},\ \39%
\>{noc},\ \39\|S,\ \39\|H,\ \39\>{nfile})$\7
\&{call} $\>{shalf}(\|S,\ \39\|R,\ \39\>{Cbar},\ \39\>{nbasis})$\7
\WC{ Perform SCF ��! }\6
$\&{if}\,(\>{scf}(\|H,\ \39\|S,\ \39\>{nbasis},\ \39\>{nelec},\ \39\>{nfile},\ %
\39\>{irite},\ \39\>{damp},\ \39\>{interp},\ \39\|E,\ \39\WUC{HF},\ \39\|V,\ %
\39\|R,\ \39\>{Rold},\ \39\>{Cbar},\ \39\>{epsilon},\ \39\>{crit})\WS%
\WUC{YES})$ \&{then}\1\6
$\&{write}\,(\ast,\ \39\ast)$ $\.{"\ SUCCESS"}$\2\6
\&{else}\1\6
$\&{write}\,(\ast,\ \39\ast)$ $\.{"\ You\ have\ to\ work\ it\ more\0\ ..."}$\2\6
\&{end} \&{if}\7
\WUC{STOP}\6
\.{END}\Wendc
\fi % End of section 3 (sect. 2, p. 2)

\WM4.


\fi % End of section 4 (sect. 2.0.0.1, p. 4)

\WN5.  SCF.

This is  Version 1 of the Hartree-Fock theory implemented
for closed shells (RHF) and open shells (UHF-DODS) calculations.

\ \\ \ \\
\begin{minipage}{4.5in}
\ \\
\begin{description}
\item[NAME] \    SCF     \\
 Perform LCAO-MO-SCF calculation on a molecule.

\item[SYNOPSIS] \     \\
 {\tt double precision function scf(H, C, nbasis, nelec, nfile, \\
                              irite, damp, interp, E, HF, V, R, Rold, Ubar,
eps, crit)
   \ \\
  integer nbasis, nelec, nfile, irite \\
  double precision damp, E \\
  double precision H(ARB), C(ARB), HF(ARB), V(ARB), R(ARB) \\
  double precision Rold(ARB), Ubar(ARB), eps(ARB) \\
 }

\item[DESCRIPTION] \  \\
 Perform LCAO-MO calculation of either closed-shell RHF type or more general
open-shell (real) UHF-DODS type. The method is traditional Roothan repeated
diagonalizations of Hartree-Fock matrix until self-consistency is reached:
\[
   {\bf F} \cdot {\bf C} = {\bf S} \cdot {\bf C} \cdot {\boldmath \epsilon}
\]

\item[ARGUMENTS] \    \\
\begin{description}
\item[H] Input: One-electron Hamiltonian of size ({\tt nbasis x nbasis}), i.e.,
                matrix elements of one-electron operator
\item[C] Input/Output: An initial MO matrix - it must at least
                orthigonalize the basis. Normally, it is simply the
orthogonalization
                matrix ${\bf S}^{-\frac{1}{2}}$. On output the SCF {\bf C}
matrix
                is placed here.
\item[nbasis] Input: the number of \emph{spatial} orbitals in the basis (i.e.,
half
                of the number of the spin-basis set functions if {\tt nelec} $%
\ge 0$)
\item[nelec]  Input: The number of electrons in the system.
\item[nfile]  The electron-repulsion file unit.
\item[itite]  Channel number for convergence information or zero if this
information
              is not necessary.
\item[damp]   Hartree-Fock damping parameter.
\item[interp] Interpolation parameter. If 0 no interpolation will be
undertaken.
\item[HF]     Output: for use as the Fock matrix
\item[V]      Workspace:
\item[R]      Output: Density matrix
\item[Rold]   Workspace:
\item[Ubar]   Workspace:
\item[eps]    Output: orbital energies (first {\tt nelec} are the occupied
orbitals)
\item[E]      Output: Total HF electronic energy
\item[crit]   Convergence of the SCF procedure
\end{description}

\item[RETURNS] \    \\
%\begin{description}
   {\tt YES} if the calculation is converged in {\tt MAX\_SCF\_ITERATIONS} \\
   {\tt NO } if no convergence is met. Typical usage:
   {\tt if ( SCF(......) .EQ. YES ) then \\
               output succesful calculation
   }
%\end{description}

\item[SEE ALSO] \    \\
%\begin{description}
 {\tt scfR, scfGR, eigen}
%\end{description}

\item[DIAGNOSTICS] \  \\
Happy!
\end{description}
\ \\ \ \\
\end{minipage}
\ \\ \ \\

\WY\WP \Wunnamed{defs}{main.f}%
\WMd{}\WUC{MAX\_ITERATIONS}\5
$\WO{50}$\WY\Wendd
\WY\WP \Wunnamed{code}{main.f}%
\7
\&{integer} \&{function} \1$\WUC{SCF}(\|H,\ \39\|C,\ \39\>{nbasis},\ \39%
\>{nelec},\ \39\>{nfile},\ \39\>{irite},\ \39\>{damp},\ \39\>{interp},\ \39\|E,%
\ \39\WUC{HF},\ \39\|V,\ \39\|R,\ \39\>{Rold},\ \39\>{Cbar},\ \39\>{epsilon},\ %
\39\>{crit})$\2\1\7
\WX{\M{6}}Global SCF Declarations\X \X\7
\WX{\M{7}}Internal SCF Declarations\X \X\7
\WX{\M{8}}Select SCF Type\X \X\7
\WX{\M{9}}Set initial matrices and counters\X \X\7
\&{do} $\>{while}((\>{icon}\WI\WO{0})\WW(\>{kount}<\WUC{MAX\_ITERATIONS}))$\1\7
\WX{\M{10}}Sigle SCF iteration\X \X\2\7
\&{end} \&{do}\7
\WX{\M{11}}Write the output result\X \X\7
\WX{\M{12}}Formats\X \X\7
\&{return}\2\6
\&{end}\WY\Wendc
\fi % End of section 5 (sect. 3, p. 5)

\WM6.

\WY\WP\4\4\WX{\M{6}}Global SCF Declarations\X \X${}\WSQ{}$\6
\&{implicit} \1\&{double} \&{precision}$\,(\|a-\|h,\ \39\|o-\|z)$\2\6
\&{integer}~\1\>{nbasis}$,$ \>{nelec}$,$ \>{nfile}$,$ \>{irite}\2\6
\&{integer}~\1\>{interp}\2\6
\&{double} \&{precision}~\1\|H$(\WUC{ARB}),$ \|C$(\WUC{ARB}),$ \WUC{HF}$(%
\WUC{ARB}),$ \|V$(\WUC{ARB}),$ \|R$(\WUC{ARB})$\2\6
\&{double} \&{precision}~\1\>{Rold}$(\WUC{ARB}),$ \>{Cbar}$(\WUC{ARB})$\2\6
\&{double} \&{precision}~\1\>{epsilon}$(\WUC{ARB})$\2\6
\&{double} \&{precision}~\1\|E$,$ \>{damp}$,$ \>{crit}\2\Wendc
\WU section~\M{5}.
\fi % End of section 6 (sect. 3.0.0.1, p. 7a)

\WM7.

\WY\WP\4\4\WX{\M{7}}Internal SCF Declarations\X \X${}\WSQ{}$\6
\&{integer}~\1\>{scftype}$,$ \>{kount}$,$ \>{maxit}$,$ \>{nocc}$,$ \|m$,$ %
\>{mm}$,$ \|i\2\6
\&{double} \&{precision}~\1\>{term}$,$ \>{turm}$,$ \>{Rsum}\2\6
\&{double} \&{precision}~\1\>{zero}$,$ \>{half}\2\6
\&{data} ~\1\>{zero}$,$ \>{half}${/}\WO{0.0\^D+00},\ \39\WO{0.5\^D+00}{/}$\2%
\Wendc
\WU section~\M{5}.
\fi % End of section 7 (sect. 3.0.0.2, p. 7b)

\WM8.

\WY\WP\4\4\WX{\M{8}}Select SCF Type\X \X${}\WSQ{}$\6
$\&{if}\,(\>{nelec}>\>{zero})$ \&{then}\1\5
\WC{ closed shell case }\6
$\>{scftype}=\WUC{CLOSED\_SHELL\_CALCULATION}$\6
$\>{nocc}=\>{nelec}\WSl\WO{2}$\6
$\|m=\>{nbasis}$\6
$\WUC{WRITE}(\ast,\ \39\ast)\.{"RHF\ CALCULATION\ CHOSEN"}$\2\6
\&{else}\1\5
\WC{ open shell case }\6
$\>{scftype}=\WUC{UHF\_CALCULATION}$\6
$\>{nocc}=\@{abs}(\>{nelec})$\6
$\|m=\>{nbasis}\ast\WO{2}$\6
\&{call} $\>{spinor}(\|H,\ \39\>{nbasis})$\6
\&{call} $\>{spinor}(\|C,\ \39\>{nbasis})$\6
$\WUC{WRITE}(\ast,\ \39\ast)\.{"UHF\ CALCULATION\ CHOSEN"}$\2\6
\&{end} \&{if}\Wendc
\WU section~\M{5}.
\fi % End of section 8 (sect. 3.0.0.3, p. 8a)

\WM9.

\WY\WP\4\4\WX{\M{9}}Set initial matrices and counters\X \X${}\WSQ{}$\6
\WC{ basis set size }\6
$\>{mm}=\|m\ast\|m$\6
\&{do} $\|i=\WO{1},\ \39\>{mm}$\1\6
$\|R(\|i)=\>{zero};$\6
$\>{Rold}(\|i)=\>{zero}$\2\6
\&{end} \&{do}\6
$\WUC{SCF}=\WUC{YES}$\6
$\>{kount}=\WO{0}$\6
$\>{icon}=\WO{100}$\Wendc
\WU section~\M{5}.
\fi % End of section 9 (sect. 3.0.0.4, p. 8b)

\WM10.

\WY\WP\4\4\WX{\M{10}}Sigle SCF iteration\X \X${}\WSQ{}$\6
$\>{kount}=\>{kount}+\WO{1}$\6
$\|E=\>{zero};$\6
$\>{icon}=\WO{0}$\6
\&{do} $\|i=\WO{1},\ \39\>{mm}$\1\6
$\WUC{HF}(\|i)=\|H(\|i)$\6
$\|E=\|E+\|R(\|i)\ast\WUC{HF}(\|i)$\2\6
\&{enddo}\6
\&{call} $\>{scfGR}(\|R,\ \39\WUC{HF},\ \39\>{nbasis},\ \39\>{nfile},\ \39%
\>{scftype})$\6
\&{do} $\|i=\WO{1},\ \39\>{mm}$\1\6
$\|E=\|E+\|R(\|i)\ast\WUC{HF}(\|i)$\2\6
\&{enddo}\7
$\&{if}\,(\>{scftype}\WS\WUC{UHF\_CALCULATION})$\1\6
$\|E=\>{half}\ast\|E$\2\7
$\&{write}\,(\WUC{ERROR\_OUTPUT\_UNIT},\ \39\WO{200})$ \|E\7
\&{call} $\>{gtprd}(\|C,\ \39\WUC{HF},\ \39\|R,\ \39\|m,\ \39\|m,\ \39\|m)$\6
\&{call} $\>{gmprd}(\|R,\ \39\|C,\ \39\WUC{HF},\ \39\|m,\ \39\|m,\ \39\|m)$\6
\&{call} $\>{eigen}(\WUC{HF},\ \39\>{Cbar},\ \39\|m)$\6
\&{do} $\|i=\WO{1},\ \39\|m$\1\6
$\>{epsilon}(\|i)=\WUC{HF}(\|m\ast(\|i-\WO{1})+\|i)$\2\6
\&{enddo}\6
\&{call} $\>{gmprd}(\|C,\ \39\>{Cbar},\ \39\|V,\ \39\|m,\ \39\|m,\ \39\|m)$\6
\&{call} $\>{scfR}(\|V,\ \39\|R,\ \39\|m,\ \39\>{nocc})$\6
$\>{Rsum}=\>{zero}$\6
\&{do} $\|i=\WO{1},\ \39\>{mm}$\1\6
$\>{turm}=\|R(\|i)-\>{Rold}(\|i)$\6
$\>{term}=\@{dabs}(\>{turm})$\6
$\>{Rold}(\|i)=\|R(\|i)$\6
$\|C(\|i)=\|V(\|i)$\6
$\&{if}\,(\>{term}>\>{crit})$\1\6
$\>{icon}=\>{icon}+\WO{1}$\2\6
$\>{Rsum}=\>{Rsum}+\>{term}$\6
$\&{if}\,(\>{kount}<\>{interp})$\1\6
$\|R(\|i)=\|R(\|i)-\>{damp}\ast\>{turm}$\2\2\6
\&{enddo}\Wendc
\WU section~\M{5}.
\fi % End of section 10 (sect. 3.0.0.5, p. 9)

\WM11.

\WY\WP\4\4\WX{\M{11}}Write the output result\X \X${}\WSQ{}$\6
$\&{write}\,(\WUC{ERROR\_OUTPUT\_UNIT},\ \39\WO{201})$ \>{Rsum}$,$ \>{icon}\7
$\&{if}\,((\>{kount}\WS\WUC{MAX\_ITERATIONS})\WW(\>{icon}\WI\WO{0}))$ \&{then}%
\1\6
$\&{write}\,(\WUC{ERROR\_OUTPUT\_UNIT},\ \39\WO{204})$ \6
$\WUC{SCF}=\WUC{NO}$\2\6
\&{else}\1\6
$\&{write}\,(\WUC{ERROR\_OUTPUT\_UNIT},\ \39\WO{202})$ \>{kount}\6
\&{write}$\,(\WUC{ERROR\_OUTPUT\_UNIT},\ \39\WO{203})$ $(\>{epsilon}(\|i),\ \39%
\|i=\WO{1},\ \39\>{nocc})$\2\6
\&{endif}\Wendc
\WU section~\M{5}.
\fi % End of section 11 (sect. 3.0.0.6, p. 10a)

\WM12.

\WY\WP\4\4\WX{\M{12}}Formats\X \X${}\WSQ{}$\6
\Wlbl{\WO{200}\Colon\ }$\&{format}\,(\.{"\ Current\ Electronic\ Energ\0y\ =\
"},\ \39\>{f12.6})$ \6
\Wlbl{\WO{201}\Colon\ }$\&{format}\,(\.{"\ Convergence\ in\ R\ =\ "},\ \39%
\>{f12.5},\ \39\>{i6},\ \39\.{"\ \ Changing"})$ \6
\Wlbl{\WO{202}\Colon\ }$\&{format}\,(\.{"\ SCF\ converged\ in"},\ \39\>{i4},\ %
\39\.{"\ iterations"})$ \6
\Wlbl{\WO{203}\Colon\ }$\&{format}\,(\.{"\ Orbital\ Energies\ "},\ \39(\WO{7}%
\>{f10.5}))$ \6
\Wlbl{\WO{204}\Colon\ }$\&{format}\,(\.{"\ SCF\ did\ not\ converged...\0\
quitting"})$ \Wendc
\WU section~\M{5}.
\fi % End of section 12 (sect. 3.0.0.7, p. 10b)

\WM13.

\fi % End of section 13 (sect. 3.0.0.8, p. 10c)

\WN14. 1 scfGR.

\WY\WP \Wunnamed{defs}{main.f}%
\WMd{}\>{locGR}$(\|i,\|j)$\5
$(\|m\ast(\|j-\WO{1})+\|i)$\Wendd
\WY\WP \Wunnamed{code}{main.f}%
 \&{subroutine} \1$\>{scfGR}(\|R,\ \39\|G,\ \39%
\|n,\ \39\>{nfile},\ \39\>{ntype})$\2 \&{double} \&{precision}~\1\|R$(\ast),$ %
\|G$(\ast)$\2\6
\&{integer}~\1\|m$,$ \|n$,$ \>{nfile}$,$ \>{ntype}\2\5
\WC{ m: total basis size              n: spatial basis size }\7
\&{double} \&{precision}~\1\>{val}\2\6
\&{integer}~\1\|i$,$ \|j$,$ \|k$,$ \|l$,$ \>{is}$,$ \>{js}$,$ \>{ks}$,$ %
\>{ls}$,$ \>{ijs}$,$ \>{kls}$,$ \>{mu}\2\6
\&{integer}~\1\>{getint}\2\6
\&{double} \&{precision}~\1\>{zero}$,$ \>{one}$,$ \>{cJ}$,$ \>{cK}\2\6
\&{integer}~\1\>{pointer}$,$ \>{spin}$,$ \>{skip}\2\6
\&{data} ~\1\>{zero}$,$ \>{one}$,$ \>{two}${/}\WO{0.0\^D+00},\ \39\WO{1.0%
\^D+00},\ \39\WO{2.0\^D+00}{/}$\2\7
\&{rewind} \>{nfile}\6
$\>{pointer}=\WO{0}$\7
\WX{\M{15}}Establish the type of calculation\X \X\7
\&{do} $\>{while}(\>{getint}(\>{nfile},\ \39\>{is},\ \39\>{js},\ \39\>{ks},\ %
\39\>{ls},\ \39\>{mu},\ \39\>{val},\ \39\>{pointer})\WI\WUC{END\_OF\_FILE})$\1\7
$\>{ijs}=\>{is}\ast(\>{is}-\WO{1})\WSl\WO{2}+\>{js}$\6
$\>{kls}=\>{ks}\ast(\>{ks}-\WO{1})\WSl\WO{2}+\>{ls}$\7
\&{do} $\>{spin}=\WO{1},\ \39\WO{4}$\1\7
\WX{\M{16}}Check the UHF or RHF case\X \X\6
$\>{skip}=\WUC{NO}$\7
$\>{select}\>{case}(\>{spin})$\6
$\>{case}(\WO{1})$\6
$\|i=\>{is};$\6
$\|j=\>{js};$\6
$\|k=\>{ks};$\6
$\|l=\>{ls}$\6
$\>{case}(\WO{2})$\6
$\|i=\>{is}+\|n;$\6
$\|j=\>{js}+\|n;$\6
$\|k=\>{ks}+\|n;$\6
$\|l=\>{ls}+\|n$\6
$\>{case}(\WO{3})$\6
$\|i=\>{is}+\|n;$\6
$\|j=\>{js}+\|n;$\6
$\|k=\>{ks};$\6
$\|l=\>{ls}$\6
$\>{case}(\WO{4})$\6
$\&{if}\,(\>{ijs}\WS\>{kls})$\1\6
$\>{skip}=\WUC{YES}$\2\6
$\|i=\>{is};$\6
$\|j=\>{js};$\6
$\|k=\>{ks}+\|n;$\6
$\|l=\>{ls}+\|n$\6
\&{call} $\>{order}(\|i,\ \39\|j,\ \39\|k,\ \39\|l)$ \&{end} \>{select}\7
$\&{if}\,(\>{skip}\WS\WUC{YES})$\1\6
\>{cycle}\2\7
$\>{cK}=\>{one}$\6
$\&{if}\,(\>{spin}\WG\WO{3})$\1\6
$\>{cK}=\>{zero}$\2\7
\&{call} $\>{GofR}(\|R,\ \39\|G,\ \39\|m,\ \39\>{cJ},\ \39\>{cK},\ \39\|i,\ \39%
\|j,\ \39\|k,\ \39\|l,\ \39\>{val})$\2\6
\&{end} \&{do}\2\6
\&{enddo}\7
\WX{\M{17}}Symmetrize G matrix\X \X\7
\&{return} \&{end}\Wendc
\fi % End of section 14 (sect. 3.1, p. 11)

\WM15.

\WY\WP\4\4\WX{\M{15}}Establish the type of calculation\X \X${}\WSQ{}$\6
$\&{if}\,(\>{ntype}\WS\WUC{CLOSED\_SHELL\_CALCULATION})$ \&{then}\1\5
\WC{ RHF case }\6
$\|m=\|n$\5
\WC{ size of basis: spatial basis }\6
$\>{cJ}=\>{two}$\6
$\>{cK}=\>{one}$\5
\WC{ G(R) = 2J(R) - K(R) }\2\6
\&{else}\1\5
\WC{ UHF case }\6
$\|m=\WO{2}\ast\|n$\5
\WC{ size of basis: spin basis }\6
$\>{cJ}=\>{one}$\6
$\>{cK}=\>{one}$\5
\WC{ G(R) = J(R) - K(R) }\2\6
\&{end} \&{if}\Wendc
\WU section~\M{14}.
\fi % End of section 15 (sect. 3.1.0.1, p. 12a)

\WM16.

\WY\WP\4\4\WX{\M{16}}Check the UHF or RHF case\X \X${}\WSQ{}$\6
$\&{if}\,((\>{spin}>\WO{1})\WW(\>{ntype}\WS\WUC{CLOSED\_SHELL\_CALCULATION}))$%
\1\6
\>{exit}\2\Wendc
\WU section~\M{14}.
\fi % End of section 16 (sect. 3.1.0.2, p. 12b)

\WM17.

\WY\WP\4\4\WX{\M{17}}Symmetrize G matrix\X \X${}\WSQ{}$\6
\&{do} $\|i=\WO{1},\ \39\|m$\1\6
\&{do} $\|j=\WO{1},\ \39\|i-\WO{1}$\1\6
$\>{ij}=\>{locGR}(\|i,\ \39\|j);$\6
$\>{ji}=\>{locGR}(\|j,\ \39\|i)$\6
$\|G(\>{ji})=\|G(\>{ij})$\2\6
\&{end} \&{do}\2\6
\&{end} \&{do}\Wendc
\WU section~\M{14}.
\fi % End of section 17 (sect. 3.1.0.3, p. 12c)

\WM18.

\fi % End of section 18 (sect. 3.1.0.4, p. 12d)

\WN19. 2 GofR.

\WY\WP \Wunnamed{defs}{main.f}%
\WMd{}\>{locGR}$(\|i,\|j)$\5
$(\|m\ast(\|j-\WO{1})+\|i)$\Wendd
\WY\WP \Wunnamed{code}{main.f}%
\&{subroutine} \1$\>{GofR}(\|R,\ \39\|G,\ \39%
\|m,\ \39\|a,\ \39\|b,\ \39\|i,\ \39\|j,\ \39\|k,\ \39\|l,\ \39\>{val})$\2\1\6
\&{double} \&{precision}~\1\|R$(\ast),$ \|G$(\ast)$\2\6
\&{double} \&{precision}~\1\>{val}$,$ \|a$,$ \|b\2\6
\&{integer}~\1\|i$,$ \|j$,$ \|k$,$ \|l$,$ \|m\2\6
\&{integer}~\1\>{ij}$,$ \>{kl}$,$ \>{il}$,$ \>{ik}$,$ \>{jk}$,$ \>{jl}\2\6
\&{double} \&{precision}~\1\>{coul1}$,$ \>{coul2}$,$ \>{coul3}$,$ \>{exch}\2\7
$\>{ij}=\>{locGR}(\|i,\ \39\|j);$\6
$\>{kl}=\>{locGR}(\|k,\ \39\|l)$\6
$\>{il}=\>{locGR}(\|i,\ \39\|l);$\6
$\>{ik}=\>{locGR}(\|i,\ \39\|k)$\6
$\>{jk}=\>{locGR}(\|j,\ \39\|k);$\6
$\>{jl}=\>{locGR}(\|j,\ \39\|l)$\6
$\&{if}\,(\|j<\|k)$\1\6
$\>{jk}=\>{locGR}(\|k,\ \39\|j)$\2\6
$\&{if}\,(\|j<\|l)$\1\6
$\>{jl}=\>{locGR}(\|l,\ \39\|j)$\2\7
$\>{coul1}=\|a\ast\|R(\>{ij})\ast\>{val};$\6
$\>{coul2}=\|a\ast\|R(\>{kl})\ast\>{val};$\6
$\>{exch}=\|b\ast\>{val}$\7
$\&{if}\,(\|k\WI\|l)$ \&{then}\1\6
$\>{coul2}=\>{coul2}+\>{coul2}$\6
$\|G(\>{ik})=\|G(\>{ik})-\|R(\>{jl})\ast\>{exch}$\6
$\&{if}\,((\|i\WI\|j)\WW(\|j\WG\|k))$\1\6
$\|G(\>{jk})=\|G(\>{jk})-\|R(\>{il})\ast\>{exch}$\2\2\6
\&{end} \&{if}\7
$\|G(\>{il})=\|G(\>{il})-\|R(\>{jk})\ast\>{exch};$\6
$\|G(\>{ij})=\|G(\>{ij})+\>{coul2}$\7
$\&{if}\,((\|i\WI\|j)\WW(\|j\WG\|l))$\1\6
$\|G(\>{jl})=\|G(\>{jl})-\|R(\>{ik})\ast\>{exch}$\2\7
$\&{if}\,(\>{ij}\WI\>{kl})$ \&{then}\1\6
$\>{coul3}=\>{coul1}$\6
$\&{if}\,(\|i\WI\|j)$\1\6
$\>{coul3}=\>{coul3}+\>{coul1}$\2\6
$\&{if}\,(\|j\WL\|k)$ \&{then}\1\6
$\|G(\>{jk})=\|G(\>{jk})-\|R(\>{il})\ast\>{exch}$\6
$\&{if}\,((\|i\WI\|j)\WW(\|i\WL\|k))$\1\6
$\|G(\>{ik})=\|G(\>{ik})-\|R(\>{jl})\ast\>{exch}$\2\6
$\&{if}\,((\|k\WI\|l)\WW(\|j\WL\|l))$\1\6
$\|G(\>{jl})=\|G(\>{jl})-\|R(\>{ik})\ast\>{exch}$\2\2\6
\&{end} \&{if}\6
$\|G(\>{kl})=\|G(\>{kl})+\>{coul3}$\2\6
\&{end} \&{if}\7
\&{return}\2\6
\&{end}\Wendc
\fi % End of section 19 (sect. 3.1.1, p. 13)

\WM20.

\fi % End of section 20 (sect. 3.1.1.1, p. 14a)

\WN21. 2 order.

\WY\WP \Wunnamed{code}{main.f}%
\&{subroutine} \1$\>{order}(\|i,\ \39\|j,\ \39%
\|k,\ \39\|l)$\2\1\6
\&{integer}~\1\|i$,$ \|j$,$ \|k$,$ \|l\2\6
\&{integer}~\1\>{integ}\2\7
$\|i=\@{abs}(\|i);$\6
$\|j=\@{abs}(\|j);$\6
$\|k=\@{abs}(\|k);$\6
$\|l=\@{abs}(\|l)$\7
$\&{if}\,(\|i<\|j)$ \&{then}\1\6
$\>{integ}=\|i$\6
$\|i=\|j$\6
$\|j=\>{integ}$\2\6
\&{end} \&{if}\7
$\&{if}\,(\|k<\|l)$ \&{then}\1\6
$\>{integ}=\|k$\6
$\|k=\|l$\6
$\|l=\>{integ}$\2\6
\&{end} \&{if}\7
$\&{if}\,((\|i<\|k)\WOR((\|i\WS\|k)\WW(\|j<\|l)))$ \&{then}\1\6
$\>{integ}=\|i$\6
$\|i=\|k$\6
$\|k=\>{integ}$\6
$\>{integ}=\|j$\6
$\|j=\|l$\6
$\|l=\>{integ}$\2\6
\&{end} \&{if}\7
\&{return}\2\6
\&{end}\Wendc
\fi % End of section 21 (sect. 3.1.2, p. 14b)

\WM22.

\fi % End of section 22 (sect. 3.1.2.1, p. 14c)

\WN23. 1 scfR.

\WY\WP \Wunnamed{code}{main.f}%
\&{subroutine} \1$\>{scfR}(\|C,\ \39\|R,\ \39%
\|m,\ \39\>{nocc})$\2\1\6
\&{double} \&{precision}~\1\|C$(\WUC{ARB}),$ \|R$(\WUC{ARB})$\2\6
\&{integer}~\1\|m$,$ \>{nocc}\2\7
\&{double} \&{precision}~\1\>{suma}$,$ \>{zero}\2\6
\&{integer}~\1\|i$,$ \|j$,$ \|k$,$ \>{ij}$,$ \>{ji}$,$ \>{kk}$,$ \>{ik}$,$ %
\>{jk}\2\6
\&{data} ~\1\>{zero}${/}\WO{0.0\^D+00}{/}$\2\7
\&{do} $\|i=\WO{1},\ \39\|m$\1\6
\&{do} $\|j=\WO{1},\ \39\|i$\1\6
$\>{suma}=\>{zero}$\6
\&{do} $\|k=\WO{1},\ \39\>{nocc}$\1\6
$\>{kk}=\|m\ast(\|k-\WO{1})$\6
$\>{ik}=\>{kk}+\|i$\6
$\>{jk}=\>{kk}+\|j$\6
$\>{suma}=\>{suma}+\|C(\>{ik})\ast\|C(\>{jk})$\2\6
\&{enddo}\6
$\>{ij}=\|m\ast(\|j-\WO{1})+\|i$\6
$\>{ji}=\|m\ast(\|i-\WO{1})+\|j$\6
$\|R(\>{ij})=\>{suma}$\6
$\|R(\>{ji})=\>{suma}$\2\6
\&{enddo}\2\6
\&{enddo}\7
\&{return}\2\6
\&{end}\Wendc
\fi % End of section 23 (sect. 3.2, p. 15a)

\WM24.

\fi % End of section 24 (sect. 3.2.0.1, p. 15b)

\WN25.  INTEGRALS.

\fi % End of section 25 (sect. 4, p. 15c)

\WM26.

\fi % End of section 26 (sect. 4.0.0.1, p. 15d)

\WN27. 1 genoei.
Function to compute the one-electron integrals (overlap,
kinetic energy and nuclear attraction).
The STRUCTURES and GENOEI manual pages must be
consulted for a detailed description of the calling sequence.

The overlap and kinetic energy integrals are expressed in terms of
a basic one-dimensional Cartesian overlap component computed by
\WCD{ \&{function} \>{ovrlap}} while the more involved nuclear-attraction
integrals are computed as a sum of geometrical factors computed by
\WCD{ \&{subroutine} \>{aform}} and the standard $F_\nu$ computed by \WCD{ %
\&{function} \>{fmch}}.


\WY\WP \Wunnamed{code}{main.f}%
 \&{double} \&{precision} \&{function} \1$%
\>{genoei}(\|i,\ \39\|j,\ \39\>{eta},\ \39\>{ngmx},\ \39\>{nfirst},\ \39%
\>{nlast},\ \39\>{ntype},\ \39\>{nr},\ \39\>{ntmx},\ \39\>{vlist},\ \39\>{noc},%
\ \39\>{ncmx},\ \39\>{ovltot},\ \39\>{kintot})$\2 \&{implicit} \1\&{double} %
\&{precision}$\,(\|a-\|h,\ \39\|o-\|z)$\2\6
\&{integer}~\1\|i$,$ \|j$,$ \>{ngmx}$,$ \>{ncmx}$,$ \>{noc}$,$ \>{ntmx}\2\6
\&{integer}~\1\>{nfirst}$(\ast),$ \>{nlast}$(\ast),$ \>{ntype}$(\ast),$ %
\>{nr}$(\>{ntmx},\ \39\WO{3})$\2\6
\&{double} \&{precision}~\1\>{ovltot}$,$ \>{kintot}\2\6
\&{double} \&{precision}~\1\>{eta}$(\WUC{MAX\_PRIMITIVES},\ \39\WO{5}),$ %
\>{vlist}$(\WUC{MAX\_CENTRES},\ \39\WO{4})$\2\7
\WC{ Insert delarations which are purely local to \WCD{ \>{genoei}} }\7
\WX{\M{28}}genoei local declarations\X \X\7
\WC{ Insert the Factorials }\7
\WX{\M{39}}Factorials\X \X\7
\WC{ Obtain the powers of x,y,z and summation limits }\7
\WX{\M{29}}One-electron Integer Setup\X \X\7
\WC{ Inter-nuclear distance }\7
$\>{rAB}=(\>{eta}(\>{iss},\ \39\WO{1})-\>{eta}(\>{jss},\ \39\WO{1}))\WEE{%
\WO{2}}+(\>{eta}(\>{iss},\ \39\WO{2})-\>{eta}(\>{jss},\ \39\WO{2}))\WEE{%
\WO{2}}+(\>{eta}(\>{iss},\ \39\WO{3})-\>{eta}(\>{jss},\ \39\WO{3}))\WEE{%
\WO{2}}$\7
\WC{ Initialise all accumulators   }\7
$\>{genoei}=\>{zero}$\6
$\>{totnai}=\>{zero}$\6
$\>{kintot}=\>{zero}$\6
$\>{ovltot}=\>{zero}$\7
\WC{ Now start the summations over the contracted GTFs  }\7
\&{do} $\>{irun}=\>{iss},\ \39\>{il}$\1\5
\WC{ start of "i" contraction }\7
\&{do} $\>{jrun}=\>{jss},\ \39\>{jl}$\1\5
\WC{ start of "j" contraction }\7
\WX{\M{41}}Compute PA\X \X\5
\WC{ Use the Gaussian-product theorem to find $\vec{P}$ }\7
\WX{\M{30}}Overlap Components\X \X\7
$\>{ovltot}=\>{ovltot}+\>{anorm}\ast\>{bnorm}\ast\>{ovl}$\5
\WC{ accumulate Overlap }\7
\WX{\M{32}}Kinetic Energy Components\X \X\7
$\>{kintot}=\>{kintot}+\>{anorm}\ast\>{bnorm}\ast\>{kin}$\5
\WC{  accumulate  Kinetic energy  }\7
\WC{  now the nuclear attraction integral   }\6
$\>{tnai}=\>{zero}$\7
\WX{\M{33}}Form fj\X \X\5
\WC{ Generate the required $f_j$ coefficients }\7
\&{do} $\|n=\WO{1},\ \39\>{noc}$\1\5
\WC{ loop over nuclei }\7
$\>{pn}=\>{zero}$\5
\WC{ Initialise current contribution  }\7
\WC{ Get the attracting-nucleus information;  co-ordinates }\7
\WX{\M{36}}Nuclear data\X \X\7
$\|t=\>{t1}\ast\>{pcsq}$\7
\&{call} $\>{auxg}(\|m,\ \39\|t,\ \39\|g)$\5
\WC{ Generate all the $F_\nu$ required }\7
\WX{\M{34}}Form As\X \X\5
\WC{ Generate the geometrical $A$-factors }\7
\WC{ Now sum the products of the geometrical $A$-factors and the $F_\nu$ }\7
\&{do} $\>{ii}=\WO{1},\ \39\>{imax}$\1\6
\&{do} $\>{jj}=\WO{1},\ \39\>{jmax}$\1\6
\&{do} $\>{kk}=\WO{1},\ \39\>{kmax}$\1\6
$\>{nu}=\>{ii}+\>{jj}+\>{kk}-\WO{2}$\6
$\>{pn}=\>{pn}+\>{Airu}(\>{ii})\ast\>{Ajsv}(\>{jj})\ast\>{Aktw}(\>{kk})\ast\|g(%
\>{nu})$\2\6
\&{end} \&{do}\2\6
\&{end} \&{do}\2\6
\&{end} \&{do}\7
$\>{tnai}=\>{tnai}-\>{pn}\ast\>{vn},\ \39\WO{4}$ $)$  \5
\WC{ Add to total multiplied by currentrrent charge }\2\7
\&{end} \&{do}\5
\WC{  end of loop over nuclei  }\6
$\>{totnai}=\>{totnai}+\>{prefa}\ast\>{tnai}$\2\6
\&{end} \&{do}\5
\WC{  end of "j" contraction  }\2\6
\&{end} \&{do}\5
\WC{ end of "i" contraction  }\7
$\>{genoei}=\>{totnai}+\>{kintot}$\5
\WC{ "T + V"  }\6
\&{return} \&{end}\WY\Wendc
\fi % End of section 27 (sect. 4.1, p. 16)

\WM28. These are the declarations which are local to \WCD{ \>{genoei}},
working space {\em etc.}

\WY\WP\4\4\WX{\M{28}}genoei local declarations\X \X${}\WSQ{}$\6
\&{double} \&{precision}~\1\>{Airu}$(\WO{10}),$ \>{Ajsv}$(\WO{10}),$ \>{Aktw}$(%
\WO{10})$\2\6
\&{double} \&{precision}~\1\|p$(\WO{3}),$ \>{sf}$(\WO{10},\ \39\WO{3}),$ %
\>{tf}$(\WO{20})$\2\6
\&{double} \&{precision}~\1\>{fact}$(\WO{20}),$ \|g$(\WO{50})$\2\6
\&{double} \&{precision}~\1\>{kin}\2\6
\&{data} ~\1\>{zero}$,$ \>{one}$,$ \>{two}$,$ \>{half}$,$ \>{quart}${/}\WO{0.0%
\^D00},\ \39\WO{1.0\^D00},\ \39\WO{2.0\^D00},\ \39\WO{0.5\^D00},\ \39\WO{0.25%
\^D00}{/}$\2\6
\&{data} ~\1\>{pi}${/}\WO{3.141592653589\^D00}{/}$\2\WY\Wendc
\WU section~\M{27}.
\fi % End of section 28 (sect. 4.1.0.1, p. 17)

\WM29. Get the various powers of $x$, $y$ and $z$ required from the data
structures and obtain the contraction limits etc.

\WY\WP\4\4\WX{\M{29}}One-electron Integer Setup\X \X${}\WSQ{}$\6
$\>{ityp}=\>{ntype}(\|i);$\6
$\>{jtyp}=\>{ntype}(\|j)$\6
$\>{l1}=\>{nr}(\>{ityp},\ \39\WO{1});$\6
$\>{m1}=\>{nr}(\>{ityp},\ \39\WO{2});$\6
$\>{n1}=\>{nr}(\>{ityp},\ \39\WO{3})$\6
$\>{l2}=\>{nr}(\>{jtyp},\ \39\WO{1});$\6
$\>{m2}=\>{nr}(\>{jtyp},\ \39\WO{2});$\6
$\>{n2}=\>{nr}(\>{jtyp},\ \39\WO{3})$\6
$\>{imax}=\>{l1}+\>{l2}+\WO{1};$\6
$\>{jmax}=\>{m1}+\>{m2}+\WO{1};$\6
$\>{kmax}=\>{n1}+\>{n2}+\WO{1}$\6
$\>{maxall}=\>{imax}$\6
$\&{if}\,(\>{maxall}<\>{jmax})$\1\6
$\>{maxall}=\>{jmax}$\2\6
$\&{if}\,(\>{maxall}<\>{kmax})$\1\6
$\>{maxall}=\>{kmax}$\2\6
$\&{if}\,(\>{maxall}<\WO{2})$\1\6
$\>{maxall}=\WO{2}$\2\5
\WC{ when all functions are "s" type }\6
$\>{iss}=\>{nfirst}(\|i);$\6
$\>{il}=\>{nlast}(\|i)$\6
$\>{jss}=\>{nfirst}(\|j);$\6
$\>{jl}=\>{nlast}(\|j)$\WY\Wendc
\WU section~\M{27}.
\fi % End of section 29 (sect. 4.1.0.2, p. 18)

\WM30. This simple code gets the Cartesian overlap components and
assembles the total integral. It also computes the overlaps required
to calculate the kinetic energy integral used in a later module.

\WY\WP\4\4\WX{\M{30}}Overlap Components\X \X${}\WSQ{}$\6
$\>{prefa}=\>{two}\ast\>{prefa}$\6
$\>{expab}=\@{dexp}({-}\>{aexp}\ast\>{bexp}\ast\>{rAB}\WSl\>{t1})$\6
$\>{s00}=(\>{pi}\WSl\>{t1})\WEE{\WO{1.5}}\ast\>{expab}$\6
$\>{dum}=\>{one};$\6
$\>{tf}(\WO{1})=\>{one};$\6
$\>{del}=\>{half}\WSl\>{t1}$\6
\&{do} $\|n=\WO{2},\ \39\>{maxall}$\1\6
$\>{tf}(\|n)=\>{tf}(\|n-\WO{1})\ast\>{dum}\ast\>{del}$\6
$\>{dum}=\>{dum}+\>{two}$\2\6
\&{end} \&{do}\7
$\>{ox0}=\>{ovrlap}(\>{l1},\ \39\>{l2},\ \39\>{pax},\ \39\>{pbx},\ \39\>{tf})$\6
$\>{oy0}=\>{ovrlap}(\>{m1},\ \39\>{m2},\ \39\>{pay},\ \39\>{pby},\ \39\>{tf})$\6
$\>{oz0}=\>{ovrlap}(\>{n1},\ \39\>{n2},\ \39\>{paz},\ \39\>{pbz},\ \39\>{tf})$\6
$\>{ox2}=\>{ovrlap}(\>{l1},\ \39\>{l2}+\WO{2},\ \39\>{pax},\ \39\>{pbx},\ \39%
\>{tf})$\6
$\>{oxm2}=\>{ovrlap}(\>{l1},\ \39\>{l2}-\WO{2},\ \39\>{pax},\ \39\>{pbx},\ \39%
\>{tf})$\6
$\>{oy2}=\>{ovrlap}(\>{m1},\ \39\>{m2}+\WO{2},\ \39\>{pay},\ \39\>{pby},\ \39%
\>{tf})$\6
$\>{oym2}=\>{ovrlap}(\>{m1},\ \39\>{m2}-\WO{2},\ \39\>{pay},\ \39\>{pby},\ \39%
\>{tf})$\6
$\>{oz2}=\>{ovrlap}(\>{n1},\ \39\>{n2}+\WO{2},\ \39\>{paz},\ \39\>{pbz},\ \39%
\>{tf})$\6
$\>{ozm2}=\>{ovrlap}(\>{n1},\ \39\>{n2}-\WO{2},\ \39\>{paz},\ \39\>{pbz},\ \39%
\>{tf})$\6
$\>{ov0}=\>{ox0}\ast\>{oy0}\ast\>{oz0};$\6
$\>{ovl}=\>{ov0}\ast\>{s00}$\6
$\>{ov1}=\>{ox2}\ast\>{oy0}\ast\>{oz0};$\6
$\>{ov4}=\>{oxm2}\ast\>{oy0}\ast\>{oz0}$\6
$\>{ov2}=\>{ox0}\ast\>{oy2}\ast\>{oz0};$\6
$\>{ov5}=\>{ox0}\ast\>{oym2}\ast\>{oz0}$\6
$\>{ov3}=\>{ox0}\ast\>{oy0}\ast\>{oz2};$\6
$\>{ov6}=\>{ox0}\ast\>{oy0}\ast\>{ozm2}$\WY\Wendc
\WU section~\M{27}.
\fi % End of section 30 (sect. 4.1.0.3, p. 19)

\WN31. 2 ovrlap.
One-dimensional Cartesian overlap. This function uses the
precomputed factors in \WCD{ \>{tf}} to evaluate the simple Cartesian
components
of the overlap integral which must be multiplied together to
form the total overlap integral.

\WY\WP \Wunnamed{code}{main.f}%
 \&{double} \&{precision} \&{function} \1$%
\>{ovrlap}(\>{l1},\ \39\>{l2},\ \39\>{pax},\ \39\>{pbx},\ \39\>{tf})$\2\1\6
\&{implicit} \1\&{double} \&{precision}$\,(\|a-\|h,\ \39\|o-\|z)$\2\6
\&{integer}~\1\>{l1}$,$ \>{l2}\2\6
\&{double} \&{precision}~\1\>{pax}$,$ \>{pbx}\2\6
\&{double} \&{precision}~\1\>{tf}$(\ast)$\2\5
\WC{ pre-computed exponent and double factorial
    factors: tf(i+1) = (2i-1)�!/(2**i*(A+B)**i) }\7
\&{double} \&{precision}~\1\>{zero}$,$ \>{one}$,$ \>{dum}\2\6
\&{data} ~\1\>{zero}$,$ \>{one}${/}\WO{0.0\^D00},\ \39\WO{1.0\^D00}{/}$\2\7
$\&{if}\,((\>{l1}<\WO{0})\WOR(\>{l2}<\WO{0}))$ \&{then}\1\6
$\>{ovrlap}=\>{zero}$\6
\&{return}\2\6
\&{end} \&{if}\7
$\&{if}\,((\>{l1}\WS\WO{0})\WW(\>{l2}\WS\WO{0}))$ \&{then}\1\6
$\>{ovrlap}=\>{one}$\6
\&{return}\2\6
\&{end} \&{if}\7
$\>{dum}=\>{zero};$\6
$\>{maxkk}=(\>{l1}+\>{l2})\WSl\WO{2}+\WO{1}$\7
\&{do} $\>{kk}=\WO{1},\ \39\>{maxkk}$\1\6
$\>{dum}=\>{dum}+\>{tf}(\>{kk})\ast\>{fj}(\>{l1},\ \39\>{l2},\ \39\WO{2}\ast%
\>{kk}-\WO{2},\ \39\>{pax},\ \39\>{pbx})$\2\6
\&{end} \&{do}\7
$\>{ovrlap}=\>{dum}$\7
\&{return}\2\6
\&{end}\WY\Wendc
\fi % End of section 31 (sect. 4.1.1, p. 20a)

\WM32. Use the previously-computed overlap components to
generate the Kinetic energy components and
hence the total integral.

\WY\WP\4\4\WX{\M{32}}Kinetic Energy Components\X \X${}\WSQ{}$\6
$\>{xl}=\@{dfloat}(\>{l2}\ast(\>{l2}-\WO{1}));$\6
$\>{xm}=\@{dfloat}(\>{m2}\ast(\>{m2}-\WO{1}))$\6
$\>{xn}=\@{dfloat}(\>{n2}\ast(\>{n2}-\WO{1}));$\6
$\>{xj}=\@{dfloat}(\WO{2}\ast(\>{l2}+\>{m2}+\>{n2})+\WO{3})$\6
$\>{kin}=\>{s00}\ast(\>{bexp}\ast(\>{xj}\ast\>{ov0}-\>{two}\ast\>{bexp}\ast(%
\>{ov1}+\>{ov2}+\>{ov3}))-\>{half}\ast(\>{xl}\ast\>{ov4}+\>{xm}\ast\>{ov5}+%
\>{xn}\ast\>{ov6}))$\WY\Wendc
\WU section~\M{27}.
\fi % End of section 32 (sect. 4.1.1.1, p. 20b)

\WM33. Form the $f_j$ coefficients needed for the nuclear attraction integral.

\WY\WP\4\4\WX{\M{33}}Form fj\X \X${}\WSQ{}$\6
$\|m=\>{imax}+\>{jmax}+\>{kmax}-\WO{2}$\6
\&{do} $\|n=\WO{1},\ \39\>{imax}$\1\6
$\>{sf}(\|n,\ \39\WO{1})=\>{fj}(\>{l1},\ \39\>{l2},\ \39\|n-\WO{1},\ \39%
\>{pax},\ \39\>{pbx})$\2\6
\&{end} \&{do}\7
\&{do} $\|n=\WO{1},\ \39\>{jmax}$\1\6
$\>{sf}(\|n,\ \39\WO{2})=\>{fj}(\>{m1},\ \39\>{m2},\ \39\|n-\WO{1},\ \39%
\>{pay},\ \39\>{pby})$\2\6
\&{end} \&{do}\7
\&{do} $\|n=\WO{1},\ \39\>{kmax}$\1\6
$\>{sf}(\|n,\ \39\WO{3})=\>{fj}(\>{n1},\ \39\>{n2},\ \39\|n-\WO{1},\ \39%
\>{paz},\ \39\>{pbz})$\2\6
\&{end} \&{do}\WY\Wendc
\WU section~\M{27}.
\fi % End of section 33 (sect. 4.1.1.2, p. 21a)

\WM34. Use \WCD{ \>{aform}} to compute the required $A$-factors for each
Cartesian component.

\WY\WP\4\4\WX{\M{34}}Form As\X \X${}\WSQ{}$\6
$\>{epsi}=\>{quart}\WSl\>{t1}$\6
\&{do} $\>{ii}=\WO{1},\ \39\WO{10}$\1\6
$\>{Airu}(\>{ii})=\>{zero}$\6
$\>{Ajsv}(\>{ii})=\>{zero}$\6
$\>{Aktw}(\>{ii})=\>{zero}$\2\6
\&{end} \&{do}\7
\&{call} $\>{aform}(\>{imax},\ \39\>{sf},\ \39\>{fact},\ \39\>{cpx},\ \39%
\>{epsi},\ \39\>{Airu},\ \39\WO{1})$\5
\WC{ form $A_{i,r,u}$  }\6
\&{call} $\>{aform}(\>{jmax},\ \39\>{sf},\ \39\>{fact},\ \39\>{cpy},\ \39%
\>{epsi},\ \39\>{Ajsv},\ \39\WO{2})$\5
\WC{ form $A_{j,s,v}$  }\6
\&{call} $\>{aform}(\>{kmax},\ \39\>{sf},\ \39\>{fact},\ \39\>{cpz},\ \39%
\>{epsi},\ \39\>{Aktw},\ \39\WO{3})$\5
\WC{ form $A_{k,t,w}$  }\WY\Wendc
\WU section~\M{27}.
\fi % End of section 34 (sect. 4.1.1.3, p. 21b)

\WN35. 2 aform. Compute the nuclear-attraction $A$ factors. These quantitities
arise from the components of the three position vectors of the two
basis functions and the attracting centre with respect to the
centre of the product Gaussian. There is one
of these for each of the three dimensions of Cartesian space; a typical
one (the $x$ component) is:
$$
A_{\ell,r,i} ( \ell_1 , \ell_2 , \vec{A}_x , \vec{B}_x , \vec{C}_x ,\gamma )
= (-1)^{\ell} f_{\ell} ( \ell_1, \ell_2 , \vec{PA}_x , \vec{PB}_x )
  {{ (-1)^i \ell ! \vec{PC}_x^{\ell-2r-2i} \epsilon^{r+i}} \over
       {r! i! (\ell -2r-2i)!}}
$$


\WY\WP \Wunnamed{code}{main.f}%
\&{subroutine} \1$\>{aform}(\>{imax},\ \39%
\>{sf},\ \39\>{fact},\ \39\>{cpx},\ \39\>{epsi},\ \39\>{Airu},\ \39\>{xyorz})$%
\2\1\6
\&{implicit} \1\&{double} \&{precision}$\,(\|a-\|h,\ \39\|o-\|z)$\2\6
\&{integer}~\1\>{imax}$,$ \>{xyorz}\2\6
\&{double} \&{precision}~\1\>{Airu}$(\ast),$ \>{fact}$(\ast),$ \>{sf}$(\WO{10},%
\ \39\ast)$\2\7
\&{double} \&{precision}~\1\>{one}\2\6
\&{data} ~\1\>{one}${/}\WO{1.0\^D00}{/}$\2\7
\&{do} $\|i=\WO{1},\ \39\>{imax}$\1\6
$\>{ai}=({-}\>{one})\WEE{(\|i-\WO{1})}\ast\>{sf}(\|i,\ \39\>{xyorz})\ast%
\>{fact}(\|i)$\6
$\>{irmax}=(\|i-\WO{1})\WSl\WO{2}+\WO{1}$\6
\&{do} $\>{ir}=\WO{1},\ \39\>{irmax}$\1\6
$\>{irumax}=\>{irmax}-\>{ir}+\WO{1}$\6
\&{do} $\>{iru}=\WO{1},\ \39\>{irumax}$\1\6
$\>{iq}=\>{ir}+\>{iru}-\WO{2}$\6
$\>{ip}=\|i-\WO{2}\ast\>{iq}-\WO{1}$\6
$\>{at5}=\>{one}$\6
$\&{if}\,(\>{ip}>\WO{0})$\1\6
$\>{at5}=\>{cpx}\WEE{\>{ip}}$\2\6
$\>{tiru}=\>{ai}\ast({-}\>{one})\WEE{(\>{iru}-\WO{1})}\ast\>{at5}\ast\>{epsi}%
\WEE{\>{iq}}\WSl(\>{fact}(\>{ir})\ast\>{fact}(\>{iru})\ast\>{fact}(\>{ip}+%
\WO{1}))$\6
$\>{nux}=\>{ip}+\>{iru}$\6
$\>{Airu}(\>{nux})=\>{Airu}(\>{nux})+\>{tiru}$\2\6
\&{end} \&{do}\2\6
\&{end} \&{do}\2\6
\&{end} \&{do}\7
\&{return}\2\6
\&{end}\WY\Wendc
\fi % End of section 35 (sect. 4.1.2, p. 22a)

\WM36. Get the co-ordinates of the attracting nucleus with respect to $%
\vec{P}$.

\WY\WP\4\4\WX{\M{36}}Nuclear data\X \X${}\WSQ{}$\6
$\>{cpx}=\|p(\WO{1})-\>{vlist}(\|n,\ \39\WO{1})$\6
$\>{cpy}=\|p(\WO{2})-\>{vlist}(\|n,\ \39\WO{2})$\6
$\>{cpz}=\|p(\WO{3})-\>{vlist}(\|n,\ \39\WO{3})$\6
$\>{pcsq}=\>{cpx}\ast\>{cpx}+\>{cpy}\ast\>{cpy}+\>{cpz}\ast\>{cpz}$\WY\Wendc
\WU section~\M{27}.
\fi % End of section 36 (sect. 4.1.2.1, p. 22b)

\WN37. 1 generi.
The general electron-repulsion integral formula for contracted
Gaussian basis functions. The STRUCTURES and GENERI manual pages must be
consulted for a detailed description of the calling sequence.

\WY\WP \Wunnamed{code}{main.f}%
 \&{double} \&{precision} \&{function} \1$%
\>{generi}(\|i,\ \39\|j,\ \39\|k,\ \39\|l,\ \39\>{xyorz},\ \39\>{eta},\ \39%
\>{ngmx},\ \39\>{nfirst},\ \39\>{nlast},\ \39\>{ntype},\ \39\>{nr},\ \39%
\>{ntmx})$\2\1\7
\&{implicit} \1\&{double} \&{precision}$\,(\|a-\|h,\ \39\|o-\|z)$\2\6
\&{integer}~\1\|i$,$ \|j$,$ \|k$,$ \|l$,$ \>{xyorz}$,$ \>{ngmx}$,$ \>{ntmx}\2\6
\&{double} \&{precision}~\1\>{eta}$(\WUC{MAX\_PRIMITIVES},\ \39\WO{5})$\2\6
\&{integer}~\1\>{nfirst}$(\ast),$ \>{nlast}$(\ast),$ \>{ntype}$(\ast),$ %
\>{nr}$(\>{ntmx},\ \39\WO{3})$\2\7
\WC{  Variables local to the function }\7
\WX{\M{38}}generi local declarations\X \X\7
\WC{ Insert the \WCD{  \&{data}}  statement for the factorials }\7
\WX{\M{39}}Factorials\X \X\7
\WC{ Get the various integers from the data structures for           the
summation limits, Cartesian monomial powers etc. from           the main
integer data structures  }\7
\WX{\M{40}}Two-electron Integer Setup\X \X\7
\WC{ Two internuclear distances this time  }\7
$\>{rAB}=(\>{eta}(\>{is},\ \39\WO{1})-\>{eta}(\>{js},\ \39\WO{1}))\WEE{%
\WO{2}}+(\>{eta}(\>{is},\ \39\WO{2})-\>{eta}(\>{js},\ \39\WO{2}))\WEE{\WO{2}}+(%
\>{eta}(\>{is},\ \39\WO{3})-\>{eta}(\>{js},\ \39\WO{3}))\WEE{\WO{2}}$\6
$\>{rCD}=(\>{eta}(\>{ks},\ \39\WO{1})-\>{eta}(\>{ls},\ \39\WO{1}))\WEE{%
\WO{2}}+(\>{eta}(\>{ks},\ \39\WO{2})-\>{eta}(\>{ls},\ \39\WO{2}))\WEE{\WO{2}}+(%
\>{eta}(\>{ks},\ \39\WO{3})-\>{eta}(\>{ls},\ \39\WO{3}))\WEE{\WO{2}}$\7
\WC{ Initialise the accumulator }\7
$\>{generi}=\>{zero}$\7
\WC{ Now the real work, begin the four contraction loops }\7
\&{do} $\>{irun}=\>{is},\ \39\>{il}$\1\5
\WC{ start of "i" contraction}\7
\&{do} $\>{jrun}=\>{js},\ \39\>{jl}$\1\5
\WC{ start of "j" contraction}\7
\WC{ Get the data for the two basis functions referring to
electron 1; orbital exponents and Cartesian co-ordinates              and hence
compute the vector $\vec{P}$ and the components of              $\vec{PA}$ and
$\vec{PB}$   }\7
\WX{\M{41}}Compute PA\X \X\7
\WC{ Use \WCD{ \&{function} \>{fj}} and \WCD{ \&{subroutine} \>{theta}} to
calculate the            geometric factors arising from the expansion of the
product of            Cartesian monomials for the basis functions of electron 1
 }\7
\WX{\M{43}}Thetas for electron 1\X \X\7
\&{do} $\>{krun}=\>{ks},\ \39\>{kl}$\1\5
\WC{ start of "k" contraction}\6
\&{do} $\>{lrun}=\>{ls},\ \39\>{ll}$\1\5
\WC{ start of "l" contraction}\6
$\>{eribit}=\>{zero}$\5
\WC{ local accumulator }\7
\WC{ Get the data for the two basis functions referring to             electron
2; orbital exponents and Cartesian co-ordinates             and hence compute
the vector $\vec{Q}$ and the components of             $\vec{QC}$   and $%
\vec{QD}$   }\7
\WX{\M{42}}Compute QC\X \X\7
$\|w=\>{pi}\WSl(\>{t1}+\>{t2})$\7
\WC{ Repeat the use of \WCD{ \&{function} \>{fj}}             to obtain the
geometric factors arising from the expansion             of Cartesian monomials
for the basis functions of electron 2  }\7
\WX{\M{44}}fj for electron 2\X \X\7
\&{call} $\>{auxg}(\|m,\ \39\|t,\ \39\|g)$\5
\WC{ Obtain the $F_\nu$ by recursion }\7
\WC{ Now use the pre-computed $\theta$ factors for both electron
distributions to form the overall $B$ factors }\7
\WX{\M{45}}Form Bs\X \X\7
\WC{ Form the limits and add up all the bits, the products of              %
\WCD{ \|x}, \WCD{ \|y} and \WCD{ \|z} related B factors and the $F_{\nu}$ }\7
$\>{jt1}=\>{i1max}+\>{i2max}-\WO{1}$\6
$\>{jt2}=\>{j1max}+\>{j2max}-\WO{1}$\6
$\>{jt3}=\>{k1max}+\>{k2max}-\WO{1}$\7
\&{do} $\>{ii}=\WO{1},\ \39\>{jt1}$\1\6
\&{do} $\>{jj}=\WO{1},\ \39\>{jt2}$\1\6
\&{do} $\>{kk}=\WO{1},\ \39\>{jt3}$\1\6
$\>{nu}=\>{ii}+\>{jj}+\>{kk}-\WO{2}$\6
$\&{if}\,(\>{xyorz}\WI\WO{0})$\1\6
$\>{nu}=\>{nu}+\WO{1}$\2\7
\WC{ \WCD{ \>{eribit}} is a  repulsion integral over primitive GTFs }\7
$\>{eribit}=\>{eribit}+\|g(\>{nu})\ast\>{bbx}(\>{ii})\ast\>{bby}(\>{jj})\ast%
\>{bbz}(\>{kk})$\2\7
\&{end} \&{do}\2\6
\&{end} \&{do}\2\6
\&{end} \&{do}\7
\WC{ Now accumulate the primitive integrals into the integral
over contracted GTFs including some constant factors               and
contraction coefficients    }\7
$\>{generi}=\>{generi}+\>{prefa}\ast\>{prefc}\ast\>{eribit}\ast\@{dsqrt}(\|w)$%
\2\7
\&{end} \&{do}\5
\WC{ end of "l" contraction loop }\2\6
\&{end} \&{do}\5
\WC{ end of "k" contraction loop }\2\6
\&{end} \&{do}\5
\WC{ end of "j" contraction loop }\2\6
\&{end} \&{do}\5
\WC{ end of "i" contraction loop }\7
$\&{if}\,(\>{xyorz}\WS\WO{0})$\1\6
$\>{generi}=\>{generi}\ast\>{two}$\2\6
\&{return}\2\6
\&{end}\WY\Wendc
\fi % End of section 37 (sect. 4.2, p. 23)

\WM38. Here are the local declarations (workspoace {\em etc.})
for the two-electron main function \WCD{ \>{generi}}.

\WY\WP\4\4\WX{\M{38}}generi local declarations\X \X${}\WSQ{}$\6
\&{double} \&{precision}~\1\|p$(\WO{3}),$ \|q$(\WO{3}),$ \>{ppx}$(\WO{20}),$ %
\>{ppy}$(\WO{20}),$ \>{ppz}$(\WO{20})$\2\6
\&{double} \&{precision}~\1\>{bbx}$(\WO{20}),$ \>{bby}$(\WO{20}),$ \>{bbz}$(%
\WO{20}),$ \>{sf}$(\WO{10},\ \39\WO{6})$\2\6
\&{double} \&{precision}~\1\>{xleft}$(\WO{5},\ \39\WO{10}),$ \>{yleft}$(\WO{5},%
\ \39\WO{10}),$ \>{zleft}$(\WO{5},\ \39\WO{10})$\2\6
\&{double} \&{precision}~\1\|r$(\WO{3}),$ \>{fact}$(\WO{20}),$ \|g$(\WO{50})$\2%
\6
\&{data} ~\1\>{zero}$,$ \>{one}$,$ \>{two}$,$ \>{half}${/}\WO{0.0\^D00},\ \39%
\WO{1.0\^D00},\ \39\WO{2.0\^D00},\ \39\WO{0.5\^D00}{/}$\2\6
\&{data} ~\1\>{pi}${/}\WO{3.141592653589\^D00}{/}$\2\WY\Wendc
\WU section~\M{37}.
\fi % End of section 38 (sect. 4.2.0.1, p. 25a)

\WM39. These numbers are the first 20 factorials \WCD{ $\>{fact}(\|i)$}
contains $(i-1)!$.

\WY\WP\4\4\WX{\M{39}}Factorials\X \X${}\WSQ{}$\6
\&{data} ~\1\>{fact}${/}\WO{1.0\^D00},\ \39\WO{1.0\^D00},\ \39\WO{2.0\^D00},\ %
\39\WO{6.0\^D00},\ \39\WO{24.0\^D00},\ \39\WO{120.0\^D00},\ \39\WO{720.0\^D00},%
\ \39\WO{5040.0\^D00},\ \39\WO{40320.0\^D00},\ \39\WO{362880.0\^D00},\ \39%
\WO{3628800.0\^D00},\ \39\WO{39916800.0\^D00},\ \39\WO{479001600.0\^D00},\ \39%
\WO{6227020800.0\^D00},\ \39\WO{6}\ast\WO{0.0\^D00}{/}$\2\WY\Wendc
\WU sections~\M{27}, \M{37}, and~\M{47}.
\fi % End of section 39 (sect. 4.2.0.2, p. 25b)

\WM40. This tedious code extracts the (integer) setup data; the powers of
$x$, $y$ and $z$ in each of the Cartesian monomials of
each of the four basis functions and the limits of the contraction
in each case.

\WY\WP\4\4\WX{\M{40}}Two-electron Integer Setup\X \X${}\WSQ{}$\6
$\>{ityp}=\>{ntype}(\|i)$\6
$\>{jtyp}=\>{ntype}(\|j)$\6
$\>{ktyp}=\>{ntype}(\|k)$\6
$\>{ltyp}=\>{ntype}(\|l)$\6
$\>{l1}=\>{nr}(\>{ityp},\ \39\WO{1})$\6
$\>{m1}=\>{nr}(\>{ityp},\ \39\WO{2})$\6
$\>{n1}=\>{nr}(\>{ityp},\ \39\WO{3})$\6
$\>{l2}=\>{nr}(\>{jtyp},\ \39\WO{1})$\6
$\>{m2}=\>{nr}(\>{jtyp},\ \39\WO{2})$\6
$\>{n2}=\>{nr}(\>{jtyp},\ \39\WO{3})$\6
$\>{l3}=\>{nr}(\>{ktyp},\ \39\WO{1})$\6
$\>{m3}=\>{nr}(\>{ktyp},\ \39\WO{2})$\6
$\>{n3}=\>{nr}(\>{ktyp},\ \39\WO{3})$\6
$\>{l4}=\>{nr}(\>{ltyp},\ \39\WO{1})$\6
$\>{m4}=\>{nr}(\>{ltyp},\ \39\WO{2})$\6
$\>{n4}=\>{nr}(\>{ltyp},\ \39\WO{3})$\6
$\>{is}=\>{nfirst}(\|i)$\6
$\>{il}=\>{nlast}(\|i)$\6
$\>{js}=\>{nfirst}(\|j)$\6
$\>{jl}=\>{nlast}(\|j)$\6
$\>{ks}=\>{nfirst}(\|k)$\6
$\>{kl}=\>{nlast}(\|k)$\6
$\>{ls}=\>{nfirst}(\|l)$\6
$\>{ll}=\>{nlast}(\|l)$\WY\Wendc
\WU section~\M{37}.
\fi % End of section 40 (sect. 4.2.0.3, p. 25c)

\WM41. Use the Gaussian Product Theorem to find the position vector
$\vec{P}$, of the product of the two Gaussian exponential factors
of the basis functions for electron 1.

\WY\WP\4\4\WX{\M{41}}Compute PA\X \X${}\WSQ{}$\6
$\>{aexp}=\>{eta}(\>{irun},\ \39\WO{4});$\6
$\>{anorm}=\>{eta}(\>{irun},\ \39\WO{5})$\6
$\>{bexp}=\>{eta}(\>{jrun},\ \39\WO{4});$\6
$\>{bnorm}=\>{eta}(\>{jrun},\ \39\WO{5})$\7
\WC{ \WCD{ \>{aexp}} and \WCD{ \>{bexp}} are the primitive GTF exponents for
    GTF \WCD{ \>{irun}} and \WCD{ \>{jrun}}, \WCD{ \>{anorm}} and \WCD{ %
\>{bnorm}} are the        corresponding contraction coefficients bundled up
into        \WCD{ \>{prefa}}  }\7
$\>{t1}=\>{aexp}+\>{bexp};$\6
$\>{deleft}=\>{one}\WSl\>{t1}$\7
$\|p(\WO{1})=(\>{aexp}\ast\>{eta}(\>{irun},\ \39\WO{1})+\>{bexp}\ast\>{eta}(%
\>{jrun},\ \39\WO{1}))\ast\>{deleft}$\6
$\|p(\WO{2})=(\>{aexp}\ast\>{eta}(\>{irun},\ \39\WO{2})+\>{bexp}\ast\>{eta}(%
\>{jrun},\ \39\WO{2}))\ast\>{deleft}$\6
$\|p(\WO{3})=(\>{aexp}\ast\>{eta}(\>{irun},\ \39\WO{3})+\>{bexp}\ast\>{eta}(%
\>{jrun},\ \39\WO{3}))\ast\>{deleft}$\7
$\>{pax}=\|p(\WO{1})-\>{eta}(\>{irun},\ \39\WO{1})$\6
$\>{pay}=\|p(\WO{2})-\>{eta}(\>{irun},\ \39\WO{2})$\6
$\>{paz}=\|p(\WO{3})-\>{eta}(\>{irun},\ \39\WO{3})$\7
$\>{pbx}=\|p(\WO{1})-\>{eta}(\>{jrun},\ \39\WO{1})$\6
$\>{pby}=\|p(\WO{2})-\>{eta}(\>{jrun},\ \39\WO{2})$\6
$\>{pbz}=\|p(\WO{3})-\>{eta}(\>{jrun},\ \39\WO{3})$\7
$\>{prefa}=\@{dexp}({-}\>{aexp}\ast\>{bexp}\ast\>{rAB}\WSl\>{t1})\ast\>{pi}\ast%
\>{anorm}\ast\>{bnorm}\WSl\>{t1}$\WY\Wendc
\WU sections~\M{27} and~\M{37}.
\fi % End of section 41 (sect. 4.2.0.4, p. 26)

\WM42. Use the Gaussian Product Theorem to find the position vector
$\vec{Q}$, of the product of the two Gaussian exponential factors
of the basis functions for electron 2.

\WY\WP\4\4\WX{\M{42}}Compute QC\X \X${}\WSQ{}$\6
$\>{cexpp}=\>{eta}(\>{krun},\ \39\WO{4});$\6
$\>{cnorm}=\>{eta}(\>{krun},\ \39\WO{5})$\6
$\>{dexpp}=\>{eta}(\>{lrun},\ \39\WO{4});$\6
$\>{dnorm}=\>{eta}(\>{lrun},\ \39\WO{5})$\7
\WC{ \WCD{ \@{cexp}} and \WCD{ \@{dexp}} are the primitive GTF exponents for
    GTF \WCD{ \>{krun}} and \WCD{ \>{lrun}}, \WCD{ \>{cnorm}} and \WCD{ %
\>{dnorm}} are the        corresponding contraction coefficients bundled up
into        \WCD{ \>{prefc}}  }\7
$\>{t2}=\>{cexpp}+\>{dexpp}$\6
$\>{t2m1}=\>{one}\WSl\>{t2}$\6
$\>{fordel}=\>{t2m1}+\>{deleft}$\7
$\|q(\WO{1})=(\>{cexpp}\ast\>{eta}(\>{krun},\ \39\WO{1})+\>{dexpp}\ast\>{eta}(%
\>{lrun},\ \39\WO{1}))\ast\>{t2m1}$\6
$\|q(\WO{2})=(\>{cexpp}\ast\>{eta}(\>{krun},\ \39\WO{2})+\>{dexpp}\ast\>{eta}(%
\>{lrun},\ \39\WO{2}))\ast\>{t2m1}$\6
$\|q(\WO{3})=(\>{cexpp}\ast\>{eta}(\>{krun},\ \39\WO{3})+\>{dexpp}\ast\>{eta}(%
\>{lrun},\ \39\WO{3}))\ast\>{t2m1}$\7
$\>{qcx}=\|q(\WO{1})-\>{eta}(\>{krun},\ \39\WO{1})$\6
$\>{qcy}=\|q(\WO{2})-\>{eta}(\>{krun},\ \39\WO{2})$\6
$\>{qcz}=\|q(\WO{3})-\>{eta}(\>{krun},\ \39\WO{3})$\7
$\>{qdx}=\|q(\WO{1})-\>{eta}(\>{lrun},\ \39\WO{1})$\6
$\>{qdy}=\|q(\WO{2})-\>{eta}(\>{lrun},\ \39\WO{2})$\6
$\>{qdz}=\|q(\WO{3})-\>{eta}(\>{lrun},\ \39\WO{3})$\7
$\|r(\WO{1})=\|p(\WO{1})-\|q(\WO{1})$\6
$\|r(\WO{2})=\|p(\WO{2})-\|q(\WO{2})$\6
$\|r(\WO{3})=\|p(\WO{3})-\|q(\WO{3})$\7
$\|t=(\|r(\WO{1})\ast\|r(\WO{1})+\|r(\WO{2})\ast\|r(\WO{2})+\|r(\WO{3})\ast\|r(%
\WO{3}))\WSl\>{fordel}$\6
$\>{prefc}=\@{exp}({-}\>{cexpp}\ast\>{dexpp}\ast\>{rCD}\WSl\>{t2})\ast\>{pi}%
\ast\>{cnorm}\ast\>{dnorm}\WSl\>{t2}$\WY\Wendc
\WU section~\M{37}.
\fi % End of section 42 (sect. 4.2.0.5, p. 27)

\WM43. The series of terms arising from the expansion of the
Cartesian monomials like $(x - PA)^{\ell_1}(x - PB)^{\ell_2}$ are
computed by first forming the $f_j$ and hence the $\theta$s.

\WY\WP\4\4\WX{\M{43}}Thetas for electron 1\X \X${}\WSQ{}$\6
$\>{i1max}=\>{l1}+\>{l2}+\WO{1}$\6
$\>{j1max}=\>{m1}+\>{m2}+\WO{1}$\6
$\>{k1max}=\>{n1}+\>{n2}+\WO{1}$\7
$\>{mleft}=\>{i1max}+\>{j1max}+\>{k1max}$\7
\&{do} $\|n=\WO{1},\ \39\>{i1max}$\1\6
$\>{sf}(\|n,\ \39\WO{1})=\>{fj}(\>{l1},\ \39\>{l2},\ \39\|n-\WO{1},\ \39%
\>{pax},\ \39\>{pbx})$\2\6
\&{end} \&{do}\7
\&{do} $\|n=\WO{1},\ \39\>{j1max}$\1\6
$\>{sf}(\|n,\ \39\WO{2})=\>{fj}(\>{m1},\ \39\>{m2},\ \39\|n-\WO{1},\ \39%
\>{pay},\ \39\>{pby})$\2\6
\&{end} \&{do}\7
\&{do} $\|n=\WO{1},\ \39\>{k1max}$\1\6
$\>{sf}(\|n,\ \39\WO{3})=\>{fj}(\>{n1},\ \39\>{n2},\ \39\|n-\WO{1},\ \39%
\>{paz},\ \39\>{pbz})$\2\6
\&{end} \&{do}\7
\&{call} $\>{theta}(\>{i1max},\ \39\>{sf},\ \39\WO{1},\ \39\>{fact},\ \39%
\>{t1},\ \39\>{xleft})$\6
\&{call} $\>{theta}(\>{j1max},\ \39\>{sf},\ \39\WO{2},\ \39\>{fact},\ \39%
\>{t1},\ \39\>{yleft})$\6
\&{call} $\>{theta}(\>{k1max},\ \39\>{sf},\ \39\WO{3},\ \39\>{fact},\ \39%
\>{t1},\ \39\>{zleft})$\WY\Wendc
\WU section~\M{37}.
\fi % End of section 43 (sect. 4.2.0.6, p. 28a)

\WM44. The series of terms arising from the expansion of the
Cartesian monomials like $(x - QC)^{\ell_3}(x - QD)^{\ell_4}$ are
computed by  forming the $f_j$ and storing them in the array \WCD{ \>{sf}}
for later use by \WCD{ \>{bform}}.

\WY\WP\4\4\WX{\M{44}}fj for electron 2\X \X${}\WSQ{}$\6
$\>{i2max}=\>{l3}+\>{l4}+\WO{1}$\6
$\>{j2max}=\>{m3}+\>{m4}+\WO{1}$\6
$\>{k2max}=\>{n3}+\>{n4}+\WO{1}$\7
$\>{twodel}=\>{half}\ast\>{fordel}$\6
$\>{delta}=\>{half}\ast\>{twodel}$\7
\&{do} $\|n=\WO{1},\ \39\>{i2max}$\1\6
$\>{sf}(\|n,\ \39\WO{4})=\>{fj}(\>{l3},\ \39\>{l4},\ \39\|n-\WO{1},\ \39%
\>{qcx},\ \39\>{qdx})$\2\6
\&{end} \&{do}\7
\&{do} $\|n=\WO{1},\ \39\>{j2max}$\1\6
$\>{sf}(\|n,\ \39\WO{5})=\>{fj}(\>{m3},\ \39\>{m4},\ \39\|n-\WO{1},\ \39%
\>{qcy},\ \39\>{qdy})$\2\6
\&{end} \&{do}\7
\&{do} $\|n=\WO{1},\ \39\>{k2max}$\1\6
$\>{sf}(\|n,\ \39\WO{6})=\>{fj}(\>{n3},\ \39\>{n4},\ \39\|n-\WO{1},\ \39%
\>{qcz},\ \39\>{qdz})$\2\6
\&{end} \&{do}\7
$\|m=\>{mleft}+\>{i2max}+\>{j2max}+\>{k2max}+\WO{1}$\WY\Wendc
\WU section~\M{37}.
\fi % End of section 44 (sect. 4.2.0.7, p. 28b)

\WM45. In the central inner loops of the four contractions,
use the previously- computed $\theta$ factors to
form the combined geometrical $B$ factors.

\WY\WP\4\4\WX{\M{45}}Form Bs\X \X${}\WSQ{}$\6
$\>{ppx}(\WO{1})=\>{one};$\6
$\>{bbx}(\WO{1})=\>{zero}$\6
$\>{ppy}(\WO{1})=\>{one};$\6
$\>{bby}(\WO{1})=\>{zero}$\6
$\>{ppz}(\WO{1})=\>{one};$\6
$\>{bbz}(\WO{1})=\>{zero}$\7
$\>{jt1}=\>{i1max}+\>{i2max}$\6
\&{do} $\|n=\WO{2},\ \39\>{jt1}$\1\6
$\>{ppx}(\|n)={-}\>{ppx}(\|n-\WO{1})\ast\|r(\WO{1})$\6
$\>{bbx}(\|n)=\>{zero}$\2\6
\&{end} \&{do}\7
$\>{jt1}=\>{j1max}+\>{j2max}$\6
\&{do} $\|n=\WO{2},\ \39\>{jt1}$\1\6
$\>{ppy}(\|n)={-}\>{ppy}(\|n-\WO{1})\ast\|r(\WO{2})$\6
$\>{bby}(\|n)=\>{zero}$\2\6
\&{end} \&{do}\7
$\>{jt1}=\>{k1max}+\>{k2max}$\6
\&{do} $\|n=\WO{2},\ \39\>{jt1}$\1\6
$\>{ppz}(\|n)={-}\>{ppz}(\|n-\WO{1})\ast\|r(\WO{3})$\6
$\>{bbz}(\|n)=\>{zero}$\2\6
\&{end} \&{do}\7
\&{call} $\>{bform}(\>{i1max},\ \39\>{i2max},\ \39\>{sf},\ \39\WO{1},\ \39%
\>{fact},\ \39\>{xleft},\ \39\>{t2},\ \39\>{delta},\ \39\>{ppx},\ \39\>{bbx},\ %
\39\>{xyorz})$\6
\&{call} $\>{bform}(\>{j1max},\ \39\>{j2max},\ \39\>{sf},\ \39\WO{2},\ \39%
\>{fact},\ \39\>{yleft},\ \39\>{t2},\ \39\>{delta},\ \39\>{ppy},\ \39\>{bby},\ %
\39\>{xyorz})$\6
\&{call} $\>{bform}(\>{k1max},\ \39\>{k2max},\ \39\>{sf},\ \39\WO{3},\ \39%
\>{fact},\ \39\>{zleft},\ \39\>{t2},\ \39\>{delta},\ \39\>{ppz},\ \39\>{bbz},\ %
\39\>{xyorz})$\Wendc
\WU section~\M{37}.
\fi % End of section 45 (sect. 4.2.0.8, p. 29a)

\WM46.

\fi % End of section 46 (sect. 4.2.0.9, p. 29b)

\WN47. 1 fj.
This is the function to evaluate the coefficient of $x^j$ in the expansion
of
$$
(x + a)^\ell (x+b)^m
$$
The full expression is
$$
f_j (\ell , m , a, b) = \sum_{k = max (0, j-m)}^{min(j, \ell }
                         { \ell \choose k}{ m \choose {j-k}}
                          a^{\ell - k } b^{m + k - j}
$$
The function must take steps to do the right thing for
$0.0^0$ when it occurs.

\WY\WP \Wunnamed{code}{main.f}%
 \&{double} \&{precision} \&{function} \1$%
\>{fj}(\|l,\ \39\|m,\ \39\|j,\ \39\|a,\ \39\|b)$\2\1\7
\&{implicit} \1\&{double} \&{precision}$\,(\|a-\|h,\ \39\|o-\|z)$\2\6
\&{integer}~\1\|l$,$ \|m$,$ \|j\2\6
\&{double} \&{precision}~\1\|a$,$ \|b\2\7
\&{double} \&{precision}~\1\>{sum}$,$ \>{term}$,$ \>{aa}$,$ \>{bb}\2\6
\&{integer}~\1\|i$,$ \>{imax}$,$ \>{imin}\2\6
\&{double} \&{precision}~\1\>{fact}$(\WO{20})$\2\7
\WX{\M{39}}Factorials\X \X\7
$\>{imax}=\@{min}(\|j,\ \39\|l)$\6
$\>{imin}=\@{max}(\WO{0},\ \39\|j-\|m)$\7
$\>{sum}=\WO{0.0\^D00}$\6
\&{do} $\|i=\>{imin},\ \39\>{imax}$\1\7
$\>{term}=\>{fact}(\|l+\WO{1})\ast\>{fact}(\|m+\WO{1})\WSl(\>{fact}(\|i+\WO{1})%
\ast\>{fact}(\|j-\|i+\WO{1}))$\6
$\>{term}=\>{term}\WSl(\>{fact}(\|l-\|i+\WO{1})\ast\>{fact}(\|m-\|j+\|i+%
\WO{1}))$\6
$\>{aa}=\WO{1.0\^D00};$\6
$\>{bb}=\WO{1.0\^D00}$\6
$\&{if}\,((\|l-\|i)\WI\WO{0})$\1\6
$\>{aa}=\|a\WEE{(\|l-\|i)}$\2\7
$\&{if}\,((\|m+\|i-\|j)\WI\WO{0})$\1\6
$\>{bb}=\|b\WEE{(\|m+\|i-\|j)}$\2\7
$\>{term}=\>{term}\ast\>{aa}\ast\>{bb}$\6
$\>{sum}=\>{sum}+\>{term}$\2\7
\&{end} \&{do}\7
$\>{fj}=\>{sum}$\7
\&{return}\2\6
\&{end}\Wendc
\fi % End of section 47 (sect. 4.3, p. 30a)

\WM48.

\fi % End of section 48 (sect. 4.3.0.1, p. 30b)

\WN49. 2 theta.
Computation of all the $\theta$ factors required from one
basis-function product; any one of them is given by
$$
\theta (j , \ell_1 , \ell_2 , a, b,  r , \gamma )
 = f_{j} (\ell_1 , \ell_2 , a, b) {{ j! \gamma^{r - j}} \over
         { r! (j - 2r)!}}
$$
The $f_j$ are computed in the body of \WCD{ \>{generi}} and passed to this
routine in \WCD{ \>{sf}}, the particular ones to use are in \WCD{ $\>{sf}(\ast,%
\ \>{isf})$}.
They are stored in \WCD{ \>{xleft}}, \WCD{ \>{yleft}} and \WCD{ \>{zleft}}
because they
are associated with electron 1 (the left-hand factor in the integrand
as it is usually written $(ij,k\ell)$).

\WY\WP \Wunnamed{code}{main.f}%
\&{subroutine} \1$\>{theta}(\>{i1max},\ \39%
\>{sf},\ \39\>{isf},\ \39\>{fact},\ \39\>{t1},\ \39\>{xleft})$\2\1\7
\&{implicit} \1\&{double} \&{precision}$\,(\|a-\|h,\ \39\|o-\|z)$\2\6
\&{integer}~\1\>{i1max}$,$ \>{isf}\2\6
\&{double} \&{precision}~\1\>{t1}\2\6
\&{double} \&{precision}~\1\>{sf}$(\WO{10},\ \39\ast),$ \>{fact}$(\ast),$ %
\>{xleft}$(\WO{5},\ \39\ast)$\2\7
\&{integer}~\1\>{i1}$,$ \>{ir1}$,$ \>{ir1max}$,$ \>{jt2}\2\6
\&{double} \&{precision}~\1\>{zero}$,$ \>{sfab}$,$ \>{bbb}\2\7
\&{data} ~\1\>{zero}${/}\WO{0.0\^D00}{/}$\2\7
\&{do} $\>{i1}=\WO{1},\ \39\WO{10}$\1\6
\&{do} $\>{ir1}=\WO{1},\ \39\WO{5}$\1\6
$\>{xleft}(\>{ir1},\ \39\>{i1})=\>{zero}$\2\6
\&{end} \&{do}\2\6
\&{end} \&{do}\7
\&{do} $\WO{100}$ $\>{i1}=\WO{1},\ \39\>{i1max}$\1\6
$\>{sfab}=\>{sf}(\>{i1},\ \39\>{isf})$\7
$\&{if}\,(\>{sfab}\WS\>{zero})$\1\6
\&{go} \&{to} $\WO{100}$\2\7
$\>{ir1max}=(\>{i1}-\WO{1})\WSl\WO{2}+\WO{1}$\6
$\>{bbb}=\>{sfab}\ast\>{fact}(\>{i1})\WSl\>{t1}\WEE{(\>{i1}-\WO{1})}$\6
\&{do} $\>{ir1}=\WO{1},\ \39\>{ir1max}$\1\6
$\>{jt2}=\>{i1}+\WO{2}-\>{ir1}-\>{ir1}$\6
$\>{xleft}(\>{ir1},\ \39\>{i1})=\>{bbb}\ast(\>{t1}\WEE{(\>{ir1}-\WO{1})})\WSl(%
\>{fact}(\>{ir1})\ast\>{fact}(\>{jt2}))$\2\6
\&{end} \&{do}\2\7
\Wlbl{\WO{100}\Colon\ }\&{continue}\7
\&{return}\2\6
\&{end}\Wendc
\fi % End of section 49 (sect. 4.3.1, p. 31a)

\WM50.

\fi % End of section 50 (sect. 4.3.1.1, p. 31b)

\WN51. 2 bform.
Use the pre-computed $f_j$ and $\theta$ to form the
\lq\lq $B$ \rq\rq\ factors, the final geometrical expansion
coefficients arising from the products of Cartesian monomials. Any one
of them is given by
\begin{eqnarray*}
  B_{\ell , \ell' , r_1 , r_2 , i } (\ell_1 , \ell_2 , \vec{A}_x ,
   \vec{B}_x , \vec{P}_x , \gamma_1 ;\ell_3 , \ell_4 , \vec{C}_x ,
   \vec{D}_x , \vec{Q}_x , \gamma_2 )  \\
 =  (-1)^{\ell'}
 \theta (\ell , \ell_1 , \ell_2 , \vec{PA}_x, \vec{PB}_x, r, \gamma_1 )
\theta (\ell' , \ell_3 , \ell_4 , \vec{QC}_x, \vec{QD}_x, r', \gamma_2 ) \\
  \times  \frac{(-1)^i (2\delta)^{2(r + r')}(\ell + \ell' -2r-2r')!
     \delta^i \vec{p}_x^{\ell + \ell' -2(r + r' +i)} }
     { (4\delta)^{\ell + \ell'} i! [\ell + \ell' -2(r + r'+i)]!}
\end{eqnarray*}


\WY\WP \Wunnamed{code}{main.f}%
\&{subroutine} \1$\>{bform}(\>{i1max},\ \39%
\>{i2max},\ \39\>{sf},\ \39\>{isf},\ \39\>{fact},\ \39\>{xleft},\ \39\>{t2},\ %
\39\>{delta},\ \39\>{ppx},\ \39\>{bbx},\ \39\>{xyorz})$\2\1\7
\&{implicit} \1\&{double} \&{precision}$\,(\|a-\|h,\ \39\|o-\|z)$\2\6
\&{integer}~\1\>{i1max}$,$ \>{i2max}$,$ \>{isf}\2\6
\&{double} \&{precision}~\1\>{fact}$(\ast),$ \>{sf}$(\WO{10},\ \39\ast),$ %
\>{xleft}$(\WO{5},\ \39\ast),$ \>{bbx}$(\ast),$ \>{ppx}$(\WO{20})$\2\6
\&{double} \&{precision}~\1\>{delta}\2\6
\&{integer}~\1\>{xyorz}$,$ \>{itab}\2\7
\&{double} \&{precision}~\1\>{zero}$,$ \>{one}$,$ \>{two}$,$ \>{twodel}$,$ %
\>{fordel}$,$ \>{sfab}$,$ \>{sfcd}\2\6
\&{double} \&{precision}~\1\>{bbc}$,$ \>{bbd}$,$ \>{bbe}$,$ \>{bbf}$,$ %
\>{bbg}$,$ \>{ppqq}\2\6
\&{integer}~\1\>{i1}$,$ \>{i2}$,$ \>{jt1}$,$ \>{jt2}$,$ \>{ir1max}$,$ %
\>{ir2max}\2\6
\&{data} ~\1\>{zero}$,$ \>{one}$,$ \>{two}${/}\WO{0.0\^D00},\ \39\WO{1.0\^D00},%
\ \39\WO{2.0\^D00}{/}$\2\7
$\>{itab}=\WO{0}$\7
$\&{if}\,(\>{xyorz}\WS\>{isf})$\1\6
$\>{itab}=\WO{1}$\2\7
$\>{twodel}=\>{two}\ast\>{delta};$\6
$\>{fordel}=\>{two}\ast\>{twodel}$\7
\&{do} $\WO{200}$ $\>{i1}=\WO{1},\ \39\>{i1max}$\1\7
$\>{sfab}=\>{sf}(\>{i1},\ \39\>{isf})$\6
$\&{if}\,(\>{sfab}\WS\>{zero})$\1\6
\&{go} \&{to} $\WO{200}$\2\6
$\>{ir1max}=(\>{i1}-\WO{1})\WSl\WO{2}+\WO{1}$\7
\&{do} $\WO{210}$ $\>{i2}=\WO{1},\ \39\>{i2max}$\1\7
$\>{sfcd}=\>{sf}(\>{i2},\ \39\>{isf}+\WO{3})$\6
$\&{if}\,(\>{sfcd}\WS\>{zero})$\1\6
\&{go} \&{to} $\WO{210}$\2\6
$\>{jt1}=\>{i1}+\>{i2}-\WO{2}$\6
$\>{ir2max}=(\>{i2}-\WO{1})\WSl\WO{2}+\WO{1}$\6
$\>{bbc}=(({-}\>{one})\WEE{(\>{i2}-\WO{1})})\ast\>{sfcd}\ast\>{fact}(\>{i2})%
\WSl(\>{t2}\WEE{(\>{i2}-\WO{1})}\ast(\>{fordel}\WEE{\>{jt1}}))$\7
\&{do} $\WO{220}$ $\>{ir1}=\WO{1},\ \39\>{ir1max}$\1\7
$\>{jt2}=\>{i1}+\WO{2}-\>{ir1}-\>{ir1}$\6
$\>{bbd}=\>{bbc}\ast\>{xleft}(\>{ir1},\ \39\>{i1})$\6
$\&{if}\,(\>{bbd}\WS\>{zero})$\1\6
\&{go} \&{to} $\WO{220}$\2\7
\&{do} $\WO{230}$ $\>{ir2}=\WO{1},\ \39\>{ir2max}$\1\7
$\>{jt3}=\>{i2}+\WO{2}-\>{ir2}-\>{ir2}$\6
$\>{jt4}=\>{jt2}+\>{jt3}-\WO{2}$\6
$\>{irumax}=(\>{jt4}+\>{itab})\WSl\WO{2}+\WO{1}$\6
$\>{jt1}=\>{ir1}+\>{ir1}+\>{ir2}+\>{ir2}-\WO{4}$\7
$\>{bbe}=\>{bbd}\ast(\>{t2}\WEE{(\>{ir2}-\WO{1})})\ast(\>{twodel}\WEE{\>{jt1}})%
\ast\>{fact}(\>{jt4}+\WO{1})\WSl(\>{fact}(\>{ir2})\ast\>{fact}(\>{jt3}))$\7
\&{do} $\WO{240}$ $\>{iru}=\WO{1},\ \39\>{irumax}$\1\7
$\>{jt5}=\>{jt4}-\>{iru}-\>{iru}+\WO{3}$\6
$\>{ppqq}=\>{ppx}(\>{jt5})$\6
$\&{if}\,(\>{ppqq}\WS\>{zero})$\1\6
\&{go} \&{to} $\WO{240}$\2\7
$\>{bbf}=\>{bbe}\ast(({-}\>{delta})\WEE{(\>{iru}-\WO{1})})\ast\>{ppqq}\WSl(%
\>{fact}(\>{iru})\ast\>{fact}(\>{jt5}))$\7
$\>{bbg}=\>{one}$\7
$\&{if}\,(\>{itab}\WS\WO{1})$ \&{then}\1\7
$\>{bbg}=\@{dfloat}(\>{jt4}+\WO{1})\ast\>{ppx}(\WO{2})\WSl(\>{delta}\ast%
\@{dfloat}(\>{jt5}))$\2\7
\&{end} \&{if}\7
$\>{bbf}=\>{bbf}\ast\>{bbg}$\6
$\>{nux}=\>{jt4}-\>{iru}+\WO{2}$\6
$\>{bbx}(\>{nux})=\>{bbx}(\>{nux})+\>{bbf}$\2\7
\Wlbl{\WO{240}\Colon\ }\&{continue}\2\6
\Wlbl{\WO{230}\Colon\ }\&{continue}\2\6
\Wlbl{\WO{220}\Colon\ }\&{continue}\2\6
\Wlbl{\WO{210}\Colon\ }\&{continue}\2\6
\Wlbl{\WO{200}\Colon\ }\&{continue}\7
\&{return}\2\6
\&{end}\Wendc
\fi % End of section 51 (sect. 4.3.2, p. 32)

\WM52.

\fi % End of section 52 (sect. 4.3.2.1, p. 33)

\WN53. 1 auxg.
Find the maximum value of $F_\nu$ required, use \WCD{ \>{fmch}} to
compute it and obtain all the lower $F_\nu$ by downward recursion.
$$
F_{\nu-1}(x) = {{\exp(-x) + 2 x F_\nu (x) } \over {2 \nu -1 }}
$$

\WY\WP \Wunnamed{code}{main.f}%
\&{subroutine} \1$\>{auxg}(\>{mmax},\ \39\|x,\ %
\39\|g)$\2\1\7
\&{implicit} \1\&{double} \&{precision}$\,(\|a-\|h,\ \39\|o-\|z)$\2\6
\&{integer}~\1\>{mmax}\2\6
\&{double} \&{precision}~\1\|x$,$ \|g$(\ast)$\2\7
\&{double} \&{precision}~\1\>{fmch}\2\7
\&{double} \&{precision}~\1\>{two}$,$ \|y\2\6
\&{integer}~\1\>{mp1mx}$,$ \>{mp1}$,$ \>{md}$,$ \>{mdm}\2\6
\&{data} ~\1\>{two}${/}\WO{2.0\^D00}{/}$\2\7
$\|y=\@{dexp}({-}\|x)$\6
$\>{mp1mx}=\>{mmax}+\WO{1}$\6
$\|g(\>{mp1mx})=\>{fmch}(\>{mmax},\ \39\|x,\ \39\|y)$\6
$\&{if}\,(\>{mmax}<\WO{1})$\1\6
\&{go} \&{to} $\WO{303}$\2\5
\WC{ just in case!  }\7
\WC{ Now do the recursion  downwards }\7
\&{do} $\>{mp1}=\WO{1},\ \39\>{mmax}$\1\7
$\>{md}=\>{mp1mx}-\>{mp1}$\6
$\>{mdm}=\>{md}-\WO{1}$\6
$\|g(\>{md})=(\>{two}\ast\|x\ast\|g(\>{md}+\WO{1})+\|y)\WSl\@{dfloat}(\WO{2}%
\ast\>{mdm}+\WO{1})$\2\7
\&{end} \&{do}\7
\Wlbl{\WO{303}\Colon\ }\&{return}\2\6
\&{end}\WY\Wendc
\fi % End of section 53 (sect. 4.4, p. 34a)

\WM54.

\fi % End of section 54 (sect. 4.4.0.1, p. 34b)

\WN55. 2 fmch. This code is for the oldest and most general
and reliable of the methods of computing
\begin{equation}
 F_\nu (x) = \int_0^1 t^{2 \nu} \exp (-x t^2) dt
\end{equation}
One of two possible series expansions is used depending on the value of x.

For \WCD{ $\|x\WL\WO{10}$} (Small \WCD{ \|x} Case) the (potentially) infinite
series
\begin{equation}
 F_\nu (x) = \frac{1}{2} \exp(-x) \sum_{i=0}^{\infty}
   \frac{\Gamma (\nu + \frac{1}{2} ) }
   {\Gamma (\nu + i + \frac{3}{2})} x^i
\end{equation}
is used.

The series is truncated when the value of terms falls below $10^{-8}$.
However, if the series seems to be becoming unreasonably long before
this condition is reached (more than 50 terms), the evaluation is stopped
and the function aborted with an error message on \WCD{ \WUC{ERROR\_OUTPUT%
\_UNIT}}.

If \WCD{ $\|x>\WO{10}$} (Large \WCD{ \|x} Case) a different series expansion is
used:
%
\begin{equation}
 F_\nu(x) = \frac{\Gamma(\nu + \frac{1}{2})}{2x^{\nu + \frac{1}{2}}}
         - \frac{1}{2} \exp(-x) \sum_{i=0}^{\infty}
           \frac{\Gamma(\nu + \frac{1}{2})}{\Gamma(\nu- i + \frac{3}{2})}
           x^{-i}
\end{equation}
%
This series, in fact, diverges but it diverges so slowly that the error
obtained in truncating
it is always less than the last term in the truncated series. Thus,
Thus, to obtain a value of the function to the same accuracy as the other
series,
the expansion is terminated when the last term is less than the same criterion
($10^{-8}$).

It can be shown that the minimum term is always for \WCD{ \|i} close to
$\nu + x$, thus ifthe terms for this value of \WCD{ \|i} are not below the
criterion,
the series expansion is abandoned, a message output on \WCD{ \WUC{ERROR\_OUTPUT%
\_UNIT}}
and the function aborted.

The third argument, \WCD{ \|y}, is $exp(-x)$, since it is assumed that this
function
will only be used {\it once} to evaluate the function $F_\nu(x)$ for the
maximum value
of $\nu$ required and other values will be obtained by downward recursion of
the form
%
\begin{equation}
 F_{\nu-1}(x) = \frac{\exp(-x) + 2xF_\nu(x)}{2\nu-1}
\end{equation}
%
which also requires the value of $\exp(-x)$ to be available.
%

\ \\ \ \\
\begin{minipage}{4.5in}
\ \\
\begin{description}
\item[NAME] \         \\
 fmch

\item[SYNOPSIS] \     \\
 {\tt double precision function fmch(nu,x,y) \\
   \ \\
  implicit double precision (a-h,o-z) \\
  double precision x, y \\
  integer nu \\
 }

\item[DESCRIPTION] \  \\
 Computes
\[
  F_\nu (x) = \int_0^1 t^{2\nu} e^{-x t^2} dt
\]
given $\nu$ and $x$. It is used in the evaluation of GTF
nuclear attraction and electron-repulsion integrals.

\item[ARGUMENTS] \    \\
\begin{description}
\item[nu] Input: The value of $\nu$ in the explicit formula above ({\tt
integer})
\item[x] Input: $x$ in the formula ({\tt double precision})
\item[y] Input: $\exp(-x)$, assumed to be available.
\end{description}

\item[DIAGNOSTICS] \  \\
If the relevant series of expansion used do not converge to a tolerance
of $10^{-8}$, an error message is printed on standard output and the
computation
aborted.
\end{description}
\ \\ \ \\
\end{minipage}
\ \\ \ \\


\WY\WP \Wunnamed{code}{main.f}%
 \&{double} \&{precision} \&{function} \1$%
\>{fmch}(\>{nu},\ \39\|x,\ \39\|y)$\2\1\6
\WX{\M{56}}Declarations\X \X\5
\WC{ First, make the variable declarations }\6
\WX{\M{57}}Internal Declarations\X \X\6
$\|m=\>{nu}$\6
$\|a=\@{dfloat}(\|m)$\6
$\&{if}\,(\|x\WL\>{ten})$ \&{then}\1\6
\WX{\M{58}}Small x Case\X \X\2\6
\&{else}\1\6
\WX{\M{59}}Large x Case\X \X\2\6
\&{end} \&{if}\2\6
\&{end}\WY\Wendc
\fi % End of section 55 (sect. 4.4.1, p. 35)

\WM56. Here are the declarations and \WCD{  \&{data}}  statements which are ...

\WY\WP\4\4\WX{\M{56}}Declarations\X \X${}\WSQ{}$\6
\&{implicit} \1\&{double} \&{precision}$\,(\|a-\|h,\ \39\|o-\|z)$\2\6
\&{double} \&{precision}~\1\|x$,$ \|y\2\6
\&{integer}~\1\>{nu}\2\Wendc
\WU section~\M{55}.
\fi % End of section 56 (sect. 4.4.1.1, p. 37a)

\WM57.

\WY\WP\4\4\WX{\M{57}}Internal Declarations\X \X${}\WSQ{}$\6
\&{double} \&{precision}~\1\>{ten}$,$ \>{half}$,$ \>{one}$,$ \>{zero}$,$ %
\>{rootpi4}$,$ \>{xd}$,$ \>{crit}\2\6
\&{double} \&{precision}~\1\>{term}$,$ \>{partialsum}\2\6
\&{integer}~\1\|m$,$ \|i$,$ \>{numberofterms}$,$ \>{maxone}$,$ \>{maxtwo}\2\6
\&{data} ~\1\>{zero}$,$ \>{half}$,$ \>{one}$,$ \>{rootpi4}$,$ \>{ten}${/}%
\WO{0.0\^D00},\ \39\WO{0.5\^D00},\ \39\WO{1.0\^D00},\ \39\WO{0.88622692\^D00},\
\39\WO{10.0\^D00}{/}$\2\5
\WC{ \WCD{ \>{crit}} is required accuracy of the series expansion }\6
\&{data} ~\1\>{crit}${/}\WO{1.0\^D-08}{/}$\2\5
\WC{ \WCD{ \>{maxone}} }\6
\&{data} ~\1\>{maxone}${/}\WO{50}{/},$ \>{maxtwo}${/}\WO{200}{/}$\2\Wendc
\WU section~\M{55}.
\fi % End of section 57 (sect. 4.4.1.2, p. 37b)

\WM58.

\WY\WP\4\4\WX{\M{58}}Small x Case\X \X${}\WSQ{}$\6
$\|a=\|a+\>{half}$\6
$\>{term}=\>{one}\WSl\|a$\6
$\>{partialsum}=\>{term}$\6
\&{do} $\|i=\WO{2},\ \39\>{maxone}$\1\6
$\|a=\|a+\>{one}$\6
$\>{term}=\>{term}\ast\|x\WSl\|a$\6
$\>{partialsum}=\>{partialsum}+\>{term}$\6
$\&{if}\,(\>{term}\WSl\>{partialsum}<\>{crit})$\1\6
\&{go} \&{to} $\WO{111}$\2\2\6
\&{end} \&{do}\6
\Wlbl{\WO{111}\Colon\ }\&{continue}\6
$\&{if}\,(\|i\WS\>{maxone})$ \&{then}\1\6
$\&{write}\,(\WUC{ERROR\_OUTPUT\_UNIT},\ \39\WO{200})$ \6
\Wlbl{\WO{200}\Colon\ }$\&{format}\,(\.{'i\ >\ 50\ in\ fmch'})$ \6
\WUC{STOP}\2\6
\&{end} \&{if}\6
$\>{fmch}=\>{half}\ast\>{partialsum}\ast\|y$\6
\&{return}\Wendc
\WU section~\M{55}.
\fi % End of section 58 (sect. 4.4.1.3, p. 37c)

\WM59.

\WY\WP\4\4\WX{\M{59}}Large x Case\X \X${}\WSQ{}$\6
$\|b=\|a+\>{half}$\6
$\|a=\|a-\>{half}$\6
$\>{xd}=\>{one}\WSl\|x$\6
$\>{approx}=\>{rootpi4}\ast\@{dsqrt}(\>{xd})\ast\>{xd}\WEE{\|m}$\6
$\&{if}\,(\|m>\WO{0})$ \&{then}\1\6
\&{do} $\|i=\WO{1},\ \39\|m$\1\6
$\|b=\|b-\>{one}$\6
$\>{approx}=\>{approx}\ast\|b$\2\6
\&{end} \&{do}\2\6
\&{end} \&{if}\6
$\>{fimult}=\>{half}\ast\|y\ast\>{xd}$\6
$\>{partialsum}=\>{zero}$\7
$\&{if}\,(\>{fimult}\WS\>{zero})$ \&{then}\1\6
$\>{fmch}=\>{approx}$\6
\&{return}\2\6
\&{end} \&{if}\7
$\>{fiprop}=\>{fimult}\WSl\>{approx}$\6
$\>{term}=\>{one}$\6
$\>{partialsum}=\>{term}$\6
$\>{numberofterms}=\>{maxtwo}$\6
\&{do} $\|i=\WO{2},\ \39\>{numberofterms}$\1\6
$\>{term}=\>{term}\ast\|a\ast\>{xd}$\6
$\>{partialsum}=\>{partialsum}+\>{term}$\6
$\&{if}\,(\@{dabs}(\>{term}\ast\>{fiprop}\WSl\>{partialsum})\WL\>{crit})$ %
\&{then}\1\6
$\>{fmch}=\>{approx}-\>{fimult}\ast\>{partialsum}$\6
\&{return}\2\6
\&{end} \&{if}\6
$\|a=\|a-\>{one}$\2\6
\&{end} \&{do}\6
$\&{write}\,(\WUC{ERROR\_OUTPUT\_UNIT},\ \39\WO{201})$ \6
\Wlbl{\WO{201}\Colon\ }$\&{format}\,(\.{'\ numberofterms\ reached\ in\0\
fmch'})$ \6
\WUC{STOP}\WY\Wendc
\WU section~\M{55}.
\fi % End of section 59 (sect. 4.4.1.4, p. 38a)

\WM60.

\fi % End of section 60 (sect. 4.4.1.5, p. 38b)

\WN61.  INTEGRAL STORAGE AND PROCESSING.

\fi % End of section 61 (sect. 5, p. 38c)

\WM62.

\fi % End of section 62 (sect. 5.0.0.1, p. 38d)

\WN63. 1 getint. This function withdraws $(ij,kl)$ two-electron integral
from the \WCD{ \.{file}}.

\WY\WP \Wunnamed{code}{main.f}%
 \&{integer} \&{function} \1$\>{getint}(%
\.{file},\ \39\|i,\ \39\|j,\ \39\|k,\ \39\|l,\ \39\>{mu},\ \39\>{val},\ \39%
\>{pointer})$\2\1\7
\&{integer}~\1\.{file}$,$ \|i$,$ \|j$,$ \|k$,$ \|l$,$ \>{mu}$,$ \>{pointer}\2\6
\&{double} \&{precision}~\1\>{val}\2\6
\&{save}\1\2\7
\&{integer}~\1\>{max\_pointer}$,$ \>{id}$,$ \>{iend}\2\6
\&{double} \&{precision}~\1\>{zero}\2\6
\&{double} \&{precision}~\1\>{value}$(\WUC{INT\_BLOCK\_SIZE})$\2\6
$\&{character}{\ast\WO{8}}~$\1\>{labels}$(\WUC{INT\_BLOCK\_SIZE})$\2\6
\&{data} ~\1\>{max\_pointer}${/}\WO{0}{/},$ \>{iend}{/}\WUC{NOT\_LAST%
\_BLOCK}{/}$,$ \>{zero}${/}\WO{0.0\^D00}{/}$\2\7
\WC{ File must be rewound before first use of this function           and
pointer must be set to 0 }\7
$\&{if}\,(\>{pointer}\WS\>{max\_pointer})$ \&{then}\1\6
$\&{if}\,(\>{iend}\WS\WUC{LAST\_BLOCK})$ \&{then}\1\6
$\>{val}=\>{zero};$\6
$\|i=\WO{0};$\6
$\|j=\WO{0};$\6
$\|k=\WO{0};$\6
$\|l=\WO{0}$\6
$\>{max\_pointer}=\WO{0};$\6
$\>{iend}=\WUC{NOT\_LAST\_BLOCK}$\6
$\>{getint}=\WUC{END\_OF\_FILE}$\6
\&{return}\2\6
\&{end} \&{if}\6
$\&{read}\,(\.{file})$ \>{max\_pointer}$,$ \>{iend}$,$ \>{labels}$,$ \>{value}\6
$\>{pointer}=\WO{0}$\2\6
\&{end} \&{if}\6
$\>{pointer}=\>{pointer}+\WO{1}$\6
\&{call} $\>{unpack}(\>{labels}(\>{pointer}),\ \39\|i,\ \39\|j,\ \39\|k,\ \39%
\|l,\ \39\>{mu},\ \39\>{id})$\6
$\>{val}=\>{value}(\>{pointer})$\6
$\>{getint}=\WUC{OK}$\7
\&{return}\2\6
\&{end}\Wendc
\fi % End of section 63 (sect. 5.1, p. 39a)

\WM64.


\fi % End of section 64 (sect. 5.1.0.1, p. 39b)

\WN65. 1 putint. This function is just happy.

\WY\WP \Wunnamed{code}{main.f}%
\&{subroutine} \1$\>{putint}(\>{nfile},\ \39%
\|i,\ \39\|j,\ \39\|k,\ \39\|l,\ \39\>{mu},\ \39\>{val},\ \39\>{pointer},\ \39%
\>{last})$\2\1\6
\&{implicit} \1\&{double} \&{precision}$\,(\|a-\|h,\ \39\|o-\|z)$\2\6
\&{save}\1\2\7
\&{integer}~\1\>{nfile}$,$ \|i$,$ \|j$,$ \|k$,$ \|l$,$ \>{mu}$,$ \>{pointer}$,$
\>{last}\2\6
\&{double} \&{precision}~\1\>{value}$(\WUC{INT\_BLOCK\_SIZE})$\2\6
$\&{character}{\ast\WO{8}}~$\1\>{labels}$(\WUC{INT\_BLOCK\_SIZE})$\2\6
\&{double} \&{precision}~\1\>{val}\2\6
\&{data} ~\1\>{max\_pointer}{/}\WUC{INT\_BLOCK\_SIZE}{/}$,$ \>{id}${/}%
\WO{0}{/}$\2\5
\WC{           id is now unused     }\7
$\&{if}\,(\>{last}\WS\.{ERR})$\1\6
\&{go} \&{to} $\WO{100}$\2\6
$\>{iend}=\WUC{NOT\_LAST\_BLOCK}$\6
$\&{if}\,(\>{pointer}\WS\>{max\_pointer})$ \&{then}\1\6
$\&{write}\,(\>{nfile})$ \>{pointer}$,$ \>{iend}$,$ \>{labels}$,$ \>{value}\6
$\>{pointer}=\WO{0}$\2\6
\&{end} \&{if}\6
$\>{pointer}=\>{pointer}+\WO{1}$\6
\&{call} $\>{pack}(\>{labels}(\>{pointer}),\ \39\|i,\ \39\|j,\ \39\|k,\ \39\|l,%
\ \39\>{mu},\ \39\>{id})$\6
$\>{value}(\>{pointer})=\>{val}$\6
$\&{if}\,(\>{last}\WS\WUC{YES})$ \&{then}\1\6
$\>{iend}=\WUC{LAST\_BLOCK}$\6
$\>{last}=\.{ERR}$\6
$\&{write}\,(\>{nfile})$ \>{pointer}$,$ \>{iend}$,$ \>{labels}$,$ \>{value}\2\6
\&{end} \&{if}\7
\Wlbl{\WO{100}\Colon\ }\&{return}\2\6
\&{end}\Wendc
\fi % End of section 65 (sect. 5.2, p. 40a)

\WM66.
\fi % End of section 66 (sect. 5.2.0.1, p. 40b)

\WN67. 1 genint. This subroutine generates one- and two-electron integrals.

\WY\WP \Wunnamed{code}{main.f}%
 \&{subroutine} \1$\>{genint}(\>{ngmx},\ \39%
\>{nbfns},\ \39\>{eta},\ \39\>{ntype},\ \39\>{ncntr},\ \39\>{nfirst},\ \39%
\>{nlast},\ \39\>{vlist},\ \39\>{ncmx},\ \39\>{noc},\ \39\|S,\ \39\|H,\ \39%
\>{nfile})$\2 \&{integer}~\1\>{ngmx}$,$ \>{nbfns}$,$ \>{noc}$,$ \>{ncmx}\2\6
\&{double} \&{precision}~\1\>{eta}$(\WUC{MAX\_PRIMITIVES},\ \39\WO{5}),$ %
\>{vlist}$(\WUC{MAX\_CENTRES},\ \39\WO{4})$\2\6
\&{double} \&{precision}~\1\|S$(\WUC{ARB}),$ \|H$(\WUC{ARB})$\2\6
\&{integer}~\1\>{ntype}$(\WUC{ARB}),$ \>{nfirst}$(\WUC{ARB}),$ \>{nlast}$(%
\WUC{ARB}),$ \>{ncntr}$(\WUC{ARB}),$ \>{nfile}\2\7
\&{integer}~\1\|i$,$ \|j$,$ \|k$,$ \|l$,$ \>{ltop}$,$ \>{ij}$,$ \>{ji}$,$ %
\>{mu}$,$ \|m$,$ \|n$,$ \>{jtyp}$,$ \>{js}$,$ \>{jf}$,$ \>{ii}$,$ \>{jj}\2\6
\&{double} \&{precision}~\1\>{generi}$,$ \>{genoei}\2\6
\&{integer}~\1\>{pointer}$,$ \>{last}\2\6
\&{double} \&{precision}~\1\>{ovltot}$,$ \>{kintot}\2\6
\&{double} \&{precision}~\1\>{val}$,$ \>{crit}$,$ \>{alpha}$,$ \|t$,$ \>{t1}$,$
\>{t2}$,$ \>{t3}$,$ \>{sum}$,$ \>{pitern}\2\6
\&{double} \&{precision}~\1\WUC{SOO}\2\6
\&{double} \&{precision}~\1\>{gtoC}$(\WUC{MAX\_PRIMITIVES})$\2\6
\&{double} \&{precision}~\1\>{dfact}$(\WO{20})$\2\6
\&{integer}~\1\>{nr}$(\WUC{NO\_OF\_TYPES},\ \39\WO{3})$\2\6
\&{data} ~\1\>{nr}${/}\WO{0},\ \39\WO{1},\ \39\WO{0},\ \39\WO{0},\ \39\WO{2},\ %
\39\WO{0},\ \39\WO{0},\ \39\WO{1},\ \39\WO{1},\ \39\WO{0},\ \39\WO{3},\ \39%
\WO{0},\ \39\WO{0},\ \39\WO{2},\ \39\WO{2},\ \39\WO{1},\ \39\WO{0},\ \39\WO{1},%
\ \39\WO{0},\ \39\WO{1},\ \39\WO{0},\ \39\WO{0},\ \39\WO{1},\ \39\WO{0},\ \39%
\WO{0},\ \39\WO{2},\ \39\WO{0},\ \39\WO{1},\ \39\WO{0},\ \39\WO{1},\ \39\WO{0},%
\ \39\WO{3},\ \39\WO{0},\ \39\WO{1},\ \39\WO{0},\ \39\WO{2},\ \39\WO{2},\ \39%
\WO{0},\ \39\WO{1},\ \39\WO{1},\ \39\WO{0},\ \39\WO{0},\ \39\WO{0},\ \39\WO{1},%
\ \39\WO{0},\ \39\WO{0},\ \39\WO{2},\ \39\WO{0},\ \39\WO{1},\ \39\WO{1},\ \39%
\WO{0},\ \39\WO{0},\ \39\WO{3},\ \39\WO{0},\ \39\WO{1},\ \39\WO{0},\ \39\WO{1},%
\ \39\WO{2},\ \39\WO{2},\ \39\WO{1}{/}$\2\6
\&{data} ~\1\>{crit}$,$ \>{half}$,$ \>{onep5}$,$ \>{one}$,$ \>{zero}${/}\WO{1.0%
\^D-08},\ \39\WO{0.5\^D+00},\ \39\WO{1.5\^D+00},\ \39\WO{1.0\^D+00},\ \39%
\WO{0.0\^D+00}{/}$\2\6
\&{data} ~\1\>{dfact}${/}\WO{1.0},\ \39\WO{3.0},\ \39\WO{15.0},\ \39\WO{105.0},%
\ \39\WO{945.0},\ \39\WO{10395.0},\ \39\WO{135135.0},\ \39\WO{2027025.0},\ \39%
\WO{12}\ast\WO{0.0}{/}$\2\6
\&{data} ~\1\>{gtoC}${/}\WUC{MAX\_PRIMITIVES}\ast\WO{0.0\^D+00}{/}$\2\7
$\>{mu}=\WO{0}$\7
\WX{\M{68}}Copy GTO contraction coeffs to gtoC\X \X\7
\WX{\M{69}}Normalize the primitives\X \X\7
\WC{ one electron integrals }\7
$\WRS{DO}\,\|i$  $=$ $\WO{1},\ \39\>{nbfns}$ $\WRS{DO}\,\|j$  $=$ $\WO{1},\ \39%
\|i$\6
$\>{ij}=(\|j-\WO{1})\ast\>{nbfns}+\|i;$\6
$\>{ji}=(\|i-\WO{1})\ast\>{nbfns}+\|j$\6
$\|H(\>{ij})=\>{genoei}(\|i,\ \39\|j,\ \39\>{eta},\ \39\>{ngmx},\ \39%
\>{nfirst},\ \39\>{nlast},\ \39\>{ntype},\ \39\>{nr},\ \39\WUC{NO\_OF\_TYPES},\
\39\>{vlist},\ \39\>{noc},\ \39\>{ncmx},\ \39\>{ovltot},\ \39\>{kintot})$\6
$\|H(\>{ji})=\|H(\>{ij})$\6
$\|S(\>{ij})=\>{ovltot};$\6
$\|S(\>{ji})=\>{ovltot}$ \.{END} \WRS{DO} \.{END} \WRS{DO}\6
$\&{write}\,(\ast,\ \39\ast)$ $\.{"\ ONE\ ELECTRON\ INTEGRALS\ C\0OMPUTED"}$\7
\&{rewind} \>{nfile};\6
$\>{pointer}=\WO{0}$\6
$\>{last}=\WUC{NO}$\6
$\|i=\WO{1};$\6
$\|j=\WO{1};$\6
$\|k=\WO{1};$\6
$\|l=\WO{0}$\7
$\WRS{DO}\,\WO{10}$ \1\6
$\WUC{WHILE}(\>{next\_label}(\|i,\ \39\|j,\ \39\|k,\ \39\|l,\ \39\>{nbfns})\WS%
\WUC{YES})$\2\6
$\WUC{IF}(\|l\WS\>{nbfns})\>{last}=\WUC{YES}$\6
$\>{val}=\>{generi}(\|i,\ \39\|j,\ \39\|k,\ \39\|l,\ \39\WO{0},\ \39\>{eta},\ %
\39\>{ngmx},\ \39\>{nfirst},\ \39\>{nlast},\ \39\>{ntype},\ \39\>{nr},\ \39%
\WUC{NO\_OF\_TYPES})$ $\WUC{IF}(\@{dabs}(\>{val})<\>{crit})$ \&{go} \&{to} $%
\WO{10}$\6
$\WUC{CALL}\>{putint}(\>{nfile},\ \39\|i,\ \39\|j,\ \39\|k,\ \39\|l,\ \39%
\>{mu},\ \39\>{val},\ \39\>{pointer},\ \39\>{last})$\6
\Wlbl{\WO{10}\Colon\ }\WUC{CONTINUE}\7
\&{return} \&{end}\WY\Wendc
\fi % End of section 67 (sect. 5.3, p. 41)

\WM68.

\WY\WP\4\4\WX{\M{68}}Copy GTO contraction coeffs to gtoC\X \X${}\WSQ{}$\6
\&{do} $\|i=\WO{1},\ \39\>{ngmx}$\1\6
$\>{gtoC}(\|i)=\>{eta}(\|i,\ \39\WO{5})$\2\6
\&{end} \&{do}\Wendc
\WU section~\M{67}.
\fi % End of section 68 (sect. 5.3.0.1, p. 42)

\WM69.


\WY\WP\4\4\WX{\M{69}}Normalize the primitives\X \X${}\WSQ{}$\6
\WC{ First, normalize the primitives }\6
$\>{pitern}=\WO{5.568327997\^D+00}$\5
\WC{ pi**1.5 }\6
\&{do} $\|j=\WO{1},\ \39\>{nbfns}$\1\6
$\>{jtyp}=\>{ntype}(\|j);$\6
$\>{js}=\>{nfirst}(\|j);$\6
$\>{jf}=\>{nlast}(\|j)$\6
$\|l=\>{nr}(\>{jtyp},\ \39\WO{1});$\6
$\|m=\>{nr}(\>{jtyp},\ \39\WO{2});$\6
$\|n=\>{nr}(\>{jtyp},\ \39\WO{3})$\6
\&{do} $\|i=\>{js},\ \39\>{jf}$\1\6
$\>{alpha}=\>{eta}(\|i,\ \39\WO{4});$\6
$\WUC{SOO}=\>{pitern}\ast(\>{half}\WSl\>{alpha})\WEE{\WO{1.5}}$\6
$\>{t1}=\>{dfact}(\|l+\WO{1})\WSl\>{alpha}\WEE{\|l}$\6
$\>{t2}=\>{dfact}(\|m+\WO{1})\WSl\>{alpha}\WEE{\|m}$\6
$\>{t3}=\>{dfact}(\|n+\WO{1})\WSl\>{alpha}\WEE{\|n}$\6
$\>{eta}(\|i,\ \39\WO{5})=\>{one}\WSl\@{dsqrt}(\WUC{SOO}\ast\>{t1}\ast\>{t2}%
\ast\>{t3})$\2\6
\&{end} \&{do}\2\6
\&{end} \&{do}\5
\WC{ Now normalize the basis functions }\6
\&{do} $\|j=\WO{1},\ \39\>{nbfns}$\1\6
$\>{jtyp}=\>{ntype}(\|j);$\6
$\>{js}=\>{nfirst}(\|j);$\6
$\>{jf}=\>{nlast}(\|j)$\6
$\|l=\>{nr}(\>{jtyp},\ \39\WO{1});$\6
$\|m=\>{nr}(\>{jtyp},\ \39\WO{2});$\6
$\|n=\>{nr}(\>{jtyp},\ \39\WO{3})$\7
$\>{sum}=\>{zero}$\6
\&{do} $\>{ii}=\>{js},\ \39\>{jf}$\1\6
\&{do} $\>{jj}=\>{js},\ \39\>{jf}$\1\6
$\|t=\>{one}\WSl(\>{eta}(\>{ii},\ \39\WO{4})+\>{eta}(\>{jj},\ \39\WO{4}))$\6
$\WUC{SOO}=\>{pitern}\ast(\|t\WEE{\>{onep5}})\ast\>{eta}(\>{ii},\ \39\WO{5})%
\ast\>{eta}(\>{jj},\ \39\WO{5})$\6
$\|t=\>{half}\ast\|t$\6
$\>{t1}=\>{dfact}(\|l+\WO{1})\WSl\|t\WEE{\|l}$\6
$\>{t2}=\>{dfact}(\|m+\WO{1})\WSl\|t\WEE{\|m}$\6
$\>{t3}=\>{dfact}(\|n+\WO{1})\WSl\|t\WEE{\|n}$\6
$\>{sum}=\>{sum}+\>{gtoC}(\>{ii})\ast\>{gtoC}(\>{jj})\ast\WUC{SOO}\ast\>{t1}%
\ast\>{t2}\ast\>{t3}$\2\6
\&{end} \&{do}\2\6
\&{end} \&{do}\6
$\>{sum}=\>{one}\WSl\@{sqrt}(\>{sum})$\6
\&{do} $\>{ii}=\>{js},\ \39\>{jf}$\1\6
$\>{gtoC}(\>{ii})=\>{gtoC}(\>{ii})\ast\>{sum}$\2\6
\&{end} \&{do}\2\6
\&{end} \&{do}\7
\&{do} $\>{ii}=\WO{1},\ \39\>{ngmx}$\1\6
$\>{eta}(\>{ii},\ \39\WO{5})=\>{eta}(\>{ii},\ \39\WO{5})\ast\>{gtoC}(\>{ii})$\2%
\6
\&{end} \&{do}\Wendc
\WU section~\M{67}.
\fi % End of section 69 (sect. 5.3.0.2, p. 43)

\WM70.

\fi % End of section 70 (sect. 5.3.0.3, p. 44a)

\WN71.  UTILITIES. The utility functions

\fi % End of section 71 (sect. 6, p. 44b)

\WM72.

\fi % End of section 72 (sect. 6.0.0.1, p. 44c)

\WN73. 1 gtprd.

\WY\WP \Wunnamed{defs}{main.f}%
\WMd{}\>{loch}$(\|i,\|j)$\5
$(\|n\ast(\|j-\WO{1})+\|i)$\Wendd
\WY\WP \Wunnamed{code}{main.f}%
\7
\&{subroutine} \1$\>{gtprd}(\|A,\ \39\|B,\ \39\|R,\ \39\|n,\ \39\|m,\ \39\|l)$%
\2\1\6
\&{double} \&{precision}~\1\|A$(\WUC{ARB}),$ \|B$(\WUC{ARB})$\2\6
\&{double} \&{precision}~\1\|R$(\WUC{ARB})$\2\6
\&{integer}~\1\|n$,$ \|m$,$ \|l\2\7
\&{double} \&{precision}~\1\>{zero}\2\6
\&{integer}~\1\|k$,$ \>{ik}$,$ \|j$,$ \>{ir}$,$ \>{ij}$,$ \>{ib}\2\6
\&{data} ~\1\>{zero}${/}\WO{0.0\^D+00}{/}$\2\5
\WC{ stride counters initialization }\7
$\>{ir}=\WO{0};$\6
$\>{ik}={-}\|n$\6
\&{do} $\|k=\WO{1},\ \39\|l$\1\6
$\>{ij}=\WO{0}$\6
$\>{ik}=\>{ik}+\|m$\6
\&{do} $\|j=\WO{1},\ \39\|m$\1\6
$\>{ir}=\>{ir}+\WO{1};$\6
$\>{ib}=\>{ik}$\6
$\|R(\>{ir})=\>{zero}$\6
\&{do} $\|i=\WO{1},\ \39\|n$\1\6
$\>{ij}=\>{ij}+\WO{1};$\6
$\>{ib}=\>{ib}+\WO{1}$\6
$\|R(\>{ir})=\|R(\>{ir})+\|A(\>{ij})\ast\|B(\>{ib})$\2\6
\&{enddo}\2\6
\&{enddo}\2\6
\&{enddo}\7
\&{return}\2\6
\&{end}\Wendc
\fi % End of section 73 (sect. 6.1, p. 44d)

\WM74.

\fi % End of section 74 (sect. 6.1.0.1, p. 44e)

\WN75. 1 gmprd.

\WY\WP \Wunnamed{code}{main.f}%
\&{subroutine} \1$\>{gmprd}(\|A,\ \39\|B,\ \39%
\|R,\ \39\|n,\ \39\|m,\ \39\|l)$\2\1\6
\&{double} \&{precision}~\1\|A$(\WUC{ARB}),$ \|B$(\WUC{ARB})$\2\6
\&{double} \&{precision}~\1\|R$(\WUC{ARB})$\2\6
\&{integer}~\1\|n$,$ \|m$,$ \|l\2\7
\&{double} \&{precision}~\1\>{zero}\2\6
\&{integer}~\1\|k$,$ \>{ik}$,$ \|j$,$ \>{ir}$,$ \>{ji}$,$ \>{ib}\2\6
\&{data} ~\1\>{zero}${/}\WO{0.0\^D+00}{/}$\2\5
\WC{ stride counters initialization }\7
$\>{ir}=\WO{0};$\6
$\>{ik}={-}\|m$\6
\&{do} $\|k=\WO{1},\ \39\|l$\1\6
$\>{ik}=\>{ik}+\|m$\6
\&{do} $\|j=\WO{1},\ \39\|n$\1\6
$\>{ir}=\>{ir}+\WO{1};$\6
$\>{ji}=\|j-\|n;$\6
$\>{ib}=\>{ik}$\6
$\|R(\>{ir})=\>{zero}$\6
\&{do} $\|i=\WO{1},\ \39\|m$\1\6
$\>{ji}=\>{ji}+\|n;$\6
$\>{ib}=\>{ib}+\WO{1}$\6
$\|R(\>{ir})=\|R(\>{ir})+\|A(\>{ji})\ast\|B(\>{ib})$\2\6
\&{enddo}\2\6
\&{enddo}\2\6
\&{enddo}\7
\&{return}\2\6
\&{end}\WY\Wendc
\fi % End of section 75 (sect. 6.2, p. 45a)

\WM76.

\fi % End of section 76 (sect. 6.2.0.1, p. 45b)

\WN77. 1 eigen.

\WY\WP \Wunnamed{code}{main.f}%
\&{subroutine} \1$\>{eigen}(\|H,\ \39\|U,\ \39%
\|n)$\2\1\6
\&{implicit} \1\&{double} \&{precision}$\,(\|a-\|h,\ \39\|o-\|z)$\2\6
\&{double} \&{precision}~\1\|H$(\WO{1}),$ \|U$(\WO{1})$\2\6
\&{integer}~\1\|n\2\7
\&{data} ~\1\>{zero}$,$ \>{eps}$,$ \>{one}$,$ \>{two}$,$ \>{four}$,$ %
\>{big}${/}\WO{0.0\^D+00},\ \39\WO{1.0\^D-20},\ \39\WO{1.0\^D+00},\ \39\WO{2.0%
\^D+00},\ \39\WO{4.0\^D+00},\ \39\WO{1.0\^D+20}{/}$\2\5
\WC{ Initialize U matrix to unity }\7
\&{do} $\|i=\WO{1},\ \39\|n$\1\6
$\>{ii}=\>{loch}(\|i,\ \39\|i)$\6
\&{do} $\|j=\WO{1},\ \39\|n$\1\6
$\>{ij}=\>{loch}(\|i,\ \39\|j)$\6
$\|U(\>{ij})=\>{zero}$\2\6
\&{end} \&{do}\6
$\|U(\>{ii})=\>{one}$\2\6
\&{end} \&{do}\5
\WC{ start sweep through off-diagonal elements }\6
$\>{hmax}=\>{big}$\6
\&{do} $\WO{90}\>{while}(\>{hmax}>\>{eps})$\1\6
$\>{hmax}=\>{zero}$\6
\&{do} $\|i=\WO{2},\ \39\|n$\1\6
$\>{jtop}=\|i-\WO{1}$\6
\&{do} $\WO{10}$ $\|j=\WO{1},\ \39\>{jtop}$\1\6
$\>{ii}=\>{loch}(\|i,\ \39\|i);$\6
$\>{jj}=\>{loch}(\|j,\ \39\|j)$\6
$\>{ij}=\>{loch}(\|i,\ \39\|j);$\6
$\>{ji}=\>{loch}(\|j,\ \39\|i)$\6
$\>{hii}=\|H(\>{ii});$\6
$\>{hjj}=\|H(\>{jj});$\6
$\>{hij}=\|H(\>{ij})$\6
$\>{hsq}=\>{hij}\ast\>{hij}$\6
$\&{if}\,(\>{hsq}>\>{hmax})$\1\6
$\>{hmax}=\>{hsq}$\2\6
$\&{if}\,(\>{hsq}<\>{eps})$\1\6
\&{go} \&{to} $\WO{10}$\2\6
$\>{del}=\>{hii}-\>{hjj};$\6
$\@{sign}=\>{one}$\6
$\&{if}\,(\>{del}<\>{zero})$ \&{then}\1\6
$\@{sign}={-}\>{one}$\6
$\>{del}={-}\>{del}$\2\6
\&{end} \&{if}\6
$\>{denom}=\>{del}+\@{dsqrt}(\>{del}\ast\>{del}+\>{four}\ast\>{hsq})$\6
$\@{tan}=\>{two}\ast\@{sign}\ast\>{hij}\WSl\>{denom}$\6
$\|c=\>{one}\WSl\@{dsqrt}(\>{one}+\@{tan}\ast\@{tan})$\6
$\|s=\|c\ast\@{tan}$\6
\&{do} $\WO{20}$ $\|k=\WO{1},\ \39\|n$\1\6
$\>{kj}=\>{loch}(\|k,\ \39\|j);$\6
$\>{ki}=\>{loch}(\|k,\ \39\|i)$\6
$\>{jk}=\>{loch}(\|j,\ \39\|k);$\6
$\>{ik}=\>{loch}(\|i,\ \39\|k)$\6
$\>{temp}=\|c\ast\|U(\>{kj})-\|s\ast\|U(\>{ki})$\6
$\|U(\>{ki})=\|s\ast\|U(\>{kj})+\|c\ast\|U(\>{ki});$\6
$\|U(\>{kj})=\>{temp}$\6
$\&{if}\,((\|i\WS\|k)\WOR(\|j\WS\|k))$\1\6
\&{go} \&{to} $\WO{20}$\2\5
\WC{ update the parts of H matrix affected by a rotation }\6
$\>{temp}=\|c\ast\|H(\>{kj})-\|s\ast\|H(\>{ki})$\6
$\|H(\>{ki})=\|s\ast\|H(\>{kj})+\|c\ast\|H(\>{ki})$\6
$\|H(\>{kj})=\>{temp};$\6
$\|H(\>{ik})=\|H(\>{ki});$\6
$\|H(\>{jk})=\|H(\>{kj})$\2\6
\Wlbl{\WO{20}\Colon\ }\&{continue}\5
\WC{ now transform the four elements explicitly targeted by theta }\6
$\|H(\>{ii})=\|c\ast\|c\ast\>{hii}+\|s\ast\|s\ast\>{hjj}+\>{two}\ast\|c\ast\|s%
\ast\>{hij}$\6
$\|H(\>{jj})=\|c\ast\|c\ast\>{hjj}+\|s\ast\|s\ast\>{hii}-\>{two}\ast\|c\ast\|s%
\ast\>{hij}$\6
$\|H(\>{ij})=\>{zero};$\6
$\|H(\>{ji})=\>{zero}$\2\6
\Wlbl{\WO{10}\Colon\ }\&{continue}\2\6
\&{end} \&{do}\5
\WC{ Finish when largest off-diagonal is small enough }\6
\Wlbl{\WO{90}\Colon\ }\&{continue}\5
\WC{ Now sort the eigenvectors into eigenvalue order }\6
$\>{iq}={-}\|n$\6
\&{do} $\|i=\WO{1},\ \39\|n$\1\6
$\>{iq}=\>{iq}+\|n;$\6
$\>{ii}=\>{loch}(\|i,\ \39\|i);$\6
$\>{jq}=\|n\ast(\|i-\WO{2})$\6
\&{do} $\|j=\|i,\ \39\|n$\1\6
$\>{jq}=\>{jq}+\|n;$\6
$\>{jj}=\>{loch}(\|j,\ \39\|j)$\6
$\&{if}\,(\|H(\>{ii})<\|H(\>{jj}))$\1\6
\&{go} \&{to} $\WO{30}$\2\6
$\>{temp}=\|H(\>{ii});$\6
$\|H(\>{ii})=\|H(\>{jj});$\6
$\|H(\>{jj})=\>{temp}$\6
\&{do} $\|k=\WO{1},\ \39\|n$\1\6
$\>{ilr}=\>{iq}+\|k;$\6
$\>{imr}=\>{jq}+\|k$\6
$\>{temp}=\|U(\>{ilr});$\6
$\|U(\>{ilr})=\|U(\>{imr});$\6
$\|U(\>{imr})=\>{temp}$\2\6
\&{end} \&{do}\6
\Wlbl{\WO{30}\Colon\ }\&{continue}\2\6
\&{end} \&{do}\2\6
\&{end} \&{do}\6
\&{return}\2\6
\&{end}\Wendc
\fi % End of section 77 (sect. 6.3, p. 46)

\WM78.
\fi % End of section 78 (sect. 6.3.0.1, p. 47)

\WN79. 1 pack. Store the six electron repulsion labels.

\WY\WP \Wunnamed{code}{main.f}%
\&{subroutine} \1$\>{pack}(\|a,\ \39\|i,\ \39%
\|j,\ \39\|k,\ \39\|l,\ \39\|m,\ \39\|n)$\2\1\6
$\&{character}{\ast\WO{8}}~$\1\|a$,$ \|b\2\6
\&{integer}~\1\|i$,$ \|j$,$ \|k$,$ \|l$,$ \|m$,$ \|n\2\7
\&{data} ~\1\|b${/}\.{"\ \ \ \ \ \ \ \ "}{/}$\2\7
$\|a=\|b$\6
$\|a(\WO{1}:\WO{1})=\@{char}(\|i);$\6
$\|a(\WO{2}:\WO{2})=\@{char}(\|j)$\6
$\|a(\WO{3}:\WO{3})=\@{char}(\|k);$\6
$\|a(\WO{4}:\WO{4})=\@{char}(\|l)$\6
$\|a(\WO{5}:\WO{5})=\@{char}(\|m);$\6
$\|a(\WO{6}:\WO{6})=\@{char}(\|n)$\6
\&{return}\2\6
\&{end}\Wendc
\fi % End of section 79 (sect. 6.4, p. 48a)

\WM80.

\fi % End of section 80 (sect. 6.4.0.1, p. 48b)

\WN81. 1 unpack. Regenerate the 6 electron repulsion labels.

\WY\WP \Wunnamed{code}{main.f}%
\&{subroutine} \1$\>{unpack}(\|a,\ \39\|i,\ \39%
\|j,\ \39\|k,\ \39\|l,\ \39\|m,\ \39\|n)$\2\1\6
$\&{character}{\ast\WO{8}}~$\1\|a\2\6
\&{integer}~\1\|i$,$ \|j$,$ \|k$,$ \|l$,$ \|m$,$ \|n\2\7
$\|i=\@{ichar}(\|a(\WO{1}:\WO{1}));$\6
$\|j=\@{ichar}(\|a(\WO{2}:\WO{2}))$\6
$\|k=\@{ichar}(\|a(\WO{3}:\WO{3}));$\6
$\|l=\@{ichar}(\|a(\WO{4}:\WO{4}))$\6
$\|m=\@{ichar}(\|a(\WO{5}:\WO{5}));$\6
$\|n=\@{ichar}(\|a(\WO{6}:\WO{6}))$\7
\&{return}\2\6
\&{end}\Wendc
\fi % End of section 81 (sect. 6.5, p. 48c)

\WM82.

\fi % End of section 82 (sect. 6.5.0.1, p. 48d)

\WN83. 1 next\_label. Generate the next label of electron repulsion integral.

A function to generate  the
four standard loops which are used to generate (or, more rarely) process
the electron repulsion integrals.

The sets of integer values are generated in the usual
standard order in canonical form, that is, equivalent to the set of loops: \\
 \WCD{ \&{do} $\|i=\WO{1},\ \|n$ $\{$ \&{do} $\|j=\WO{1},\ \|i$ $\{$ \&{do} $%
\|k=\WO{1},\ \|i$ $\{$ $\>{ltop}=\|k$ $\&{if}\,(\|i\WS\|k)$ $\>{ltop}=\|j$ %
\&{do} $\|l=\WO{1},\ \>{ltop}$ $\{$ \&{do} \>{something}\>{with}\|i\|j\|k\|l $%
\}$ $\}$ $\}$ $\}$}
\ \\
Note that, just as is the case with the \WCD{ \&{do}}-loops,
the whole process must be {\em initialised\/} by
setting initial values of \WCD{ \|i}, \WCD{ \|j}, \WCD{ \|k} and \WCD{ \|l}.
If the whole set of labels is required then \\
\WCD{ $\|i=\WO{1}$}, \WCD{ $\|j=\WO{1}$}, \WCD{ $\|k=\WO{1}$}, \WCD{ \|l}=0 \\
is appropriate.

Usage is, typically, \\
\WCD{$\|i=\WO{0}$ $\|j=\WO{0}$ $\|k=\WO{0}$ $\|l=\WO{0}$} \\
\WCD{ $\>{while}(\>{next\_label}(\|i,\ \|j,\ \|k,\ \|l,\ \|n)\WS\WUC{YES})$} \\
\WCD{ $\{$} \\
  do something with i j k and l \\
\WCD{ $\}$} \\

\WY\WP \Wunnamed{code}{main.f}%
 \&{integer} \&{function} \1$\>{next\_label}(%
\|i,\ \39\|j,\ \39\|k,\ \39\|l,\ \39\|n)$\2\1\6
\&{integer}~\1\|i$,$ \|j$,$ \|k$,$ \|l$,$ \|n\2\7
\&{integer}~\1\>{ltop}\2\7
$\>{next\_label}=\WUC{YES}$\6
$\>{ltop}=\|k$\6
$\&{if}\,(\|i\WS\|k)$\1\6
$\>{ltop}=\|j$\2\7
$\&{if}\,(\|l<\>{ltop})$ \&{then}\1\6
$\|l=\|l+\WO{1}$\2\6
\&{else}\1\6
$\|l=\WO{1}$\6
$\&{if}\,(\|k<\|i)$ \&{then}\1\6
$\|k=\|k+\WO{1}$\2\6
\&{else}\1\6
$\|k=\WO{1}$\6
$\&{if}\,(\|j<\|i)$ \&{then}\1\6
$\|j=\|j+\WO{1}$\2\6
\&{else}\1\6
$\|j=\WO{1}$\6
$\&{if}\,(\|i<\|n)$ \&{then}\1\6
$\|i=\|i+\WO{1}$\2\6
\&{else}\1\6
$\>{next\_label}=\WUC{NO}$\2\6
\&{end} \&{if}\2\6
\&{end} \&{if}\2\6
\&{end} \&{if}\2\6
\&{end} \&{if}\6
\&{return}\2\6
\&{end}\WY\Wendc
\fi % End of section 83 (sect. 6.6, p. 49)

\WM84.

\fi % End of section 84 (sect. 6.6.0.1, p. 50a)

\WN85. 1 shalf. This subroutine calculates ${\bf S}^{-\frac{1}{2}}$ matrix
from ${\bf S}$ matrix.

\WY\WP \Wunnamed{code}{main.f}%
\&{subroutine} \1$\>{shalf}(\|S,\ \39\|U,\ \39%
\|W,\ \39\|m)$\2\1\6
\&{implicit} \1\&{double} \&{precision}$\,(\|a-\|h,\ \39\|o-\|z)$\2\6
\&{double} \&{precision}~\1\|S$(\ast),$ \|U$(\ast),$ \|W$(\ast)$\2\6
\&{integer}~\1\|m\2\7
\&{data} ~\1\>{crit}$,$ \>{one}${/}\WO{1.0\^D-10},\ \39\WO{1.0\^D+00}{/}$\2\7
\&{call} $\>{eigen}(\|S,\ \39\|U,\ \39\|m)$\5
\WC{ Transpose the eigenvalues of S for convenience }\6
\&{do} $\|i=\WO{1},\ \39\|m$\1\6
\&{do} $\|j=\WO{1},\ \39\|i$\1\6
$\>{ij}=\|m\ast(\|j-\WO{1})+\|i;$\6
$\>{ji}=\|m\ast(\|i-\WO{1})+\|j;$\6
$\|d=\|U(\>{ij})$\6
$\|U(\>{ij})=\|U(\>{ji});$\6
$\|U(\>{ji})=\|d$\2\6
\&{end} \&{do}\2\6
\&{end} \&{do}\5
\WC{ Get the inverse root of the eigenvalues }\6
\&{do} $\|i=\WO{1},\ \39\|m$\1\6
$\>{ii}=(\|i-\WO{1})\ast\|m+\|i$\6
$\&{if}\,(\|S(\>{ii})<\>{crit})$ \&{then}\1\6
$\&{write}\,(\WUC{ERROR\_OUTPUT\_UNIT},\ \39\WO{200})$ \6
\WUC{STOP}\2\6
\&{end} \&{if}\6
$\|S(\>{ii})=\>{one}\WSl\@{dsqrt}(\|S(\>{ii}))$\2\6
\&{end} \&{do}\6
\&{call} $\>{gtprd}(\|U,\ \39\|S,\ \39\|W,\ \39\|m,\ \39\|m,\ \39\|m)$\6
\&{call} $\>{gmprd}(\|W,\ \39\|U,\ \39\|S,\ \39\|m,\ \39\|m,\ \39\|m)$\7
\&{return}\6
\Wlbl{\WO{200}\Colon\ }$\&{format}\,(\.{"\ Basis\ is\ linearly\ depend\0ent;\ S%
\ is\ singular!\ "})$ \2\6
\&{end}\WY\Wendc
\fi % End of section 85 (sect. 6.7, p. 50b)

\WM86.


\fi % End of section 86 (sect. 6.7.0.1, p. 50c)

\WN87. 1 spinor.

\WY\WP \Wunnamed{code}{main.f}%
\&{subroutine} \1$\>{spinor}(\|H,\ \39\|m)$\2\1%
\6
\&{double} \&{precision}~\1\|H$(\ast)$\2\6
\&{integer}~\1\|m\2\7
\&{double} \&{precision}~\1\>{zero}\2\6
\&{integer}~\1\|i$,$ \|j$,$ \>{ij}$,$ \>{ji}$,$ \>{ip}$,$ \>{jp}$,$ \>{ijp}$,$ %
\>{ijd}$,$ \>{nl}$,$ \|n\2\6
\&{data} ~\1\>{zero}${/}\WO{0.0\^D+00}{/}$\2\7
$\|n=\WO{2}\ast\|m;$\6
$\>{nl}=\|m+\WO{1}$\7
\&{do} $\|i=\WO{1},\ \39\|m$\1\6
\&{do} $\|j=\WO{1},\ \39\|m$\1\6
$\>{ij}=\|m\ast(\|j-\WO{1})+\|i;$\6
$\>{ip}=\|i+\|m;$\6
$\>{jp}=\|j+\|m$\6
$\>{ijp}=\|n\ast(\>{jp}-\WO{1})+\>{ip};$\6
$\|H(\>{ijp})=\|H(\>{ij})$\2\6
\&{end} \&{do}\2\6
\&{end} \&{do}\7
\&{do} $\|i=\WO{1},\ \39\|m$\1\6
\&{do} $\|j=\WO{1},\ \39\|m$\1\6
$\>{ip}=\|i+\|m;$\6
$\>{jp}=\|j+\|m;$\6
$\>{ijp}=\|n\ast(\>{jp}-\WO{1})+\>{ip}$\6
$\>{ijd}=\|n\ast(\|j-\WO{1})+\|i;$\6
$\|H(\>{ijd})=\|H(\>{ijp})$\2\6
\&{end} \&{do}\2\6
\&{end} \&{do}\7
\&{do} $\|i=\WO{1},\ \39\|m$\1\6
\&{do} $\|j=\>{nl},\ \39\|n$\1\6
$\>{ij}=\|n\ast(\|j-\WO{1})+\|i;$\6
$\>{ji}=\|n\ast(\|i-\WO{1})+\|j$\6
$\|H(\>{ij})=\>{zero}$\6
$\|H(\>{ji})=\>{zero}$\2\6
\&{end} \&{do}\2\6
\&{end} \&{do}\7
\&{return}\2\6
\&{end}\Wendc
\fi % End of section 87 (sect. 6.8, p. 51a)

\WM88.

\fi % End of section 88 (sect. 6.8.0.1, p. 51b)

\WN89.  INDEX.
\fi % End of section 89 (sect. 7, p. 52)

\input INDEX.tex
\input MODULES.tex

\Winfo{"fweave -C3 main.web"}  {"main.web"} {(none)}
 {\Fortran}


\Wcon{89}
\FWEBend
