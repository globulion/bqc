% FWEAVE v1.62 (September 23, 1998)

% --- FWEB's macro package ---
\input fwebmac.sty

% --- Initialization parameters from FWEB's style file `fweb.sty' ---
\Wbegin[;]
  % #1 --- [LaTeX.class.options;LaTeX.package.options]
{article;}
  % #2 --- {LaTeX.class;LaTeX.package}
{1em}
  % #3 --- {indent.TeX}
{1em}
  % #4 --- {indent.code}
{CONTENTS.tex}
  % #5 --- {contents.TeX}
{ % #6 ---
 {\&\WRS}
  % #1 --- {{format.reserved}{format.RESERVED}}
 {\|}
  % #2 --- {format.short_id}
 {\>\WUC}
  % #3 --- {{format.id}{format.ID}}
 {\>\WUC}
  % #4 --- {{format.outer_macro}{format.OUTER_MACRO}}
 {\>\WUC}
  % #5 --- {{format.WEB_macro}{format.WEB_MACRO}}
 {\@}
  % #6 --- {format.intrinsic}
 {\.\.}
  % #7 --- {{format.keyword}{format.KEYWORD}}
 {\.}
  % #8 --- {format.typewriter}
 {}
  % #9 --- (For future use)
}
{\M}
  % #7 --- {encap.prefix}
{;}
  % #8 --- {doc.preamble;doc.postamble}
{INDEX}
  % #9 --- {index.name}


% --- Beginning of user's limbo section ---

\def\title{---INTEGRALS---}





% --- Limbo text from style-file parameter `limbo.end' ---
\FWEBtoc

\WN1.  genoei.
Function to compute the one-electron integrals (overlap,
kinetic energy and nuclear attraction).
The STRUCTURES and GENOEI manual pages must be
consulted for a detailed description of the calling sequence.

The overlap and kinetic energy integrals are expressed in terms of
a basic one-dimensional Cartesian overlap component computed by
\WCD{ \&{function} \>{ovrlap}} while the more involved nuclear-attraction
integrals are computed as a sum of geometrical factors computed by
\WCD{ \&{subroutine} \>{aform}} and the standard $F_\nu$ computed by \WCD{ %
\&{function} \>{fmch}}.


\WY\WP \Wunnamed{defs}{integral.f}%
\WMd{}\WUC{ERROR\_OUTPUT\_UNIT}\5
$\WO{6}$\WY\Wendd
\WY\WP \Wunnamed{code}{integral.f}%
\7
\&{double} \&{precision} \&{function} \1$\>{genoei}(\|i,\ \39\|j,\ \39\>{eta},\
\39\>{ngmx},\ \39\>{nfirst},\ \39\>{nlast},\ \39\>{ntype},\ \39\>{nr},\ \39%
\>{ntmx},\ \39\>{vlist},\ \39\>{noc},\ \39\>{ncmx},\ \39\>{ovltot},\ \39%
\>{kintot})$\2 \&{implicit} \1\&{double} \&{precision}$\,(\|a-\|h,\ \39\|o-%
\|z)$\2\6
\&{integer}~\1\|i$,$ \|j$,$ \>{ngmx}$,$ \>{ncmx}$,$ \>{noc}$,$ \>{ntmx}\2\6
\&{integer}~\1\>{nfirst}$(\ast),$ \>{nlast}$(\ast),$ \>{ntype}$(\ast),$ %
\>{nr}$(\>{ntmx},\ \39\WO{3})$\2\6
\&{double} \&{precision}~\1\>{ovltot}$,$ \>{kintot}\2\6
\&{double} \&{precision}~\1\>{eta}$(\WUC{MAX\_PRIMITIVES},\ \39\WO{5}),$ %
\>{vlist}$(\WUC{MAX\_CENTRES},\ \39\WO{4})$\2\7
\WC{ Insert delarations which are purely local to \WCD{ \>{genoei}} }\7
\WX{\M{2}}genoei local declarations\X \X\7
\WC{ Insert the Factorials }\7
\WX{\M{13}}Factorials\X \X\7
\WC{ Obtain the powers of x,y,z and summation limits }\7
\WX{\M{3}}One-electron Integer Setup\X \X\7
\WC{ Inter-nuclear distance }\7
$\>{rAB}=(\>{eta}(\>{iss},\ \39\WO{1})-\>{eta}(\>{jss},\ \39\WO{1}))\WEE{%
\WO{2}}+(\>{eta}(\>{iss},\ \39\WO{2})-\>{eta}(\>{jss},\ \39\WO{2}))\WEE{%
\WO{2}}+(\>{eta}(\>{iss},\ \39\WO{3})-\>{eta}(\>{jss},\ \39\WO{3}))\WEE{%
\WO{2}}$\7
\WC{ Initialise all accumulators   }\7
$\>{genoei}=\>{zero}$\6
$\>{totnai}=\>{zero}$\6
$\>{kintot}=\>{zero}$\6
$\>{ovltot}=\>{zero}$\7
\WC{ Now start the summations over the contracted GTFs  }\7
\&{do} $\>{irun}=\>{iss},\ \39\>{il}$\1\5
\WC{ start of "i" contraction }\7
\&{do} $\>{jrun}=\>{jss},\ \39\>{jl}$\1\5
\WC{ start of "j" contraction }\7
\WX{\M{15}}Compute PA\X \X\5
\WC{ Use the Gaussian-product theorem to find $\vec{P}$ }\7
\WX{\M{4}}Overlap Components\X \X\7
$\>{ovltot}=\>{ovltot}+\>{anorm}\ast\>{bnorm}\ast\>{ovl}$\5
\WC{ accumulate Overlap }\7
$\&{write}\,(\ast,\ \39\ast)$ \>{ovltot}$,$ $\.{"HAPPY"}$\7
\WX{\M{6}}Kinetic Energy Components\X \X\7
$\>{kintot}=\>{kintot}+\>{anorm}\ast\>{bnorm}\ast\>{kin}$\5
\WC{  accumulate  Kinetic energy  }\7
\WC{  now the nuclear attraction integral   }\6
$\>{tnai}=\>{zero}$\7
\WX{\M{7}}Form fj\X \X\5
\WC{ Generate the required $f_j$ coefficients }\7
\&{do} $\|n=\WO{1},\ \39\>{noc}$\1\5
\WC{ loop over nuclei }\7
$\>{pn}=\>{zero}$\5
\WC{ Initialise current contribution  }\7
\WC{ Get the attracting-nucleus information;  co-ordinates }\7
\WX{\M{10}}Nuclear data\X \X\7
$\|t=\>{t1}\ast\>{pcsq}$\7
\&{call} $\>{auxg}(\|m,\ \39\|t,\ \39\|g)$\5
\WC{ Generate all the $F_\nu$ required }\7
\WX{\M{8}}Form As\X \X\5
\WC{ Generate the geometrical $A$-factors }\7
\WC{ Now sum the products of the geometrical $A$-factors and the $F_\nu$ }\7
\&{do} $\>{ii}=\WO{1},\ \39\>{imax}$\1\6
\&{do} $\>{jj}=\WO{1},\ \39\>{jmax}$\1\6
\&{do} $\>{kk}=\WO{1},\ \39\>{kmax}$\1\6
$\>{nu}=\>{ii}+\>{jj}+\>{kk}-\WO{2}$\6
$\>{pn}=\>{pn}+\>{Airu}(\>{ii})\ast\>{Ajsv}(\>{jj})\ast\>{Aktw}(\>{kk})\ast\|g(%
\>{nu})$\2\6
\&{end} \&{do}\2\6
\&{end} \&{do}\2\6
\&{end} \&{do}\7
$\>{tnai}=\>{tnai}-\>{pn}\ast\>{vn},\ \39\WO{4}$ $)$  \5
\WC{ Add to total multiplied by currentrrent charge }\2\7
\&{end} \&{do}\5
\WC{  end of loop over nuclei  }\6
$\>{totnai}=\>{totnai}+\>{prefa}\ast\>{tnai}$\2\6
\&{end} \&{do}\5
\WC{  end of "j" contraction  }\2\6
\&{end} \&{do}\5
\WC{ end of "i" contraction  }\7
$\>{genoei}=\>{totnai}+\>{kintot}$\5
\WC{ "T + V"  }\6
\&{return} \&{end}\WY\Wendc
\fi % End of section 1 (sect. 1, p. 1)

\WM2. These are the declarations which are local to \WCD{ \>{genoei}},
working space {\em etc.}

\WY\WP\4\4\WX{\M{2}}genoei local declarations\X \X${}\WSQ{}$\6
\&{double} \&{precision}~\1\>{Airu}$(\WO{10}),$ \>{Ajsv}$(\WO{10}),$ \>{Aktw}$(%
\WO{10})$\2\6
\&{double} \&{precision}~\1\|p$(\WO{3}),$ \>{sf}$(\WO{10},\ \39\WO{3}),$ %
\>{tf}$(\WO{20})$\2\6
\&{double} \&{precision}~\1\>{fact}$(\WO{20}),$ \|g$(\WO{50})$\2\6
\&{double} \&{precision}~\1\>{kin}\2\6
\&{data} ~\1\>{zero}$,$ \>{one}$,$ \>{two}$,$ \>{half}$,$ \>{quart}${/}\WO{0.0%
\^D00},\ \39\WO{1.0\^D00},\ \39\WO{2.0\^D00},\ \39\WO{0.5\^D00},\ \39\WO{0.25%
\^D00}{/}$\2\6
\&{data} ~\1\>{pi}${/}\WO{3.141592653589\^D00}{/}$\2\WY\Wendc
\WU section~\M{1}.
\fi % End of section 2 (sect. 1.1, p. 2a)

\WM3. Get the various powers of $x$, $y$ and $z$ required from the data
structures and obtain the contraction limits etc.

\WY\WP\4\4\WX{\M{3}}One-electron Integer Setup\X \X${}\WSQ{}$\6
$\>{ityp}=\>{ntype}(\|i);$\6
$\>{jtyp}=\>{ntype}(\|j)$\6
$\>{l1}=\>{nr}(\>{ityp},\ \39\WO{1});$\6
$\>{m1}=\>{nr}(\>{ityp},\ \39\WO{2});$\6
$\>{n1}=\>{nr}(\>{ityp},\ \39\WO{3})$\6
$\>{l2}=\>{nr}(\>{jtyp},\ \39\WO{1});$\6
$\>{m2}=\>{nr}(\>{jtyp},\ \39\WO{2});$\6
$\>{n2}=\>{nr}(\>{jtyp},\ \39\WO{3})$\6
$\>{imax}=\>{l1}+\>{l2}+\WO{1};$\6
$\>{jmax}=\>{m1}+\>{m2}+\WO{1};$\6
$\>{kmax}=\>{n1}+\>{n2}+\WO{1}$\6
$\>{maxall}=\>{imax}$\6
$\&{if}\,(\>{maxall}<\>{jmax})$\1\6
$\>{maxall}=\>{jmax}$\2\6
$\&{if}\,(\>{maxall}<\>{kmax})$\1\6
$\>{maxall}=\>{kmax}$\2\6
$\&{if}\,(\>{maxall}<\WO{2})$\1\6
$\>{maxall}=\WO{2}$\2\5
\WC{ when all functions are "s" type }\6
$\>{iss}=\>{nfirst}(\|i);$\6
$\>{il}=\>{nlast}(\|i)$\6
$\>{jss}=\>{nfirst}(\|j);$\6
$\>{jl}=\>{nlast}(\|j)$\WY\Wendc
\WU section~\M{1}.
\fi % End of section 3 (sect. 1.2, p. 3)

\WM4. This simple code gets the Cartesian overlap components and
assembles the total integral. It also computes the overlaps required
to calculate the kinetic energy integral used in a later module.

\WY\WP\4\4\WX{\M{4}}Overlap Components\X \X${}\WSQ{}$\6
$\>{prefa}=\>{two}\ast\>{prefa}$\6
$\>{expab}=\@{dexp}({-}\>{aexp}\ast\>{bexp}\ast\>{rAB}\WSl\>{t1})$\6
$\>{s00}=(\>{pi}\WSl\>{t1})\WEE{\WO{1.5}}\ast\>{expab}$\6
$\>{dum}=\>{one};$\6
$\>{tf}(\WO{1})=\>{one};$\6
$\>{del}=\>{half}\WSl\>{t1}$\6
\&{do} $\|n=\WO{2},\ \39\>{maxall}$\1\6
$\>{tf}(\|n)=\>{tf}(\|n-\WO{1})\ast\>{dum}\ast\>{del}$\6
$\>{dum}=\>{dum}+\>{two}$\2\6
\&{end} \&{do}\7
$\>{ox0}=\>{ovrlap}(\>{l1},\ \39\>{l2},\ \39\>{pax},\ \39\>{pbx},\ \39\>{tf})$\6
$\>{oy0}=\>{ovrlap}(\>{m1},\ \39\>{m2},\ \39\>{pay},\ \39\>{pby},\ \39\>{tf})$\6
$\>{oz0}=\>{ovrlap}(\>{n1},\ \39\>{n2},\ \39\>{paz},\ \39\>{pbz},\ \39\>{tf})$\6
$\>{ox2}=\>{ovrlap}(\>{l1},\ \39\>{l2}+\WO{2},\ \39\>{pax},\ \39\>{pbx},\ \39%
\>{tf})$\6
$\>{oxm2}=\>{ovrlap}(\>{l1},\ \39\>{l2}-\WO{2},\ \39\>{pax},\ \39\>{pbx},\ \39%
\>{tf})$\6
$\>{oy2}=\>{ovrlap}(\>{m1},\ \39\>{m2}+\WO{2},\ \39\>{pay},\ \39\>{pby},\ \39%
\>{tf})$\6
$\>{oym2}=\>{ovrlap}(\>{m1},\ \39\>{m2}-\WO{2},\ \39\>{pay},\ \39\>{pby},\ \39%
\>{tf})$\6
$\>{oz2}=\>{ovrlap}(\>{n1},\ \39\>{n2}+\WO{2},\ \39\>{paz},\ \39\>{pbz},\ \39%
\>{tf})$\6
$\>{ozm2}=\>{ovrlap}(\>{n1},\ \39\>{n2}-\WO{2},\ \39\>{paz},\ \39\>{pbz},\ \39%
\>{tf})$\6
$\>{ov0}=\>{ox0}\ast\>{oy0}\ast\>{oz0};$\6
$\>{ovl}=\>{ov0}\ast\>{s00}$\6
$\>{ov1}=\>{ox2}\ast\>{oy0}\ast\>{oz0};$\6
$\>{ov4}=\>{oxm2}\ast\>{oy0}\ast\>{oz0}$\6
$\>{ov2}=\>{ox0}\ast\>{oy2}\ast\>{oz0};$\6
$\>{ov5}=\>{ox0}\ast\>{oym2}\ast\>{oz0}$\6
$\>{ov3}=\>{ox0}\ast\>{oy0}\ast\>{oz2};$\6
$\>{ov6}=\>{ox0}\ast\>{oy0}\ast\>{ozm2}$\WY\Wendc
\WU section~\M{1}.
\fi % End of section 4 (sect. 1.3, p. 4)

\WN5.  ovrlap.
One-dimensional Cartesian overlap. This function uses the
precomputed factors in \WCD{ \>{tf}} to evaluate the simple Cartesian
components
of the overlap integral which must be multiplied together to
form the total overlap integral.

\WY\WP \Wunnamed{code}{integral.f}%
\7
\&{double} \&{precision} \&{function} \1$\>{ovrlap}(\>{l1},\ \39\>{l2},\ \39%
\>{pax},\ \39\>{pbx},\ \39\>{tf})$\2\1\6
\&{implicit} \1\&{double} \&{precision}$\,(\|a-\|h,\ \39\|o-\|z)$\2\6
\&{integer}~\1\>{l1}$,$ \>{l2}\2\6
\&{double} \&{precision}~\1\>{pax}$,$ \>{pbx}\2\6
\&{double} \&{precision}~\1\>{tf}$(\ast)$\2\5
\WC{ pre-computed exponent and double factorial
    factors: tf(i+1) = (2i-1)�!/(2**i*(A+B)**i) }\7
\&{double} \&{precision}~\1\>{zero}$,$ \>{one}$,$ \>{dum}\2\6
\&{data} ~\1\>{zero}$,$ \>{one}${/}\WO{0.0\^D00},\ \39\WO{1.0\^D00}{/}$\2\7
$\&{if}\,((\>{l1}<\WO{0})\WOR(\>{l2}<\WO{0}))$ \&{then}\1\6
$\>{ovrlap}=\>{zero}$\6
\&{return}\2\6
\&{end} \&{if}\7
$\&{if}\,((\>{l1}\WS\WO{0})\WW(\>{l2}\WS\WO{0}))$ \&{then}\1\6
$\>{ovrlap}=\>{one}$\6
\&{return}\2\6
\&{end} \&{if}\7
$\>{dum}=\>{zero};$\6
$\>{maxkk}=(\>{l1}+\>{l2})\WSl\WO{2}+\WO{1}$\7
\&{do} $\>{kk}=\WO{1},\ \39\>{maxkk}$\1\6
$\>{dum}=\>{dum}+\>{tf}(\>{kk})\ast\>{fj}(\>{l1},\ \39\>{l2},\ \39\WO{2}\ast%
\>{kk}-\WO{2},\ \39\>{pax},\ \39\>{pbx})$\2\6
\&{end} \&{do}\7
$\>{ovrlap}=\>{dum}$\7
\&{return}\2\6
\&{end}\WY\Wendc
\fi % End of section 5 (sect. 2, p. 5a)

\WM6. Use the previously-computed overlap components to
generate the Kinetic energy components and
hence the total integral.

\WY\WP\4\4\WX{\M{6}}Kinetic Energy Components\X \X${}\WSQ{}$\6
$\>{xl}=\@{dfloat}(\>{l2}\ast(\>{l2}-\WO{1}));$\6
$\>{xm}=\@{dfloat}(\>{m2}\ast(\>{m2}-\WO{1}))$\6
$\>{xn}=\@{dfloat}(\>{n2}\ast(\>{n2}-\WO{1}));$\6
$\>{xj}=\@{dfloat}(\WO{2}\ast(\>{l2}+\>{m2}+\>{n2})+\WO{3})$\6
$\>{kin}=\>{s00}\ast(\>{bexp}\ast(\>{xj}\ast\>{ov0}-\>{two}\ast\>{bexp}\ast(%
\>{ov1}+\>{ov2}+\>{ov3}))-\>{half}\ast(\>{xl}\ast\>{ov4}+\>{xm}\ast\>{ov5}+%
\>{xn}\ast\>{ov6}))$\WY\Wendc
\WU section~\M{1}.
\fi % End of section 6 (sect. 2.1, p. 5b)

\WM7. Form the $f_j$ coefficients needed for the nuclear attraction integral.

\WY\WP\4\4\WX{\M{7}}Form fj\X \X${}\WSQ{}$\6
$\|m=\>{imax}+\>{jmax}+\>{kmax}-\WO{2}$\6
\&{do} $\|n=\WO{1},\ \39\>{imax}$\1\6
$\>{sf}(\|n,\ \39\WO{1})=\>{fj}(\>{l1},\ \39\>{l2},\ \39\|n-\WO{1},\ \39%
\>{pax},\ \39\>{pbx})$\2\6
\&{end} \&{do}\7
\&{do} $\|n=\WO{1},\ \39\>{jmax}$\1\6
$\>{sf}(\|n,\ \39\WO{2})=\>{fj}(\>{m1},\ \39\>{m2},\ \39\|n-\WO{1},\ \39%
\>{pay},\ \39\>{pby})$\2\6
\&{end} \&{do}\7
\&{do} $\|n=\WO{1},\ \39\>{kmax}$\1\6
$\>{sf}(\|n,\ \39\WO{3})=\>{fj}(\>{n1},\ \39\>{n2},\ \39\|n-\WO{1},\ \39%
\>{paz},\ \39\>{pbz})$\2\6
\&{end} \&{do}\WY\Wendc
\WU section~\M{1}.
\fi % End of section 7 (sect. 2.2, p. 6a)

\WM8. Use \WCD{ \>{aform}} to compute the required $A$-factors for each
Cartesian component.

\WY\WP\4\4\WX{\M{8}}Form As\X \X${}\WSQ{}$\6
$\>{epsi}=\>{quart}\WSl\>{t1}$\6
\&{do} $\>{ii}=\WO{1},\ \39\WO{10}$\1\6
$\>{Airu}(\>{ii})=\>{zero}$\6
$\>{Ajsv}(\>{ii})=\>{zero}$\6
$\>{Aktw}(\>{ii})=\>{zero}$\2\6
\&{end} \&{do}\7
\&{call} $\>{aform}(\>{imax},\ \39\>{sf},\ \39\>{fact},\ \39\>{cpx},\ \39%
\>{epsi},\ \39\>{Airu},\ \39\WO{1})$\5
\WC{ form $A_{i,r,u}$  }\6
\&{call} $\>{aform}(\>{jmax},\ \39\>{sf},\ \39\>{fact},\ \39\>{cpy},\ \39%
\>{epsi},\ \39\>{Ajsv},\ \39\WO{2})$\5
\WC{ form $A_{j,s,v}$  }\6
\&{call} $\>{aform}(\>{kmax},\ \39\>{sf},\ \39\>{fact},\ \39\>{cpz},\ \39%
\>{epsi},\ \39\>{Aktw},\ \39\WO{3})$\5
\WC{ form $A_{k,t,w}$  }\WY\Wendc
\WU section~\M{1}.
\fi % End of section 8 (sect. 2.3, p. 6b)

\WN9.  aform. Compute the nuclear-attraction $A$ factors. These quantitities
arise from the components of the three position vectors of the two
basis functions and the attracting centre with respect to the
centre of the product Gaussian. There is one
of these for each of the three dimensions of Cartesian space; a typical
one (the $x$ component) is:
$$
A_{\ell,r,i} ( \ell_1 , \ell_2 , \vec{A}_x , \vec{B}_x , \vec{C}_x ,\gamma )
= (-1)^{\ell} f_{\ell} ( \ell_1, \ell_2 , \vec{PA}_x , \vec{PB}_x )
  {{ (-1)^i \ell ! \vec{PC}_x^{\ell-2r-2i} \epsilon^{r+i}} \over
       {r! i! (\ell -2r-2i)!}}
$$


\WY\WP \Wunnamed{code}{integral.f}%
\7
\&{subroutine} \1$\>{aform}(\>{imax},\ \39\>{sf},\ \39\>{fact},\ \39\>{cpx},\ %
\39\>{epsi},\ \39\>{Airu},\ \39\>{xyorz})$\2\1\6
\&{implicit} \1\&{double} \&{precision}$\,(\|a-\|h,\ \39\|o-\|z)$\2\6
\&{integer}~\1\>{imax}$,$ \>{xyorz}\2\6
\&{double} \&{precision}~\1\>{Airu}$(\ast),$ \>{fact}$(\ast),$ \>{sf}$(\WO{10},%
\ \39\ast)$\2\7
\&{double} \&{precision}~\1\>{one}\2\6
\&{data} ~\1\>{one}${/}\WO{1.0\^D00}{/}$\2\7
\&{do} $\|i=\WO{1},\ \39\>{imax}$\1\6
$\>{ai}=({-}\>{one})\WEE{(\|i-\WO{1})}\ast\>{sf}(\|i,\ \39\>{xyorz})\ast%
\>{fact}(\|i)$\6
$\>{irmax}=(\|i-\WO{1})\WSl\WO{2}+\WO{1}$\6
\&{do} $\>{ir}=\WO{1},\ \39\>{irmax}$\1\6
$\>{irumax}=\>{irmax}-\>{ir}+\WO{1}$\6
\&{do} $\>{iru}=\WO{1},\ \39\>{irumax}$\1\6
$\>{iq}=\>{ir}+\>{iru}-\WO{2}$\6
$\>{ip}=\|i-\WO{2}\ast\>{iq}-\WO{1}$\6
$\>{at5}=\>{one}$\6
$\&{if}\,(\>{ip}>\WO{0})$\1\6
$\>{at5}=\>{cpx}\WEE{\>{ip}}$\2\6
$\>{tiru}=\>{ai}\ast({-}\>{one})\WEE{(\>{iru}-\WO{1})}\ast\>{at5}\ast\>{epsi}%
\WEE{\>{iq}}\WSl(\>{fact}(\>{ir})\ast\>{fact}(\>{iru})\ast\>{fact}(\>{ip}+%
\WO{1}))$\6
$\>{nux}=\>{ip}+\>{iru}$\6
$\>{Airu}(\>{nux})=\>{Airu}(\>{nux})+\>{tiru}$\2\6
\&{end} \&{do}\2\6
\&{end} \&{do}\2\6
\&{end} \&{do}\7
\&{return}\2\6
\&{end}\WY\Wendc
\fi % End of section 9 (sect. 3, p. 7a)

\WM10. Get the co-ordinates of the attracting nucleus with respect to $%
\vec{P}$.

\WY\WP\4\4\WX{\M{10}}Nuclear data\X \X${}\WSQ{}$\6
$\>{cpx}=\|p(\WO{1})-\>{vlist}(\|n,\ \39\WO{1})$\6
$\>{cpy}=\|p(\WO{2})-\>{vlist}(\|n,\ \39\WO{2})$\6
$\>{cpz}=\|p(\WO{3})-\>{vlist}(\|n,\ \39\WO{3})$\6
$\>{pcsq}=\>{cpx}\ast\>{cpx}+\>{cpy}\ast\>{cpy}+\>{cpz}\ast\>{cpz}$\WY\Wendc
\WU section~\M{1}.
\fi % End of section 10 (sect. 3.1, p. 7b)

\WN11.  generi.
The general electron-repulsion integral formula for contracted
Gaussian basis functions. The STRUCTURES and GENERI manual pages must be
consulted for a detailed description of the calling sequence.

\WY\WP \Wunnamed{code}{integral.f}%
\7
\&{double} \&{precision} \&{function} \1$\>{generi}(\|i,\ \39\|j,\ \39\|k,\ \39%
\|l,\ \39\>{xyorz},\ \39\>{eta},\ \39\>{ngmx},\ \39\>{nfirst},\ \39\>{nlast},\ %
\39\>{ntype},\ \39\>{nr},\ \39\>{ntmx})$\2\1\7
\&{implicit} \1\&{double} \&{precision}$\,(\|a-\|h,\ \39\|o-\|z)$\2\6
\&{integer}~\1\|i$,$ \|j$,$ \|k$,$ \|l$,$ \>{xyorz}$,$ \>{ngmx}$,$ \>{ntmx}\2\6
\&{double} \&{precision}~\1\>{eta}$(\WUC{MAX\_PRIMITIVES},\ \39\WO{5})$\2\6
\&{integer}~\1\>{nfirst}$(\ast),$ \>{nlast}$(\ast),$ \>{ntype}$(\ast),$ %
\>{nr}$(\>{ntmx},\ \39\WO{3})$\2\7
\WC{  Variables local to the function }\7
\WX{\M{12}}generi local declarations\X \X\7
\WC{ Insert the \WCD{  \&{data}}  statement for the factorials }\7
\WX{\M{13}}Factorials\X \X\7
\WC{ Get the various integers from the data structures for           the
summation limits, Cartesian monomial powers etc. from           the main
integer data structures  }\7
\WX{\M{14}}Two-electron Integer Setup\X \X\7
\WC{ Two internuclear distances this time  }\7
$\>{rAB}=(\>{eta}(\>{is},\ \39\WO{1})-\>{eta}(\>{js},\ \39\WO{1}))\WEE{%
\WO{2}}+(\>{eta}(\>{is},\ \39\WO{2})-\>{eta}(\>{js},\ \39\WO{2}))\WEE{\WO{2}}+(%
\>{eta}(\>{is},\ \39\WO{3})-\>{eta}(\>{js},\ \39\WO{3}))\WEE{\WO{2}}$\6
$\>{rCD}=(\>{eta}(\>{ks},\ \39\WO{1})-\>{eta}(\>{ls},\ \39\WO{1}))\WEE{%
\WO{2}}+(\>{eta}(\>{ks},\ \39\WO{2})-\>{eta}(\>{ls},\ \39\WO{2}))\WEE{\WO{2}}+(%
\>{eta}(\>{ks},\ \39\WO{3})-\>{eta}(\>{ls},\ \39\WO{3}))\WEE{\WO{2}}$\7
\WC{ Initialise the accumulator }\7
$\>{generi}=\>{zero}$\7
\WC{ Now the real work, begin the four contraction loops }\7
\&{do} $\>{irun}=\>{is},\ \39\>{il}$\1\5
\WC{ start of "i" contraction}\7
\&{do} $\>{jrun}=\>{js},\ \39\>{jl}$\1\5
\WC{ start of "j" contraction}\7
\WC{ Get the data for the two basis functions referring to
electron 1; orbital exponents and Cartesian co-ordinates              and hence
compute the vector $\vec{P}$ and the components of              $\vec{PA}$ and
$\vec{PB}$   }\7
\WX{\M{15}}Compute PA\X \X\7
\WC{ Use \WCD{ \&{function} \>{fj}} and \WCD{ \&{subroutine} \>{theta}} to
calculate the            geometric factors arising from the expansion of the
product of            Cartesian monomials for the basis functions of electron 1
 }\7
\WX{\M{17}}Thetas for electron 1\X \X\7
\&{do} $\>{krun}=\>{ks},\ \39\>{kl}$\1\5
\WC{ start of "k" contraction}\6
\&{do} $\>{lrun}=\>{ls},\ \39\>{ll}$\1\5
\WC{ start of "l" contraction}\6
$\>{eribit}=\>{zero}$\5
\WC{ local accumulator }\7
\WC{ Get the data for the two basis functions referring to             electron
2; orbital exponents and Cartesian co-ordinates             and hence compute
the vector $\vec{Q}$ and the components of             $\vec{QC}$   and $%
\vec{QD}$   }\7
\WX{\M{16}}Compute QC\X \X\7
$\|w=\>{pi}\WSl(\>{t1}+\>{t2})$\7
\WC{ Repeat the use of \WCD{ \&{function} \>{fj}}             to obtain the
geometric factors arising from the expansion             of Cartesian monomials
for the basis functions of electron 2  }\7
\WX{\M{18}}fj for electron 2\X \X\7
\&{call} $\>{auxg}(\|m,\ \39\|t,\ \39\|g)$\5
\WC{ Obtain the $F_\nu$ by recursion }\7
\WC{ Now use the pre-computed $\theta$ factors for both electron
distributions to form the overall $B$ factors }\7
\WX{\M{19}}Form Bs\X \X\7
\WC{ Form the limits and add up all the bits, the products of              %
\WCD{ \|x}, \WCD{ \|y} and \WCD{ \|z} related B factors and the $F_{\nu}$ }\7
$\>{jt1}=\>{i1max}+\>{i2max}-\WO{1}$\6
$\>{jt2}=\>{j1max}+\>{j2max}-\WO{1}$\6
$\>{jt3}=\>{k1max}+\>{k2max}-\WO{1}$\7
\&{do} $\>{ii}=\WO{1},\ \39\>{jt1}$\1\6
\&{do} $\>{jj}=\WO{1},\ \39\>{jt2}$\1\6
\&{do} $\>{kk}=\WO{1},\ \39\>{jt3}$\1\6
$\>{nu}=\>{ii}+\>{jj}+\>{kk}-\WO{2}$\6
$\&{if}\,(\>{xyorz}\WI\WO{0})$\1\6
$\>{nu}=\>{nu}+\WO{1}$\2\7
\WC{ \WCD{ \>{eribit}} is a  repulsion integral over primitive GTFs }\7
$\>{eribit}=\>{eribit}+\|g(\>{nu})\ast\>{bbx}(\>{ii})\ast\>{bby}(\>{jj})\ast%
\>{bbz}(\>{kk})$\2\7
\&{end} \&{do}\2\6
\&{end} \&{do}\2\6
\&{end} \&{do}\7
\WC{ Now accumulate the primitive integrals into the integral
over contracted GTFs including some constant factors               and
contraction coefficients    }\7
$\>{generi}=\>{generi}+\>{prefa}\ast\>{prefc}\ast\>{eribit}\ast\@{dsqrt}(\|w)$%
\2\7
\&{end} \&{do}\5
\WC{ end of "l" contraction loop }\2\6
\&{end} \&{do}\5
\WC{ end of "k" contraction loop }\2\6
\&{end} \&{do}\5
\WC{ end of "j" contraction loop }\2\6
\&{end} \&{do}\5
\WC{ end of "i" contraction loop }\7
$\&{if}\,(\>{xyorz}\WS\WO{0})$\1\6
$\>{generi}=\>{generi}\ast\>{two}$\2\6
\&{return}\2\6
\&{end}\WY\Wendc
\fi % End of section 11 (sect. 4, p. 8)

\WM12. Here are the local declarations (workspoace {\em etc.})
for the two-electron main function \WCD{ \>{generi}}.

\WY\WP\4\4\WX{\M{12}}generi local declarations\X \X${}\WSQ{}$\6
\&{double} \&{precision}~\1\|p$(\WO{3}),$ \|q$(\WO{3}),$ \>{ppx}$(\WO{20}),$ %
\>{ppy}$(\WO{20}),$ \>{ppz}$(\WO{20})$\2\6
\&{double} \&{precision}~\1\>{bbx}$(\WO{20}),$ \>{bby}$(\WO{20}),$ \>{bbz}$(%
\WO{20}),$ \>{sf}$(\WO{10},\ \39\WO{6})$\2\6
\&{double} \&{precision}~\1\>{xleft}$(\WO{5},\ \39\WO{10}),$ \>{yleft}$(\WO{5},%
\ \39\WO{10}),$ \>{zleft}$(\WO{5},\ \39\WO{10})$\2\6
\&{double} \&{precision}~\1\|r$(\WO{3}),$ \>{fact}$(\WO{20}),$ \|g$(\WO{50})$\2%
\6
\&{data} ~\1\>{zero}$,$ \>{one}$,$ \>{two}$,$ \>{half}${/}\WO{0.0\^D00},\ \39%
\WO{1.0\^D00},\ \39\WO{2.0\^D00},\ \39\WO{0.5\^D00}{/}$\2\6
\&{data} ~\1\>{pi}${/}\WO{3.141592653589\^D00}{/}$\2\WY\Wendc
\WU section~\M{11}.
\fi % End of section 12 (sect. 4.1, p. 10a)

\WM13. These numbers are the first 20 factorials \WCD{ $\>{fact}(\|i)$}
contains $(i-1)!$.

\WY\WP\4\4\WX{\M{13}}Factorials\X \X${}\WSQ{}$\6
\&{data} ~\1\>{fact}${/}\WO{1.0\^D00},\ \39\WO{1.0\^D00},\ \39\WO{2.0\^D00},\ %
\39\WO{6.0\^D00},\ \39\WO{24.0\^D00},\ \39\WO{120.0\^D00},\ \39\WO{720.0\^D00},%
\ \39\WO{5040.0\^D00},\ \39\WO{40320.0\^D00},\ \39\WO{362880.0\^D00},\ \39%
\WO{3628800.0\^D00},\ \39\WO{39916800.0\^D00},\ \39\WO{479001600.0\^D00},\ \39%
\WO{6227020800.0\^D00},\ \39\WO{6}\ast\WO{0.0\^D00}{/}$\2\WY\Wendc
\WU sections~\M{1}, \M{11}, and~\M{20}.
\fi % End of section 13 (sect. 4.2, p. 10b)

\WM14. This tedious code extracts the (integer) setup data; the powers of
$x$, $y$ and $z$ in each of the Cartesian monomials of
each of the four basis functions and the limits of the contraction
in each case.

\WY\WP\4\4\WX{\M{14}}Two-electron Integer Setup\X \X${}\WSQ{}$\6
$\>{ityp}=\>{ntype}(\|i)$\6
$\>{jtyp}=\>{ntype}(\|j)$\6
$\>{ktyp}=\>{ntype}(\|k)$\6
$\>{ltyp}=\>{ntype}(\|l)$\6
$\>{l1}=\>{nr}(\>{ityp},\ \39\WO{1})$\6
$\>{m1}=\>{nr}(\>{ityp},\ \39\WO{2})$\6
$\>{n1}=\>{nr}(\>{ityp},\ \39\WO{3})$\6
$\>{l2}=\>{nr}(\>{jtyp},\ \39\WO{1})$\6
$\>{m2}=\>{nr}(\>{jtyp},\ \39\WO{2})$\6
$\>{n2}=\>{nr}(\>{jtyp},\ \39\WO{3})$\6
$\>{l3}=\>{nr}(\>{ktyp},\ \39\WO{1})$\6
$\>{m3}=\>{nr}(\>{ktyp},\ \39\WO{2})$\6
$\>{n3}=\>{nr}(\>{ktyp},\ \39\WO{3})$\6
$\>{l4}=\>{nr}(\>{ltyp},\ \39\WO{1})$\6
$\>{m4}=\>{nr}(\>{ltyp},\ \39\WO{2})$\6
$\>{n4}=\>{nr}(\>{ltyp},\ \39\WO{3})$\6
$\>{is}=\>{nfirst}(\|i)$\6
$\>{il}=\>{nlast}(\|i)$\6
$\>{js}=\>{nfirst}(\|j)$\6
$\>{jl}=\>{nlast}(\|j)$\6
$\>{ks}=\>{nfirst}(\|k)$\6
$\>{kl}=\>{nlast}(\|k)$\6
$\>{ls}=\>{nfirst}(\|l)$\6
$\>{ll}=\>{nlast}(\|l)$\WY\Wendc
\WU section~\M{11}.
\fi % End of section 14 (sect. 4.3, p. 10c)

\WM15. Use the Gaussian Product Theorem to find the position vector
$\vec{P}$, of the product of the two Gaussian exponential factors
of the basis functions for electron 1.

\WY\WP\4\4\WX{\M{15}}Compute PA\X \X${}\WSQ{}$\6
$\>{aexp}=\>{eta}(\>{irun},\ \39\WO{4});$\6
$\>{anorm}=\>{eta}(\>{irun},\ \39\WO{5})$\6
$\>{bexp}=\>{eta}(\>{jrun},\ \39\WO{4});$\6
$\>{bnorm}=\>{eta}(\>{jrun},\ \39\WO{5})$\7
\WC{ \WCD{ \>{aexp}} and \WCD{ \>{bexp}} are the primitive GTF exponents for
    GTF \WCD{ \>{irun}} and \WCD{ \>{jrun}}, \WCD{ \>{anorm}} and \WCD{ %
\>{bnorm}} are the        corresponding contraction coefficients bundled up
into        \WCD{ \>{prefa}}  }\7
$\>{t1}=\>{aexp}+\>{bexp};$\6
$\>{deleft}=\>{one}\WSl\>{t1}$\7
$\|p(\WO{1})=(\>{aexp}\ast\>{eta}(\>{irun},\ \39\WO{1})+\>{bexp}\ast\>{eta}(%
\>{jrun},\ \39\WO{1}))\ast\>{deleft}$\6
$\|p(\WO{2})=(\>{aexp}\ast\>{eta}(\>{irun},\ \39\WO{2})+\>{bexp}\ast\>{eta}(%
\>{jrun},\ \39\WO{2}))\ast\>{deleft}$\6
$\|p(\WO{3})=(\>{aexp}\ast\>{eta}(\>{irun},\ \39\WO{3})+\>{bexp}\ast\>{eta}(%
\>{jrun},\ \39\WO{3}))\ast\>{deleft}$\7
$\>{pax}=\|p(\WO{1})-\>{eta}(\>{irun},\ \39\WO{1})$\6
$\>{pay}=\|p(\WO{2})-\>{eta}(\>{irun},\ \39\WO{2})$\6
$\>{paz}=\|p(\WO{3})-\>{eta}(\>{irun},\ \39\WO{3})$\7
$\>{pbx}=\|p(\WO{1})-\>{eta}(\>{jrun},\ \39\WO{1})$\6
$\>{pby}=\|p(\WO{2})-\>{eta}(\>{jrun},\ \39\WO{2})$\6
$\>{pbz}=\|p(\WO{3})-\>{eta}(\>{jrun},\ \39\WO{3})$\7
$\>{prefa}=\@{dexp}({-}\>{aexp}\ast\>{bexp}\ast\>{rAB}\WSl\>{t1})\ast\>{pi}\ast%
\>{anorm}\ast\>{bnorm}\WSl\>{t1}$\WY\Wendc
\WU sections~\M{1} and~\M{11}.
\fi % End of section 15 (sect. 4.4, p. 11)

\WM16. Use the Gaussian Product Theorem to find the position vector
$\vec{Q}$, of the product of the two Gaussian exponential factors
of the basis functions for electron 2.

\WY\WP\4\4\WX{\M{16}}Compute QC\X \X${}\WSQ{}$\6
$\>{cexpp}=\>{eta}(\>{krun},\ \39\WO{4});$\6
$\>{cnorm}=\>{eta}(\>{krun},\ \39\WO{5})$\6
$\>{dexpp}=\>{eta}(\>{lrun},\ \39\WO{4});$\6
$\>{dnorm}=\>{eta}(\>{lrun},\ \39\WO{5})$\7
\WC{ \WCD{ \@{cexp}} and \WCD{ \@{dexp}} are the primitive GTF exponents for
    GTF \WCD{ \>{krun}} and \WCD{ \>{lrun}}, \WCD{ \>{cnorm}} and \WCD{ %
\>{dnorm}} are the        corresponding contraction coefficients bundled up
into        \WCD{ \>{prefc}}  }\7
$\>{t2}=\>{cexpp}+\>{dexpp}$\6
$\>{t2m1}=\>{one}\WSl\>{t2}$\6
$\>{fordel}=\>{t2m1}+\>{deleft}$\7
$\|q(\WO{1})=(\>{cexpp}\ast\>{eta}(\>{krun},\ \39\WO{1})+\>{dexpp}\ast\>{eta}(%
\>{lrun},\ \39\WO{1}))\ast\>{t2m1}$\6
$\|q(\WO{2})=(\>{cexpp}\ast\>{eta}(\>{krun},\ \39\WO{2})+\>{dexpp}\ast\>{eta}(%
\>{lrun},\ \39\WO{2}))\ast\>{t2m1}$\6
$\|q(\WO{3})=(\>{cexpp}\ast\>{eta}(\>{krun},\ \39\WO{3})+\>{dexpp}\ast\>{eta}(%
\>{lrun},\ \39\WO{3}))\ast\>{t2m1}$\7
$\>{qcx}=\|q(\WO{1})-\>{eta}(\>{krun},\ \39\WO{1})$\6
$\>{qcy}=\|q(\WO{2})-\>{eta}(\>{krun},\ \39\WO{2})$\6
$\>{qcz}=\|q(\WO{3})-\>{eta}(\>{krun},\ \39\WO{3})$\7
$\>{qdx}=\|q(\WO{1})-\>{eta}(\>{lrun},\ \39\WO{1})$\6
$\>{qdy}=\|q(\WO{2})-\>{eta}(\>{lrun},\ \39\WO{2})$\6
$\>{qdz}=\|q(\WO{3})-\>{eta}(\>{lrun},\ \39\WO{3})$\7
$\|r(\WO{1})=\|p(\WO{1})-\|q(\WO{1})$\6
$\|r(\WO{2})=\|p(\WO{2})-\|q(\WO{2})$\6
$\|r(\WO{3})=\|p(\WO{3})-\|q(\WO{3})$\7
$\|t=(\|r(\WO{1})\ast\|r(\WO{1})+\|r(\WO{2})\ast\|r(\WO{2})+\|r(\WO{3})\ast\|r(%
\WO{3}))\WSl\>{fordel}$\6
$\>{prefc}=\@{exp}({-}\>{cexpp}\ast\>{dexpp}\ast\>{rCD}\WSl\>{t2})\ast\>{pi}%
\ast\>{cnorm}\ast\>{dnorm}\WSl\>{t2}$\WY\Wendc
\WU section~\M{11}.
\fi % End of section 16 (sect. 4.5, p. 12)

\WM17. The series of terms arising from the expansion of the
Cartesian monomials like $(x - PA)^{\ell_1}(x - PB)^{\ell_2}$ are
computed by first forming the $f_j$ and hence the $\theta$s.

\WY\WP\4\4\WX{\M{17}}Thetas for electron 1\X \X${}\WSQ{}$\6
$\>{i1max}=\>{l1}+\>{l2}+\WO{1}$\6
$\>{j1max}=\>{m1}+\>{m2}+\WO{1}$\6
$\>{k1max}=\>{n1}+\>{n2}+\WO{1}$\7
$\>{mleft}=\>{i1max}+\>{j1max}+\>{k1max}$\7
\&{do} $\|n=\WO{1},\ \39\>{i1max}$\1\6
$\>{sf}(\|n,\ \39\WO{1})=\>{fj}(\>{l1},\ \39\>{l2},\ \39\|n-\WO{1},\ \39%
\>{pax},\ \39\>{pbx})$\2\6
\&{end} \&{do}\7
\&{do} $\|n=\WO{1},\ \39\>{j1max}$\1\6
$\>{sf}(\|n,\ \39\WO{2})=\>{fj}(\>{m1},\ \39\>{m2},\ \39\|n-\WO{1},\ \39%
\>{pay},\ \39\>{pby})$\2\6
\&{end} \&{do}\7
\&{do} $\|n=\WO{1},\ \39\>{k1max}$\1\6
$\>{sf}(\|n,\ \39\WO{3})=\>{fj}(\>{n1},\ \39\>{n2},\ \39\|n-\WO{1},\ \39%
\>{paz},\ \39\>{pbz})$\2\6
\&{end} \&{do}\7
\&{call} $\>{theta}(\>{i1max},\ \39\>{sf},\ \39\WO{1},\ \39\>{fact},\ \39%
\>{t1},\ \39\>{xleft})$\6
\&{call} $\>{theta}(\>{j1max},\ \39\>{sf},\ \39\WO{2},\ \39\>{fact},\ \39%
\>{t1},\ \39\>{yleft})$\6
\&{call} $\>{theta}(\>{k1max},\ \39\>{sf},\ \39\WO{3},\ \39\>{fact},\ \39%
\>{t1},\ \39\>{zleft})$\WY\Wendc
\WU section~\M{11}.
\fi % End of section 17 (sect. 4.6, p. 13a)

\WM18. The series of terms arising from the expansion of the
Cartesian monomials like $(x - QC)^{\ell_3}(x - QD)^{\ell_4}$ are
computed by  forming the $f_j$ and storing them in the array \WCD{ \>{sf}}
for later use by \WCD{ \>{bform}}.

\WY\WP\4\4\WX{\M{18}}fj for electron 2\X \X${}\WSQ{}$\6
$\>{i2max}=\>{l3}+\>{l4}+\WO{1}$\6
$\>{j2max}=\>{m3}+\>{m4}+\WO{1}$\6
$\>{k2max}=\>{n3}+\>{n4}+\WO{1}$\7
$\>{twodel}=\>{half}\ast\>{fordel}$\6
$\>{delta}=\>{half}\ast\>{twodel}$\7
\&{do} $\|n=\WO{1},\ \39\>{i2max}$\1\6
$\>{sf}(\|n,\ \39\WO{4})=\>{fj}(\>{l3},\ \39\>{l4},\ \39\|n-\WO{1},\ \39%
\>{qcx},\ \39\>{qdx})$\2\6
\&{end} \&{do}\7
\&{do} $\|n=\WO{1},\ \39\>{j2max}$\1\6
$\>{sf}(\|n,\ \39\WO{5})=\>{fj}(\>{m3},\ \39\>{m4},\ \39\|n-\WO{1},\ \39%
\>{qcy},\ \39\>{qdy})$\2\6
\&{end} \&{do}\7
\&{do} $\|n=\WO{1},\ \39\>{k2max}$\1\6
$\>{sf}(\|n,\ \39\WO{6})=\>{fj}(\>{n3},\ \39\>{n4},\ \39\|n-\WO{1},\ \39%
\>{qcz},\ \39\>{qdz})$\2\6
\&{end} \&{do}\7
$\|m=\>{mleft}+\>{i2max}+\>{j2max}+\>{k2max}+\WO{1}$\WY\Wendc
\WU section~\M{11}.
\fi % End of section 18 (sect. 4.7, p. 13b)

\WM19. In the central inner loops of the four contractions,
use the previously- computed $\theta$ factors to
form the combined geometrical $B$ factors.

\WY\WP\4\4\WX{\M{19}}Form Bs\X \X${}\WSQ{}$\6
$\>{ppx}(\WO{1})=\>{one};$\6
$\>{bbx}(\WO{1})=\>{zero}$\6
$\>{ppy}(\WO{1})=\>{one};$\6
$\>{bby}(\WO{1})=\>{zero}$\6
$\>{ppz}(\WO{1})=\>{one};$\6
$\>{bbz}(\WO{1})=\>{zero}$\7
$\>{jt1}=\>{i1max}+\>{i2max}$\6
\&{do} $\|n=\WO{2},\ \39\>{jt1}$\1\6
$\>{ppx}(\|n)={-}\>{ppx}(\|n-\WO{1})\ast\|r(\WO{1})$\6
$\>{bbx}(\|n)=\>{zero}$\2\6
\&{end} \&{do}\7
$\>{jt1}=\>{j1max}+\>{j2max}$\6
\&{do} $\|n=\WO{2},\ \39\>{jt1}$\1\6
$\>{ppy}(\|n)={-}\>{ppy}(\|n-\WO{1})\ast\|r(\WO{2})$\6
$\>{bby}(\|n)=\>{zero}$\2\6
\&{end} \&{do}\7
$\>{jt1}=\>{k1max}+\>{k2max}$\6
\&{do} $\|n=\WO{2},\ \39\>{jt1}$\1\6
$\>{ppz}(\|n)={-}\>{ppz}(\|n-\WO{1})\ast\|r(\WO{3})$\6
$\>{bbz}(\|n)=\>{zero}$\2\6
\&{end} \&{do}\7
\&{call} $\>{bform}(\>{i1max},\ \39\>{i2max},\ \39\>{sf},\ \39\WO{1},\ \39%
\>{fact},\ \39\>{xleft},\ \39\>{t2},\ \39\>{delta},\ \39\>{ppx},\ \39\>{bbx},\ %
\39\>{xyorz})$\6
\&{call} $\>{bform}(\>{j1max},\ \39\>{j2max},\ \39\>{sf},\ \39\WO{2},\ \39%
\>{fact},\ \39\>{yleft},\ \39\>{t2},\ \39\>{delta},\ \39\>{ppy},\ \39\>{bby},\ %
\39\>{xyorz})$\6
\&{call} $\>{bform}(\>{k1max},\ \39\>{k2max},\ \39\>{sf},\ \39\WO{3},\ \39%
\>{fact},\ \39\>{zleft},\ \39\>{t2},\ \39\>{delta},\ \39\>{ppz},\ \39\>{bbz},\ %
\39\>{xyorz})$\WY\Wendc
\WU section~\M{11}.
\fi % End of section 19 (sect. 4.8, p. 14)

\WN20.  fj.
This is the function to evaluate the coefficient of $x^j$ in the expansion
of
$$
(x + a)^\ell (x+b)^m
$$
The full expression is
$$
f_j (\ell , m , a, b) = \sum_{k = max (0, j-m)}^{min(j, \ell }
                         { \ell \choose k}{ m \choose {j-k}}
                          a^{\ell - k } b^{m + k - j}
$$
The function must take steps to do the right thing for
$0.0^0$ when it occurs.

\WY\WP \Wunnamed{code}{integral.f}%
\7
\&{double} \&{precision} \&{function} \1$\>{fj}(\|l,\ \39\|m,\ \39\|j,\ \39\|a,%
\ \39\|b)$\2\1\7
\&{implicit} \1\&{double} \&{precision}$\,(\|a-\|h,\ \39\|o-\|z)$\2\6
\&{integer}~\1\|l$,$ \|m$,$ \|j\2\6
\&{double} \&{precision}~\1\|a$,$ \|b\2\7
\&{double} \&{precision}~\1\>{sum}$,$ \>{term}$,$ \>{aa}$,$ \>{bb}\2\6
\&{integer}~\1\|i$,$ \>{imax}$,$ \>{imin}\2\6
\&{double} \&{precision}~\1\>{fact}$(\WO{20})$\2\7
\WX{\M{13}}Factorials\X \X\7
$\>{imax}=\@{min}(\|j,\ \39\|l)$\6
$\>{imin}=\@{max}(\WO{0},\ \39\|j-\|m)$\7
$\>{sum}=\WO{0.0\^D00}$\6
\&{do} $\|i=\>{imin},\ \39\>{imax}$\1\7
$\>{term}=\>{fact}(\|l+\WO{1})\ast\>{fact}(\|m+\WO{1})\WSl(\>{fact}(\|i+\WO{1})%
\ast\>{fact}(\|j-\|i+\WO{1}))$\6
$\>{term}=\>{term}\WSl(\>{fact}(\|l-\|i+\WO{1})\ast\>{fact}(\|m-\|j+\|i+%
\WO{1}))$\6
$\>{aa}=\WO{1.0\^D00};$\6
$\>{bb}=\WO{1.0\^D00}$\6
$\&{if}\,((\|l-\|i)\WI\WO{0})$\1\6
$\>{aa}=\|a\WEE{(\|l-\|i)}$\2\7
$\&{if}\,((\|m+\|i-\|j)\WI\WO{0})$\1\6
$\>{bb}=\|b\WEE{(\|m+\|i-\|j)}$\2\7
$\>{term}=\>{term}\ast\>{aa}\ast\>{bb}$\6
$\>{sum}=\>{sum}+\>{term}$\2\7
\&{end} \&{do}\7
$\>{fj}=\>{sum}$\7
\&{return}\2\6
\&{end}\WY\Wendc
\fi % End of section 20 (sect. 5, p. 15)

\WN21.  theta.
Computation of all the $\theta$ factors required from one
basis-function product; any one of them is given by
$$
\theta (j , \ell_1 , \ell_2 , a, b,  r , \gamma )
 = f_{j} (\ell_1 , \ell_2 , a, b) {{ j! \gamma^{r - j}} \over
         { r! (j - 2r)!}}
$$
The $f_j$ are computed in the body of \WCD{ \>{generi}} and passed to this
routine in \WCD{ \>{sf}}, the particular ones to use are in \WCD{ $\>{sf}(\ast,%
\ \>{isf})$}.
They are stored in \WCD{ \>{xleft}}, \WCD{ \>{yleft}} and \WCD{ \>{zleft}}
because they
are associated with electron 1 (the left-hand factor in the integrand
as it is usually written $(ij,k\ell)$).

\WY\WP \Wunnamed{code}{integral.f}%
\7
\&{subroutine} \1$\>{theta}(\>{i1max},\ \39\>{sf},\ \39\>{isf},\ \39\>{fact},\ %
\39\>{t1},\ \39\>{xleft})$\2\1\7
\&{implicit} \1\&{double} \&{precision}$\,(\|a-\|h,\ \39\|o-\|z)$\2\6
\&{integer}~\1\>{i1max}$,$ \>{isf}\2\6
\&{double} \&{precision}~\1\>{t1}\2\6
\&{double} \&{precision}~\1\>{sf}$(\WO{10},\ \39\ast),$ \>{fact}$(\ast),$ %
\>{xleft}$(\WO{5},\ \39\ast)$\2\7
\&{integer}~\1\>{i1}$,$ \>{ir1}$,$ \>{ir1max}$,$ \>{jt2}\2\6
\&{double} \&{precision}~\1\>{zero}$,$ \>{sfab}$,$ \>{bbb}\2\7
\&{data} ~\1\>{zero}${/}\WO{0.0\^D00}{/}$\2\7
\&{do} $\>{i1}=\WO{1},\ \39\WO{10}$\1\6
\&{do} $\>{ir1}=\WO{1},\ \39\WO{5}$\1\6
$\>{xleft}(\>{ir1},\ \39\>{i1})=\>{zero}$\2\6
\&{end} \&{do}\2\6
\&{end} \&{do}\7
\&{do} $\WO{100}$ $\>{i1}=\WO{1},\ \39\>{i1max}$\1\6
$\>{sfab}=\>{sf}(\>{i1},\ \39\>{isf})$\7
$\&{if}\,(\>{sfab}\WS\>{zero})$\1\6
\&{go} \&{to} $\WO{100}$\2\7
$\>{ir1max}=(\>{i1}-\WO{1})\WSl\WO{2}+\WO{1}$\6
$\>{bbb}=\>{sfab}\ast\>{fact}(\>{i1})\WSl\>{t1}\WEE{(\>{i1}-\WO{1})}$\6
\&{do} $\>{ir1}=\WO{1},\ \39\>{ir1max}$\1\6
$\>{jt2}=\>{i1}+\WO{2}-\>{ir1}-\>{ir1}$\6
$\>{xleft}(\>{ir1},\ \39\>{i1})=\>{bbb}\ast(\>{t1}\WEE{(\>{ir1}-\WO{1})})\WSl(%
\>{fact}(\>{ir1})\ast\>{fact}(\>{jt2}))$\2\6
\&{end} \&{do}\2\7
\Wlbl{\WO{100}\Colon\ }\&{continue}\7
\&{return}\2\6
\&{end}\WY\Wendc
\fi % End of section 21 (sect. 6, p. 16)

\WN22.  bform.
Use the pre-computed $f_j$ and $\theta$ to form the
\lq\lq $B$ \rq\rq\ factors, the final geometrical expansion
coefficients arising from the products of Cartesian monomials. Any one
of them is given by
\begin{eqnarray*}
  B_{\ell , \ell' , r_1 , r_2 , i } (\ell_1 , \ell_2 , \vec{A}_x ,
   \vec{B}_x , \vec{P}_x , \gamma_1 ;\ell_3 , \ell_4 , \vec{C}_x ,
   \vec{D}_x , \vec{Q}_x , \gamma_2 )  \\
 =  (-1)^{\ell'}
 \theta (\ell , \ell_1 , \ell_2 , \vec{PA}_x, \vec{PB}_x, r, \gamma_1 )
\theta (\ell' , \ell_3 , \ell_4 , \vec{QC}_x, \vec{QD}_x, r', \gamma_2 ) \\
  \times  \frac{(-1)^i (2\delta)^{2(r + r')}(\ell + \ell' -2r-2r')!
     \delta^i \vec{p}_x^{\ell + \ell' -2(r + r' +i)} }
     { (4\delta)^{\ell + \ell'} i! [\ell + \ell' -2(r + r'+i)]!}
\end{eqnarray*}


\WY\WP \Wunnamed{code}{integral.f}%
\7
\&{subroutine} \1$\>{bform}(\>{i1max},\ \39\>{i2max},\ \39\>{sf},\ \39\>{isf},\
\39\>{fact},\ \39\>{xleft},\ \39\>{t2},\ \39\>{delta},\ \39\>{ppx},\ \39%
\>{bbx},\ \39\>{xyorz})$\2\1\7
\&{implicit} \1\&{double} \&{precision}$\,(\|a-\|h,\ \39\|o-\|z)$\2\6
\&{integer}~\1\>{i1max}$,$ \>{i2max}$,$ \>{isf}\2\6
\&{double} \&{precision}~\1\>{fact}$(\ast),$ \>{sf}$(\WO{10},\ \39\ast),$ %
\>{xleft}$(\WO{5},\ \39\ast),$ \>{bbx}$(\ast),$ \>{ppx}$(\WO{20})$\2\6
\&{double} \&{precision}~\1\>{delta}\2\6
\&{integer}~\1\>{xyorz}$,$ \>{itab}\2\7
\&{double} \&{precision}~\1\>{zero}$,$ \>{one}$,$ \>{two}$,$ \>{twodel}$,$ %
\>{fordel}$,$ \>{sfab}$,$ \>{sfcd}\2\6
\&{double} \&{precision}~\1\>{bbc}$,$ \>{bbd}$,$ \>{bbe}$,$ \>{bbf}$,$ %
\>{bbg}$,$ \>{ppqq}\2\6
\&{integer}~\1\>{i1}$,$ \>{i2}$,$ \>{jt1}$,$ \>{jt2}$,$ \>{ir1max}$,$ %
\>{ir2max}\2\6
\&{data} ~\1\>{zero}$,$ \>{one}$,$ \>{two}${/}\WO{0.0\^D00},\ \39\WO{1.0\^D00},%
\ \39\WO{2.0\^D00}{/}$\2\7
$\>{itab}=\WO{0}$\7
$\&{if}\,(\>{xyorz}\WS\>{isf})$\1\6
$\>{itab}=\WO{1}$\2\7
$\>{twodel}=\>{two}\ast\>{delta};$\6
$\>{fordel}=\>{two}\ast\>{twodel}$\7
\&{do} $\WO{200}$ $\>{i1}=\WO{1},\ \39\>{i1max}$\1\7
$\>{sfab}=\>{sf}(\>{i1},\ \39\>{isf})$\6
$\&{if}\,(\>{sfab}\WS\>{zero})$\1\6
\&{go} \&{to} $\WO{200}$\2\6
$\>{ir1max}=(\>{i1}-\WO{1})\WSl\WO{2}+\WO{1}$\7
\&{do} $\WO{210}$ $\>{i2}=\WO{1},\ \39\>{i2max}$\1\7
$\>{sfcd}=\>{sf}(\>{i2},\ \39\>{isf}+\WO{3})$\6
$\&{if}\,(\>{sfcd}\WS\>{zero})$\1\6
\&{go} \&{to} $\WO{210}$\2\6
$\>{jt1}=\>{i1}+\>{i2}-\WO{2}$\6
$\>{ir2max}=(\>{i2}-\WO{1})\WSl\WO{2}+\WO{1}$\6
$\>{bbc}=(({-}\>{one})\WEE{(\>{i2}-\WO{1})})\ast\>{sfcd}\ast\>{fact}(\>{i2})%
\WSl(\>{t2}\WEE{(\>{i2}-\WO{1})}\ast(\>{fordel}\WEE{\>{jt1}}))$\7
\&{do} $\WO{220}$ $\>{ir1}=\WO{1},\ \39\>{ir1max}$\1\7
$\>{jt2}=\>{i1}+\WO{2}-\>{ir1}-\>{ir1}$\6
$\>{bbd}=\>{bbc}\ast\>{xleft}(\>{ir1},\ \39\>{i1})$\6
$\&{if}\,(\>{bbd}\WS\>{zero})$\1\6
\&{go} \&{to} $\WO{220}$\2\7
\&{do} $\WO{230}$ $\>{ir2}=\WO{1},\ \39\>{ir2max}$\1\7
$\>{jt3}=\>{i2}+\WO{2}-\>{ir2}-\>{ir2}$\6
$\>{jt4}=\>{jt2}+\>{jt3}-\WO{2}$\6
$\>{irumax}=(\>{jt4}+\>{itab})\WSl\WO{2}+\WO{1}$\6
$\>{jt1}=\>{ir1}+\>{ir1}+\>{ir2}+\>{ir2}-\WO{4}$\7
$\>{bbe}=\>{bbd}\ast(\>{t2}\WEE{(\>{ir2}-\WO{1})})\ast(\>{twodel}\WEE{\>{jt1}})%
\ast\>{fact}(\>{jt4}+\WO{1})\WSl(\>{fact}(\>{ir2})\ast\>{fact}(\>{jt3}))$\7
\&{do} $\WO{240}$ $\>{iru}=\WO{1},\ \39\>{irumax}$\1\7
$\>{jt5}=\>{jt4}-\>{iru}-\>{iru}+\WO{3}$\6
$\>{ppqq}=\>{ppx}(\>{jt5})$\6
$\&{if}\,(\>{ppqq}\WS\>{zero})$\1\6
\&{go} \&{to} $\WO{240}$\2\7
$\>{bbf}=\>{bbe}\ast(({-}\>{delta})\WEE{(\>{iru}-\WO{1})})\ast\>{ppqq}\WSl(%
\>{fact}(\>{iru})\ast\>{fact}(\>{jt5}))$\7
$\>{bbg}=\>{one}$\7
$\&{if}\,(\>{itab}\WS\WO{1})$ \&{then}\1\7
$\>{bbg}=\@{dfloat}(\>{jt4}+\WO{1})\ast\>{ppx}(\WO{2})\WSl(\>{delta}\ast%
\@{dfloat}(\>{jt5}))$\2\7
\&{end} \&{if}\7
$\>{bbf}=\>{bbf}\ast\>{bbg}$\6
$\>{nux}=\>{jt4}-\>{iru}+\WO{2}$\6
$\>{bbx}(\>{nux})=\>{bbx}(\>{nux})+\>{bbf}$\2\7
\Wlbl{\WO{240}\Colon\ }\&{continue}\2\6
\Wlbl{\WO{230}\Colon\ }\&{continue}\2\6
\Wlbl{\WO{220}\Colon\ }\&{continue}\2\6
\Wlbl{\WO{210}\Colon\ }\&{continue}\2\6
\Wlbl{\WO{200}\Colon\ }\&{continue}\7
\&{return}\2\6
\&{end}\WY\Wendc
\fi % End of section 22 (sect. 7, p. 17)

\WN23.  auxg.
Find the maximum value of $F_\nu$ required, use \WCD{ \>{fmch}} to
compute it and obtain all the lower $F_\nu$ by downward recursion.
$$
F_{\nu-1}(x) = {{\exp(-x) + 2 x F_\nu (x) } \over {2 \nu -1 }}
$$

\WY\WP \Wunnamed{code}{integral.f}%
\7
\&{subroutine} \1$\>{auxg}(\>{mmax},\ \39\|x,\ \39\|g)$\2\1\7
\&{implicit} \1\&{double} \&{precision}$\,(\|a-\|h,\ \39\|o-\|z)$\2\6
\&{integer}~\1\>{mmax}\2\6
\&{double} \&{precision}~\1\|x$,$ \|g$(\ast)$\2\7
\&{double} \&{precision}~\1\>{fmch}\2\7
\&{double} \&{precision}~\1\>{two}$,$ \|y\2\6
\&{integer}~\1\>{mp1mx}$,$ \>{mp1}$,$ \>{md}$,$ \>{mdm}\2\6
\&{data} ~\1\>{two}${/}\WO{2.0\^D00}{/}$\2\7
$\|y=\@{dexp}({-}\|x)$\6
$\>{mp1mx}=\>{mmax}+\WO{1}$\6
$\|g(\>{mp1mx})=\>{fmch}(\>{mmax},\ \39\|x,\ \39\|y)$\6
$\&{if}\,(\>{mmax}<\WO{1})$\1\6
\&{go} \&{to} $\WO{303}$\2\5
\WC{ just in case!  }\7
\WC{ Now do the recursion  downwards }\7
\&{do} $\>{mp1}=\WO{1},\ \39\>{mmax}$\1\7
$\>{md}=\>{mp1mx}-\>{mp1}$\6
$\>{mdm}=\>{md}-\WO{1}$\6
$\|g(\>{md})=(\>{two}\ast\|x\ast\|g(\>{md}+\WO{1})+\|y)\WSl\@{dfloat}(\WO{2}%
\ast\>{mdm}+\WO{1})$\2\7
\&{end} \&{do}\7
\Wlbl{\WO{303}\Colon\ }\&{return}\2\6
\&{end}\WY\Wendc
\fi % End of section 23 (sect. 8, p. 19)

\WN24.  fmch. This code is for the oldest and most general
and reliable of the methods of computing
\begin{equation}
 F_\nu (x) = \int_0^1 t^{2 \nu} \exp (-x t^2) dt
\end{equation}
One of two possible series expansions is used depending on the value of x.

For \WCD{ $\|x\WL\WO{10}$} (Small \WCD{ \|x} Case) the (potentially) infinite
series
\begin{equation}
 F_\nu (x) = \frac{1}{2} \exp(-x) \sum_{i=0}^{\infty}
   \frac{\Gamma (\nu + \frac{1}{2} ) }
   {\Gamma (\nu + i + \frac{3}{2})} x^i
\end{equation}
is used.

The series is truncated when the value of terms falls below $10^{-8}$.
However, if the series seems to be becoming unreasonably long before
this condition is reached (more than 50 terms), the evaluation is stopped
and the function aborted with an error message on \WCD{ \WUC{ERROR\_OUTPUT%
\_UNIT}}.

If \WCD{ $\|x>\WO{10}$} (Large \WCD{ \|x} Case) a different series expansion is
used:
%
\begin{equation}
 F_\nu(x) = \frac{\Gamma(\nu + \frac{1}{2})}{2x^{\nu + \frac{1}{2}}}
         - \frac{1}{2} \exp(-x) \sum_{i=0}^{\infty}
           \frac{\Gamma(\nu + \frac{1}{2})}{\Gamma(\nu- i + \frac{3}{2})}
           x^{-i}
\end{equation}
%
This series, in fact, diverges but it diverges so slowly that the error
obtained in truncating
it is always less than the last term in the truncated series. Thus,
Thus, to obtain a value of the function to the same accuracy as the other
series,
the expansion is terminated when the last term is less than the same criterion
($10^{-8}$).

It can be shown that the minimum term is always for \WCD{ \|i} close to
$\nu + x$, thus ifthe terms for this value of \WCD{ \|i} are not below the
criterion,
the series expansion is abandoned, a message output on \WCD{ \WUC{ERROR\_OUTPUT%
\_UNIT}}
and the function aborted.

The third argument, \WCD{ \|y}, is $exp(-x)$, since it is assumed that this
function
will only be used {\it once} to evaluate the function $F_\nu(x)$ for the
maximum value
of $\nu$ required and other values will be obtained by downward recursion of
the form
%
\begin{equation}
 F_{\nu-1}(x) = \frac{\exp(-x) + 2xF_\nu(x)}{2\nu-1}
\end{equation}
%
which also requires the value of $\exp(-x)$ to be available.
%

\ \\ \ \\
\begin{minipage}{4.5in}
\ \\
\begin{description}
\item[NAME] \         \\
 fmch

\item[SYNOPSIS] \     \\
 {\tt double precision function fmch(nu,x,y) \\
   \ \\
  implicit double precision (a-h,o-z) \\
  double precision x, y \\
  integer nu \\
 }

\item[DESCRIPTION] \  \\
 Computes
\[
  F_\nu (x) = \int_0^1 t^{2\nu} e^{-x t^2} dt
\]
given $\nu$ and $x$. It is used in the evaluation of GTF
nuclear attraction and electron-repulsion integrals.

\item[ARGUMENTS] \    \\
\begin{description}
\item[nu] Input: The value of $\nu$ in the explicit formula above ({\tt
integer})
\item[x] Input: $x$ in the formula ({\tt double precision})
\item[y] Input: $\exp(-x)$, assumed to be available.
\end{description}

\item[DIAGNOSTICS] \  \\
If the relevant series of expansion used do not converge to a tolerance
of $10^{-8}$, an error message is printed on standard output and the
computation
aborted.
\end{description}
\ \\ \ \\
\end{minipage}
\ \\ \ \\


\WY\WP \Wunnamed{code}{integral.f}%
 \&{double} \&{precision} \&{function} \1$%
\>{fmch}(\>{nu},\ \39\|x,\ \39\|y)$\2\1\6
\WX{\M{25}}Declarations\X \X\5
\WC{ First, make the variable declarations }\6
\WX{\M{26}}Internal Declarations\X \X\6
$\|m=\>{nu}$\6
$\|a=\@{dfloat}(\|m)$\6
$\&{if}\,(\|x\WL\>{ten})$ \&{then}\1\6
\WX{\M{27}}Small x Case\X \X\2\6
\&{else}\1\6
\WX{\M{28}}Large x Case\X \X\2\6
\&{end} \&{if}\2\6
\&{end}\WY\Wendc
\fi % End of section 24 (sect. 9, p. 20)

\WM25. Here are the declarations and \WCD{  \&{data}}  statements which are ...

\WY\WP\4\4\WX{\M{25}}Declarations\X \X${}\WSQ{}$\6
\&{implicit} \1\&{double} \&{precision}$\,(\|a-\|h,\ \39\|o-\|z)$\2\6
\&{double} \&{precision}~\1\|x$,$ \|y\2\6
\&{integer}~\1\>{nu}\2\Wendc
\WU section~\M{24}.
\fi % End of section 25 (sect. 9.1, p. 22a)

\WM26.

\WY\WP\4\4\WX{\M{26}}Internal Declarations\X \X${}\WSQ{}$\6
\&{double} \&{precision}~\1\>{ten}$,$ \>{half}$,$ \>{one}$,$ \>{zero}$,$ %
\>{rootpi4}$,$ \>{xd}$,$ \>{crit}\2\6
\&{double} \&{precision}~\1\>{term}$,$ \>{partialsum}\2\6
\&{integer}~\1\|m$,$ \|i$,$ \>{numberofterms}$,$ \>{maxone}$,$ \>{maxtwo}\2\6
\&{data} ~\1\>{zero}$,$ \>{half}$,$ \>{one}$,$ \>{rootpi4}$,$ \>{ten}${/}%
\WO{0.0\^D00},\ \39\WO{0.5\^D00},\ \39\WO{1.0\^D00},\ \39\WO{0.88622692\^D00},\
\39\WO{10.0\^D00}{/}$\2\5
\WC{ \WCD{ \>{crit}} is required accuracy of the series expansion }\6
\&{data} ~\1\>{crit}${/}\WO{1.0\^D-08}{/}$\2\5
\WC{ \WCD{ \>{maxone}} }\6
\&{data} ~\1\>{maxone}${/}\WO{50}{/},$ \>{maxtwo}${/}\WO{200}{/}$\2\Wendc
\WU section~\M{24}.
\fi % End of section 26 (sect. 9.2, p. 22b)

\WM27.

\WY\WP\4\4\WX{\M{27}}Small x Case\X \X${}\WSQ{}$\6
$\|a=\|a+\>{half}$\6
$\>{term}=\>{one}\WSl\|a$\6
$\>{partialsum}=\>{term}$\6
\&{do} $\|i=\WO{2},\ \39\>{maxone}$\1\6
$\|a=\|a+\>{one}$\6
$\>{term}=\>{term}\ast\|x\WSl\|a$\6
$\>{partialsum}=\>{partialsum}+\>{term}$\6
$\&{if}\,(\>{term}\WSl\>{partialsum}<\>{crit})$\1\6
\&{go} \&{to} $\WO{111}$\2\2\6
\&{end} \&{do}\6
\Wlbl{\WO{111}\Colon\ }\&{continue}\6
$\&{if}\,(\|i\WS\>{maxone})$ \&{then}\1\6
$\&{write}\,(\WUC{ERROR\_OUTPUT\_UNIT},\ \39\WO{200})$ \6
\Wlbl{\WO{200}\Colon\ }$\&{format}\,(\.{'i\ >\ 50\ in\ fmch'})$ \6
\WUC{STOP}\2\6
\&{end} \&{if}\6
$\>{fmch}=\>{half}\ast\>{partialsum}\ast\|y$\6
\&{return}\Wendc
\WU section~\M{24}.
\fi % End of section 27 (sect. 9.3, p. 22c)

\WM28.

\WY\WP\4\4\WX{\M{28}}Large x Case\X \X${}\WSQ{}$\6
$\|b=\|a+\>{half}$\6
$\|a=\|a-\>{half}$\6
$\>{xd}=\>{one}\WSl\|x$\6
$\>{approx}=\>{rootpi4}\ast\@{dsqrt}(\>{xd})\ast\>{xd}\WEE{\|m}$\6
$\&{if}\,(\|m>\WO{0})$ \&{then}\1\6
\&{do} $\|i=\WO{1},\ \39\|m$\1\6
$\|b=\|b-\>{one}$\6
$\>{approx}=\>{approx}\ast\|b$\2\6
\&{end} \&{do}\2\6
\&{end} \&{if}\6
$\>{fimult}=\>{half}\ast\|y\ast\>{xd}$\6
$\>{partialsum}=\>{zero}$\7
$\&{if}\,(\>{fimult}\WS\>{zero})$ \&{then}\1\6
$\>{fmch}=\>{approx}$\6
\&{return}\2\6
\&{end} \&{if}\7
$\>{fiprop}=\>{fimult}\WSl\>{approx}$\6
$\>{term}=\>{one}$\6
$\>{partialsum}=\>{term}$\6
$\>{numberofterms}=\>{maxtwo}$\6
\&{do} $\|i=\WO{2},\ \39\>{numberofterms}$\1\6
$\>{term}=\>{term}\ast\|a\ast\>{xd}$\6
$\>{partialsum}=\>{partialsum}+\>{term}$\6
$\&{if}\,(\@{dabs}(\>{term}\ast\>{fiprop}\WSl\>{partialsum})\WL\>{crit})$ %
\&{then}\1\6
$\>{fmch}=\>{approx}-\>{fimult}\ast\>{partialsum}$\6
\&{return}\2\6
\&{end} \&{if}\6
$\|a=\|a-\>{one}$\2\6
\&{end} \&{do}\6
$\&{write}\,(\WUC{ERROR\_OUTPUT\_UNIT},\ \39\WO{201})$ \6
\Wlbl{\WO{201}\Colon\ }$\&{format}\,(\.{'\ numberofterms\ reached\ in\0\
fmch'})$ \6
\WUC{STOP}\WY\Wendc
\WU section~\M{24}.
\fi % End of section 28 (sect. 9.4, p. 23a)

\WM29.

\fi % End of section 29 (sect. 9.5, p. 23b)

\WN30.  INDEX.



\fi % End of section 30 (sect. 10, p. 24)

\input INDEX.tex
\input MODULES.tex

\Winfo{"fweave integral.web"}  {"integral.web"} {(none)}
 {\Fortran}


\Wcon{30}
\FWEBend
