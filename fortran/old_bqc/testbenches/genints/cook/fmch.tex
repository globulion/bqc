% FWEAVE v1.62 (September 23, 1998)

% --- FWEB's macro package ---
\input fwebmac.sty

% --- Initialization parameters from FWEB's style file `fweb.sty' ---
\Wbegin[;]
  % #1 --- [LaTeX.class.options;LaTeX.package.options]
{article;}
  % #2 --- {LaTeX.class;LaTeX.package}
{1em}
  % #3 --- {indent.TeX}
{1em}
  % #4 --- {indent.code}
{CONTENTS.tex}
  % #5 --- {contents.TeX}
{ % #6 ---
 {\&\WRS}
  % #1 --- {{format.reserved}{format.RESERVED}}
 {\|}
  % #2 --- {format.short_id}
 {\>\WUC}
  % #3 --- {{format.id}{format.ID}}
 {\>\WUC}
  % #4 --- {{format.outer_macro}{format.OUTER_MACRO}}
 {\>\WUC}
  % #5 --- {{format.WEB_macro}{format.WEB_MACRO}}
 {\@}
  % #6 --- {format.intrinsic}
 {\.\.}
  % #7 --- {{format.keyword}{format.KEYWORD}}
 {\.}
  % #8 --- {format.typewriter}
 {}
  % #9 --- (For future use)
}
{\M}
  % #7 --- {encap.prefix}
{;}
  % #8 --- {doc.preamble;doc.postamble}
{INDEX}
  % #9 --- {index.name}


% --- Beginning of user's limbo section ---




% --- Limbo text from style-file parameter `limbo.end' ---
\FWEBtoc

\WN1.  fmch. This  code is for the oldest and most general
and reliable of the methods of computing
\[
 F_\nu (x)  = \int_0^1 t^{2 \nu} \exp (-x t^2) dt
\]
One of two possible series expansions is used depending on the
value of \WCD{ \|x}.

For \WCD{ $\|x\WL\WO{10}$} (Small \WCD{ \|x} Case) the (potentially) infinite
series
\[
F_\nu (x)  = \frac{1}{2} \exp(-x) \sum_{i=0}^{\infty}
   \frac{\Gamma (\nu + \frac{1}{2} )}
   {\Gamma (\nu + i + \frac{3}{2} ) }
    x^i
\]
is used.

The series is truncated when the value of terms falls below
$10^{-8}$.

However, if the series seems to be becoming unreasonably
long before this condition is reached (more than 50 terms), the
evaluation is stopped and the function aborted with an
error message
on \WCD{ \WUC{ERROR\_OUTPUT\_UNIT}}.

If \WCD{ $\|x>\WO{10}$} (Large \WCD{ \|x} Case) a different series
expansion is used:
\[
F_\nu (x)  =  \frac{\Gamma (\nu + \frac{1}{2} )}
          {2x^{\nu + \frac{1}{2} ) }}
- \frac{1}{2} \exp(-x) \sum_{i=0}^{\infty}
   \frac{\Gamma (\nu + \frac{1}{2} }
   {\Gamma (\nu - i + \frac{3}{2} ) }
    x^{-i}
\]
This series, in fact, diverges but it diverges
so slowly that the error
obtained in truncating it is always less than
the last term in the
truncated series. Thus, to obtain a value of the function to the
same accuracy as the other series,
the expansion is terminated when the
last term is less than the same criterion ($10^{-8}$).

It can be shown that the minimum term is always for
\WCD{ \|i} close to $\nu + x$, thus if the terms for
this value of \WCD{ \|i} are not below the criterion,
the series expansion is abandoned, a message output
on \WCD{ \WUC{ERROR\_OUTPUT\_UNIT}}
and the function aborted.

The third argument, \WCD{ \|y}, is $\exp(-x)$, since it
is assumed that this
function will only be used {\it once} to evaluate the function
$F_\nu (x) $ for the maximum value
of $\nu$ required and other values of
$F_\nu (x)$ will be obtained by downward recursion:
\[
F_{\nu-1}(x) = \frac{\exp(-x) + 2 x F_\nu (x) }{2 \nu -1 }
\]
which also requires the value of $\exp(-x)$ to be available.
%
\  \\ \ \\
%\begin{minipage}{4.3in}
\begin{minipage}{4.3in}
\ \\
\begin{description}
\item[NAME]  \      \\
fmch

\item[SYNOPSIS] \ \\
\begin{verbatim}
double precision function fmch(nu,x,y);

implicit double precision (a-h,o-z);
double precision x, y;
integer nu;

\end{verbatim}
\item[DESCRIPTION] \ \\
Computes
\[
 F_\nu (x)  = \int_0^1 t^{2 \nu} \exp (-x t^2) dt
\]
given $\nu$ and $x$. It is used in the evaluation of GTF
nuclear-attraction and electron-repulsion integrals.
\item[ARGUMENTS:] \  \\
\begin{description}
\item[nu] Input: The value of $\nu$ in the explicit
formula above ({\tt integer}).

\item[x] Input: $x$ in the formula ({\tt double precision}).

\item[y] Input: $\exp(-x)$, assumed to be available.

\end{description}

\item[DIAGNOSTICS] \  \\
If the relevant series expansions used do not converge
to a tolerance of $10^{-8}$, an error message is printed
on standard output and the computation aborted.
\end{description}
\ \\ \ \\
%\end{minipage}
\end{minipage}
\ \\ \ \\

\WY\WP \Wunnamed{defs}{fmch.f}%
\WMd{}\WUC{ARB}\5
$\WO{1}$\Wendd
\WP\WMd{}\WUC{YES}\5
$\WO{0}$\Wendd
\WP\WMd{}\WUC{NO}\5
$\WO{1}$\Wendd
\WP\WMd{}\.{ERR}\5
${-}\WO{1}$\WY\Wendd
\WP\WMd{}\WUC{BYTES\_PER\_INTEGER}\5
$\WO{4}$\Wendd
\WP\WMd{}\WUC{LEAST\_BYTE}\5
$\WO{1}$\WY\Wendd
\WP\WMd{}\WUC{END\_OF\_FILE}\5
${-}\WO{1}$\WY\Wendd
\WP\WMd{}\WUC{NO\_OF\_TYPES}\5
$\WO{20}$\Wendd
\WP\WMd{}\WUC{INT\_BLOCK\_SIZE}\5
$\WO{20}$\WY\Wendd
\WP\WMd{}\WUC{LAST\_BLOCK}\5
$\WO{1}$\Wendd
\WP\WMd{}\WUC{NOT\_LAST\_BLOCK}\5
$\WO{0}$\WY\Wendd
\WP\WMd{}\WUC{ERROR\_OUTPUT\_UNIT}\5
$\WO{6}$\WY\Wendd
\WP\WMd{}\WUC{MAX\_BASIS\_FUNCTIONS}\5
$\WO{255}$\Wendd
\WP\WMd{}\WUC{MAX\_PRIMITIVES}\5
$\WO{1000}$\Wendd
\WP\WMd{}\WUC{MAX\_CENTRES}\5
$\WO{50}$\WY\Wendd
\WP\WMd{}\WUC{MAX\_ITERATIONS}\5
$\WO{60}$\WY\Wendd
\WP\WMd{}\WUC{UHF\_CALCULATION}\5
$\WO{10}$\Wendd
\WP\WMd{}\WUC{CLOSED\_SHELL\_CALCULATION}\5
$\WO{20}$\Wendd
\WY\WP \Wunnamed{code}{fmch.f}%
\7
\&{double} \&{precision} \&{function}~\1\>{fmch}$(\>{nu},\ \39\|x,\ \39\|y);$\2%
\7
\&{implicit} \1\&{double} \&{precision}$\,(\|a-\|h,\ \39\|o-\|z);$\2\6
\&{double} \&{precision}~\1\|x$,$ \|y;\2\6
\&{integer}~\1\>{nu};\2\7
$\{$\7
\WC{ First, make the variable declarations; notice that comments, even in
Fortran, are ``C-style'' }\7
\WX{\M{2}}Internal Declarations\X \X\7
\WC{ Temporary integer and real values for \WCD{ \>{nu} $=$} $\nu$ }\7
\|m$=\>{nu};$\6
$\|a=\@{dble}(\@{float}(\|m));$\7
\WC{ Now do the computation of $F_\nu (x)$ }\7
$\&{if}\,(\|x\WL\>{ten})$\7
$\{$\7
\WC{ Use the expansion for the smaller value of \WCD{ \|x} }\7
\WX{\M{3}}Small x Case\X \X\7
$\}$\7
\&{else}\7
$\{$\7
\WC{ Use the other expansion for the larger value of \WCD{ \|x} }\7
\WX{\M{4}}Large x Case\X \X\7
$\}$\WY\Wendc
\fi % End of section 1 (sect. 1, p. 1)

\WM2. Here are the declarations and \WCD{  \&{data}}  statements which are
entirely internal to \WCD{ \>{fmch}}. Nomenclature is fairly obvious
except possible \WCD{ \>{rootpi4}} which is $\sqrt{(\pi/4)}$.

\WY\WP\4\4\WX{\M{2}}Internal Declarations\X \X${}\WSQ{}$\6
\&{double} \&{precision}~\1\>{ten}$,$ \>{half}$,$ \>{one}$,$ \>{zero}$,$ %
\>{rootpi4}$,$ \>{xd}$,$ \>{crit};\2\6
\&{double} \&{precision}~\1\>{term}$,$ \>{partialsum};\2\6
\&{integer}~\1\|m$,$ \|i$,$ \>{numberofterms}$,$ \>{maxone}$,$ \>{maxtwo};\2\6
\&{data} ~\1\>{zero}$,$ \>{half}$,$ \>{one}$,$ \>{rootpi4}$,$ \>{ten}${/}%
\WO{0.0},\ \39\WO{0.5},\ \39\WO{1.0},\ \39\WO{0.88622692},\ \39\WO{10.0}{/};$\2%
\7
\WC{ \WCD{ \>{crit}} is the required accuracy of the series expansion }\7
\&{data} ~\1\>{crit}${/}\WO{1.0\^E-08}{/};$\2\7
\WC{ \WCD{ \>{maxone}} is the upper limit of the summation in the small-$x$
summation; \WCD{ \>{maxtwo}} is the corresponding limit in the large-$x$
summation }\7
\&{data} ~\1\>{maxone}${/}\WO{50}{/},$ \>{maxtwo}${/}\WO{200}{/};$\2\WY\Wendc
\WU section~\M{1}.
\fi % End of section 2 (sect. 1.1, p. 4a)

\WM3. This is the case of \WCD{ \|x} less than $10$ where the series is:
\[
F_\nu (x)  = \frac{1}{2} \exp(-x) \sum_{i=0}^{\infty}
   \frac{\Gamma (\nu + \frac{1}{2} )}
   {\Gamma (\nu + i + \frac{3}{2} ) }
    x^i
\]

The series is summed by using a recursion relationship
between succesive terms in the series. The sum is accumulated
in \WCD{ \>{partialsum}} and the terms are obtained from previous ones
by the ratio \WCD{ \|x\WSl\|a} with \WCD{ \|a} incremented by $1$ for
each new \WCD{ \>{term}}.

\WY\WP\4\4\WX{\M{3}}Small x Case\X \X${}\WSQ{}$\6
$\|a=\|a+\>{half};$\6
$\>{term}=\>{one}\WSl\|a;$\7
\WC{ Remember that \WCD{ \|a} is $\nu$ }\7
$\>{partialsum}=\>{term};$\6
$\&{for}\,(\|i=\WO{2};$ $\|i\WL\>{maxone};$ $\|i=\|i+\WO{1})$ \1\6
$\{$\6
$\|a=\|a+\>{one};$\6
$\>{term}=\>{term}\ast\|x\WSl\|a;$\6
$\>{partialsum}=\>{partialsum}+\>{term};$\5
\WC{ Here is the possibility of convergence }\6
$\&{if}\,(\>{term}\WSl\>{partialsum}<\>{crit})$\1\6
\&{break};\2\6
$\}$\2\7
\WC{ If the upper summation limit is reached, it means that the series has
failed to converge, so say so and exit.  }\6
$\&{if}\,(\|i\WS\>{maxone})$\1\6
$\{$\6
$\&{write}\,(\WUC{ERROR\_OUTPUT\_UNIT},\ \39\WO{200})$ ;\7
\4\WO{200}\Colon\ \6
$\&{format}\,(\.{'\ i\ >\ 50\ in\ fmch'})$ ;\6
\WUC{STOP};\6
$\}$\2\7
\WC{  Otherwise return silently  }\7
\&{return} $(\>{half}\ast\>{partialsum}\ast\|y);$\WY\Wendc
\WU section~\M{1}.
\fi % End of section 3 (sect. 1.2, p. 5)

\WM4. This is the other case for \WCD{ \|x} greater than ten:
\[
F_\nu (x)  =  \frac{\Gamma (\nu + \frac{1}{2} )}
          {2x^{\nu + \frac{1}{2} ) }}
- \frac{1}{2} \exp(-x) \sum_{i=0}^{\infty}
   \frac{\Gamma (\nu + \frac{1}{2} }
   {\Gamma (\nu - i + \frac{3}{2} ) }
    x^{-i}
\]
again, the terms in the sum are obtained by recursion from the
previous ones.

\WY\WP\4\4\WX{\M{4}}Large x Case\X \X${}\WSQ{}$\7
$\|b=\|a+\>{half};$\6
$\|a=\|a-\>{half};$\7
\WC{ Once again, remember that \WCD{ \|a} is $\nu$ }\6
$\>{xd}=\>{one}\WSl\|x;$\6
$\>{approx}=\>{rootpi4}\ast\@{dsqrt}(\>{xd})\ast\>{xd}\WEE{\|m};$\6
$\&{if}\,(\|m>\WO{0})$\1\6
$\{$\6
$\&{for}\,(\|i=\WO{1};$ $\|i\WL\|m;$ $\|i=\|i+\WO{1})$ \1\6
$\{$\6
$\|b=\|b-\>{one};$\6
$\>{approx}=\>{approx}\ast\|b;$\6
$\}$\2\6
$\}$\2\6
$\>{fimult}=\>{half}\ast\|y\ast\>{xd};$\6
$\>{partialsum}=\>{zero};$\7
\WC{ Take care of special case }\7
$\&{if}\,(\>{fimult}\WS\>{zero})$\1\6
\&{return} $(\>{approx});$\2\7
\WC{ Otherwise continue  }\7
$\>{fiprop}=\>{fimult}\WSl\>{approx};$\6
$\>{term}=\>{one};$\6
$\>{partialsum}=\>{term};$\6
$\>{numberofterms}=\>{maxtwo};$\6
$\&{for}\,(\|i=\WO{2};$ $\|i\WL\>{numberofterms};$ $\|i=\|i+\WO{1})$ \1\6
$\{$\6
$\>{term}=\>{term}\ast\|a\ast\>{xd};$\6
$\>{partialsum}=\>{partialsum}+\>{term};$\7
\WC{ See if the criterion is satisfied, return silently if so }\7
$\&{if}\,(\@{dabs}(\>{term}\ast\>{fiprop}\WSl\>{partialsum})\WL\>{crit})$\1\6
\&{return} $(\>{approx}-\>{fimult}\ast\>{partialsum});$\2\6
$\|a=\|a-\>{one};$\5
\WC{ or carry on  }\6
$\}$\2\7
\WC{ If \WCD{ \|i} gets as far as \WCD{ \>{numberofterms}} then the expansion
has failed, print a message and quit. }\7
$\&{write}\,(\WUC{ERROR\_OUTPUT\_UNIT},\ \39\WO{201})$ ;\7
\4\WO{201}\Colon\ \6
$\&{format}\,(\.{'\ \ numberofterms\ reached\ i\0n\ fmch'})$ ;\6
\WUC{STOP};\5
\WC{ no convergence }\6
$\}$\WY\Wendc
\WU section~\M{1}.
\fi % End of section 4 (sect. 1.3, p. 6)

\WN5.  INDEX.

\fi % End of section 5 (sect. 2, p. 7)

\input INDEX.tex
\input MODULES.tex

\Winfo{"fweave fmch.web"}  {"fmch.web"} {(none)}
 {\Ratfor}


\Wcon{5}
\FWEBend
